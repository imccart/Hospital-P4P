\documentclass[t]{beamer}
\usetheme[progressbar=frametitle]{metropolis}
\usepackage{appendixnumberbeamer}

\usepackage{booktabs}
\usepackage[scale=2]{ccicons}

\usepackage{graphics,graphicx,amssymb,amsmath,pgf,comment,hyperref}
%\usepackage[xcolor=pst]{pstricks}
\usepackage{array}
\usepackage{pgfshade}
\usepackage[round]{natbib}
\usepackage[absolute,overlay]{textpos}
\usepackage{pifont}
\usepackage{dcolumn}
\usepackage{textpos}
\usepackage{color}					
\usepackage{xcolor,colortbl}
\usepackage{tikz}
\usepackage{bbm}
\usepackage{curves}
\usepackage{mathtools}
\usetikzlibrary{snakes,arrows,positioning}
\def\augie{\fontencoding{T1}\fontfamily{augie}\selectfont}

\usepackage{pgfplots}
\usepgfplotslibrary{dateplot}

\usepackage{xspace}
\newcommand{\themename}{\textbf{\textsc{metropolis}}\xspace}

\setbeamertemplate{caption}{\raggedright\insertcaption\par}
\usetikzlibrary{calc,decorations.pathmorphing,patterns}
\pgfdeclaredecoration{penciline}{initial}{
    \state{initial}[width=+\pgfdecoratedinputsegmentremainingdistance,
    auto corner on length=1mm,]{
        \pgfpathcurveto%
        {% From
            \pgfqpoint{\pgfdecoratedinputsegmentremainingdistance}
                      {\pgfdecorationsegmentamplitude}
        }
        {%  Control 1
        \pgfmathrand
        \pgfpointadd{\pgfqpoint{\pgfdecoratedinputsegmentremainingdistance}{0pt}}
                    {\pgfqpoint{-\pgfdecorationsegmentaspect
                     \pgfdecoratedinputsegmentremainingdistance}%
                               {\pgfmathresult\pgfdecorationsegmentamplitude}
                    }
        }
        {%TO
        \pgfpointadd{\pgfpointdecoratedinputsegmentlast}{\pgfpoint{1pt}{1pt}}
        }
    }
    \state{final}{}
}


\title{Hospital Pricing and Public Payments}
\date{July 26, 2018}
\author{Michael Darden, \textbf{Ian McCarthy}, and Eric Barrette}
\institute{2018 NBER Summer Institute, Health Care}

\begin{document}

\maketitle

\section{Motivation}
\begin{frame}{Main Question}
    \only<1->{
        How do hospital prices change following reductions in Medicare payments?
    }
    \only<2>{
        \begin{enumerate}
            \item Standard two-price market $\rightarrow$ price decreases
            \item Dynamic cost-shifting $\rightarrow$ price increases
        \end{enumerate}
    }
\end{frame}

\begin{frame}{Difficult to identify cost-shifting}
    \begin{itemize}
        \item Poor pricing data
        \item Different sources of public payment reductions
        \item Different magnitudes of public payment reductions
    \end{itemize}
\end{frame}

\begin{frame}{Our approach}
    \begin{itemize}
        \item Examine negotiated prices from HCCI
        \item Exploit payment changes from HRRP and HVBP
        \item Variation in penalties at both intensive and extensive margin
    \end{itemize}
\end{frame}


\section{Institutional Background}
\begin{frame}{How are hospital prices negotiated?}
    \begin{itemize}
        \item Often 3 year contracts
        \item Negotiated as \% of charge or markup over Medicare
        \item Some carve-out and stop-loss provisions
        \item Negotiations usually relatively broad (for given insurer)
    \end{itemize}
\end{frame}

\begin{frame}{Hospital Readmission Reduction Program}
    \begin{itemize}
        \item Initiated FY 2013 (October 2012)
        \item Penalty for ``excess'' risk-adjusted readmission rates for selected categories
        \item FY 2013 penalty from data in 2009-2011
        \item Penalties applied to base payments on \textbf{all} Medicare inpatient admissions
    \end{itemize}
\end{frame}

\begin{frame}{Hospital Value Based Purchasing}
    \begin{itemize}
        \item Initiated FY 2013 (October 2012)
        \item Penalty or reward based on performance in several measures
        \item FY 2013 penalty/bonus from data in 2009-2011
    \end{itemize}
\end{frame}


\section{Empirical Approach}
\begin{frame}{Data Sources}
    \begin{itemize}
        \item Health Care Cost Institute
        \item Hospital Compare
        \item American Community Survey
        \item American Hospital Association (AHA) annual surveys
        \item Healthcare Cost Report Information System (HCRIS)
    \end{itemize}
\end{frame}

\begin{frame}{Dataset}
    1,386 inpatient prospective payment system hospitals from 2010 to 2015:
    \begin{itemize}
        \item Drop smaller hospitals and those without sufficient history (such that HRRP and HVBP don't apply)
        \item Focus on acute care admissions
        \item Drop all transfer admissions and those in which the patient traveled more than 180 miles
        \item Claims with incomplete data - likely evidence of procedural errors - are dropped
        \item Claims with a payment ratio below the 5th percentile and above the 95th percentile were excluded
    \end{itemize}
\end{frame}

\begin{frame}{Initial Specification}
    \only<1>{
        \begin{equation*}
            y_{ht} = \alpha_{h} + \beta x_{ht} +  \gamma Z_{ct} + \delta 1[Penalty]  + \theta_{t}  +  \epsilon_{ht}
        \end{equation*}
    }
    \only<2>{
        \begin{equation*}
            y_{ht} = \alpha_{h} + \beta x_{ht} +  \gamma Z_{ct} + \delta 1[Penalty]  + \theta_{t}  +  \epsilon_{ht}
        \end{equation*}
        \begin{itemize}
            \item[] $y_{ht} =$ outcome for hospital $h$ in year $t$
            \item[] $\alpha_{h}=$ hospital fixed effect
            \item[] $x_{ht}=$ time-varying hospital characteristics
            \item[] $Z_{ct}=$ time-varying county characteristics
            \item[] $\theta_{t}=$ year fixed effect
        \end{itemize}
        $1[Penalty]$  penalty variable is zero in years 2010 and 2011 for all hospitals.
    }
    \only<3>{
         \begin{equation*}
            \textcolor{red}{y_{ht}} = \alpha_{h} + \beta x_{ht} +  \gamma Z_{ct} + \delta \textcolor{red}{1[Penalty]}  + \theta_{t}  +  \epsilon_{ht}
        \end{equation*}
        \begin{table}[htb!]
        \centering
        \footnotesize
        \centerline{
        \begin{tabular}{cccc}
            \hline\hline
            Fiscal & Sample 		&  Payment $\$$			& Percent \\
            Year   &  Size    		&  Mean (St. Dev.) 			& Penalized \\
            \hline
            2010 &      1,386		& 	10,729   (4,937)	& 0.00  \\
            2011 &      1,386		& 	11,603   (5,076)	& 0.00   \\
            2012 & 	1,386 		& 	12,079   (5,477) 	& 0.32   \\
            2013 & 	1,386		& 	12,668   (5,568)	& 0.74  \\
            2014 & 	1,386		&	12,796   (5,444)	& 0.76 \\
            2015 &     1,386		& 	13,398   (5,922)	& 0.79 \\
            \hline
            Total & 	8,316		& 12,212   (5,482)	&	 0.43 \\
            \hline
        \end{tabular}}
        \end{table}
    }
\end{frame}

\section{Results}
\begin{frame}{Outline}
    \only<1>{
    \begin{enumerate}
        \item Fixed effects estimates
        \item Alternative specifications and controls
        \item Extensive vs intensive margins
        \item Heterogeneities in effects (by objective function, organizational structure, market power)
        \item Other explanations
    \end{enumerate}
    }
    \only<2>{
    \begin{enumerate}
        \item \textcolor{red}{Fixed effects estimates}
        \item \textcolor{red}{Alternative specifications and controls}
        \item Extensive vs intensive margins
        \item Heterogeneities in effects (by objective function, organizational structure, market power)
        \item \textcolor{red}{Other explanations}
    \end{enumerate}
    }
\end{frame}

\begin{frame}{Initial Results}
    \only<1>{
        \begin{table}[htb!]
        \centering
        \footnotesize
        \centerline{
        \begin{tabular}{rrrrr}
            \hline
 			    Log Mean		& Log Mean          &  Log Medicaid 	   	& Log Medicare   		& Log Other  			\\
			    Payment			& Net Charge        &  Discharges          	& Discharges       		& Discharges        	\\
            \hline\hline
                0.014***		&	0.008        &   -0.045**	  &	-0.027***	  &	-0.004	\\
	               (0.005)		&	(0.008)      &   (0.021)	  &	(0.007)	      &	(0.011)
        \end{tabular}}
        \end{table}
    }
    \only<2>{
        \begin{table}[htb!]
        \centering
        \footnotesize
        \centerline{
        \begin{tabular}{rrrrr}
            \hline	
 			    Log Mean		& Log Mean          & Log Medicaid 	   	& Log Medicare   		& Log Other  			\\
			    Payment			& Net Charge        & Discharges      	& Discharges       		& Discharges        	\\
            \hline\hline
                0.014***		&	0.008        &   -0.045**	  &	-0.027***	  &	-0.004	\\
	               (0.005)		&	(0.008)      &   (0.021)	  &	(0.007)	      &	(0.011) \\
            \hline
            \multicolumn{5}{l}{Differential Trends} \\
            \hline
                0.010**	&	0.019**	&	-0.038	&	-0.026***	&	-0.011	\\
	           (0.005)	&	(0.008)	&	(0.023)	&	(0.007)	    &	(0.012)	\\ \relax
                [0.497] &  [0.041]  &  [0.250]   &  [0.005]     & [0.446]     \\
        \end{tabular}}
        \end{table}
    }
    \only<3>{
        \begin{table}[htb!]
        \centering
        \footnotesize
        \centerline{
        \begin{tabular}{rrrrr}
            \hline	
 			    Log Mean		& Log Mean          & Log Medicaid 	   	& Log Medicare   		& Log Other  			\\
			    Payment			& Net Charge        & Discharges      	& Discharges       		& Discharges        	\\
            \hline\hline
                0.014***		&	0.008        &   -0.045**	  &	-0.027***	  &	-0.004	\\
	               (0.005)		&	(0.008)      &   (0.021)	  &	(0.007)	      &	(0.011)	\\
            \hline
            \multicolumn{5}{l}{Adding County Fixed Effects} \\
            \hline
                0.015***	&	0.009	&	-0.048**	&	-0.027***	&	-0.003	\\
	            (0.005)	&	(0.008)	&	(0.022)	&	(0.007)	&	(0.011)	\\
        \end{tabular}}
        \end{table}
    }
    \only<4>{
        \begin{table}[htb!]
        \centering
        \footnotesize
        \centerline{
        \begin{tabular}{rrrrr}
            \hline	
 			    Log Mean		& Log Mean          & Log Medicaid 	   	& Log Medicare   		& Log Other  			\\
			    Payment			& Net Charge        & Discharges      	& Discharges       		& Discharges        	\\
            \hline\hline
                0.014***		&	0.008        &   -0.045**	  &	-0.027***	  &	-0.004	\\
	               (0.005)		&	(0.008)      &   (0.021)	  &	(0.007)	      &	(0.011)	\\
            \hline
            \multicolumn{5}{l}{Controlling for Medicaid Expansion} \\
            \hline
                0.014***	&	0.008	&	-0.044**	&	-0.027***	&	-0.005	\\
	           (0.005)	&	(0.008)	&	(0.021)	&	(0.007)	&	(0.010)	\\
        \end{tabular}}
        \end{table}
    }
    \only<5>{
        \begin{table}[htb!]
        \centering
        \footnotesize
        \centerline{
        \begin{tabular}{rrrrr}
            \hline	
 			    Log Mean		& Log Mean          & Log Medicaid 	   	& Log Medicare   		& Log Other  			\\
			    Payment			& Net Charge        & Discharges      	& Discharges       		& Discharges        	\\
            \hline\hline
                0.014***		&	0.008        &   -0.045**	  &	-0.027***	  &	-0.004	\\
	               (0.005)		&	(0.008)      &   (0.021)	  &	(0.007)	      &	(0.011)	\\
            \hline
            \multicolumn{5}{l}{Controlling for HCAHPS Overall Rating} \\
            \hline
                0.014***	&	0.008	&	-0.045**	&	-0.026***	&	-0.003	\\
	           (0.005)	&	(0.008)	&	(0.021)	&	(0.007)	&	(0.010)	\\
        \end{tabular}}
        \end{table}
    }
    \only<6>{
        \begin{table}[htb!]
        \centering
        \footnotesize
        \centerline{
        \begin{tabular}{rrrrr}
            \hline	
 			    Log Mean		& Log Mean          & Log Medicaid 	   	& Log Medicare   		& Log Other  			\\
			    Payment			& Net Charge        & Discharges      	& Discharges       		& Discharges        	\\
            \hline\hline
                0.014***		&	0.008        &   -0.045**	  &	-0.027***	  &	-0.004	\\
	               (0.005)		&	(0.008)      &   (0.021)	  &	(0.007)	      &	(0.011)	\\
            \hline
            \multicolumn{5}{l}{Dropping FY 2012} \\
            \hline
                0.012**	&	0.010	&	-0.045*	&	-0.028***	&	-0.007	\\
		        (0.005)	&	(0.009)		&(0.023)		&(0.007)	&	(0.012)	\\
        \end{tabular}}
        \end{table}
    }
\end{frame}

\begin{frame}{Other Mechanisms}
    \only<1>{
        \metroset{block=fill}
        \begin{block}{1. Quality increased}
            \begin{itemize}
                \item Gupta \textit{et al.} (2017) - HRRP $\rightarrow$ readmission reduction but mortality increase (Medicare only)
                \item Gupta (2016) - HRRP $\rightarrow$ slight reduction in mortality (Medicare only)
                \item Ibrahim \textit{et al.} (2016) - readmission reductions largely coding changes
                \item Demiralp \textit{et al.} (2018) - HRRP $\rightarrow$ readmission reduction for Medicare but no change for private insurance
                \item No effects of HVBP across several studies
                \item Economically small and statistically insignificant effect on private insurance readmissions in our data
            \end{itemize}
        \end{block}
    }

    \only<2>{
        \metroset{block=fill}
        \begin{block}{2. Shift toward more profitable services}
            \begin{itemize}
                \item Construct ``profitable services index'' following services identified in Horwitz and Nichols (2009)
                \item Economically small and statistically insignificant effects on types of services offered (on the extensive margin)
            \end{itemize}
        \end{block}
    }

    \only<3>{
        \metroset{block=fill}
        \begin{block}{3. Increase in the intensity of treatment}
        Economically small and statistically insignificant effects on:
            \begin{itemize}
                \item Length of stay
                \item DRG weights
            \end{itemize}
        \end{block}
    }

    \only<4>{
        \metroset{block=fill}
        \begin{block}{4. Other costly investments}
            Economically small and statistically insignificant effect on costs per discharge (from HCRIS reports)
        \end{block}
    }
\end{frame}

\begin{frame}{How could this happen?}
    \begin{enumerate}
        \item Hospital objective function and risk aversion
        \item Informal negotiation process
        \item Insurer allows higher price to maintain competition (perhaps for specific service lines)
    \end{enumerate}
\end{frame}

\begin{frame}{Summary of Results}
    \begin{itemize}
        \item Unique data on hospital pricing with plausibly exogenous changes in Medicare payments
        \item Cross-sectional and longitudinal variation in penalties on extensive and intensive margins
        \item Robust finding of significant increase in prices of around 1.4\% among penalized hospitals
    \end{itemize}
\end{frame}

\begin{frame}{Implications for P4P}
    \begin{itemize}
        \item Does \textbf{not} imply all pay for performance plans are useless
        \item HRRP/HVBP are relatively blunt instruments that may not reflect a true quality signal or new information to the market
    \end{itemize}
\end{frame}


\end{document}






