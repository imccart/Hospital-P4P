\documentclass[12pt]{article}
\usepackage{graphicx,amssymb,amsmath,setspace,comment,verbatim,titling,pgf,lscape}
\usepackage[left=1in,right=1in,top=1.5in,bottom=1.5in]{geometry}
\usepackage[round]{natbib}
\usepackage{hyperref}
\usepackage{array}
\usepackage{bbm}
\usepackage{csquotes}
\usepackage{setspace}

\usepackage{float}

\usepackage[justification=centering]{caption}
\usepackage[labelsep=period]{caption}
\captionsetup[table]{name=Table}
%\renewcommand{\thetable}{\Roman{table}}

\captionsetup[figure]{name=Figure}
%\renewcommand{\thefigure}{\Roman{figure}}


\newcommand{\deriv}[2]{\frac{\mathrm{d}#1}{\mathrm{d}#2}}
%%\usepackage{breqn}
\newcommand{\pderiv}[2]{\frac{\partial#1}{\partial#2}}
%\usepackage{siunitx}
\newcolumntype{P}[1]{>{\raggedright\arraybackslash}p{#1}}
\hypersetup{colorlinks,%
						citecolor=black,%
						filecolor=black,%
						linkcolor=black,%
						urlcolor=blue,%
						}


\setlength{\droptitle}{-50pt}

\begin{document}

\title{Hospital Pricing and Public Payments \thanks{We received no funding for this work.  We report no conflicts of interest. We thank Jonathan Ketcham, Michael Richards, Jason Hockenberry, Bryan Dowd, David Dranove, Craig Garthwaite, and seminar participants at Johns Hopkins University, The Southeastern Health Economics Study Group, The Midwest Health Economics Conference, The National Bureau of Economic Research Summer Institute, George Washington University, The Carey School of Business at Johns Hopkins University, and the Federal Trade Commission. Finally, we acknowledge the Health Care Cost Institute (HCCI), along with companies providing data (Aetna, Humana, and UnitedHealthcare) used in this analysis. Michael Darden (mdarden4@jhu.edu); Ian McCarthy (ian.mccarthy@emory.edu); Eric Barrette (eric.barrette@gmail.com)}}
\author{%
  Michael Darden \\[-0.5ex]
  Johns Hopkins University \& NBER \\
  Ian M. McCarthy \\[-0.5ex]
  Emory University \& NBER \\
  Eric Barrette \\[-0.5ex]
  Medtronic
}
\date{December 2018}

\maketitle

\begin{abstract}
A longstanding debate in health economics and health policy concerns how hospitals adjust prices with private insurers following reductions in public funding. We enter this debate by examining actual private-payer payments from a large, multi-payer database merged to plausibly exogenous variation in Medicare payment rates generated by the Affordable Care Act.  We find that hospitals facing net payment reductions from Medicare were able to negotiate 1.4\% higher average private payments - approximately \$183,700 per hospital, based on an average relative reduction in Medicare payments of \$271,000.  We find no evidence that our results are driven by quality improvements or changes in hospital services among private insurance patients.
\end{abstract}
\noindent \textit{JEL Classification:} I11; I18; L2 \\\\
\noindent \textit{Keywords:} Hospital Bargaining; Healthcare Prices; Medicare Payments; Cost-Shifting; Affordable Care Act\\\\

\newpage
\section{Introduction}
\onehalfspacing

In the United States, health care prices are frequently cited as a root cause for the dramatic difference in per capita national health expenditures relative to other developed countries, but understanding how health care providers negotiate payments with private payers is difficult in a mixed public/private reimbursement system \citep{anderson2003}.\footnote{Throughout, rather than use the term ``price,'' we refer to the financial transfer for a given procedure as the ``payment'' from a private insurance firm to a hospital. A payment is distinctly different than a hospital ``charge,'' which effectively represents a hospital's list price for a give procedure. Private insurance firms negotiate substantial discounts from charges.}  Not surprisingly, a longstanding debate in health economics concerns the role of public payment rates in the negotiation over payments between hospitals and private insurers.  Standard economic theory predicts that a reduction in the rate of public payments should put \textit{downward pressure} on private payer payments as hospitals attempt to attract a larger share of private insurance patients.  An alternative argument, termed ``cost-shifting'' and formalized in \cite{dranove1988}, is that providers pass public payment cuts to privately insured patients by negotiating for higher payments from private insurance companies.  Empirically identifying whether and to what extent private payer payments are affected by public payment changes has proven difficult, in-part because precise private payer data are typically unobserved.  Accurately measuring these effects has clear policy implications and can inform as to the full effect of proposed payment changes in large public programs such as Medicare and Medicaid.

We enter this debate with a compelling dataset on actual payments from private insurance firms to hospitals.  Our data, maintained by the Health Care Cost Institute (HCCI), contain all hospital inpatient claims to three national commercial insurers.\footnote{\cite{cooper2017} also use HCCI data to examine broad trends in hospital pricing from 2007 through 2011.}  These unique data include payments for every claim, which capture the negotiated payments between hospitals and insurers and which may differ substantially from charge-based estimates of payments often used in the literature \citep{dafny2009,dranove2017}. Our data cover approximately 28$\%$ of individuals under the age of 65 who have employer-sponsored insurance (ESI). When merged with several other datasets on hospital and county characteristics, our final analytic data constitute a balanced panel of 50$\%$ of all inpatient prospective payment hospitals in the U.S. between 2010 and 2015.

In this paper, we exploit the 2013 adoption of the Hospital Readmission Reduction Program (HRRP) and Hospital Value-Based Purchasing Program (HVBP) as sources of variation in Medicare payment rates to study how changes in public payments affect the negotiated payments between providers and private payers.  Under these programs, hospitals were penalized (or potentially rewarded under the HVBP) by up to 3$\%$ of the hospital's total Medicare revenues based on observed quality metrics.\footnote{Some contracts explicitly tie private payments to Medicare reimbursement rates \citep{cooper2017}, but in our context, there is no change in the prospective payment but rather a downward adjustment of some percentage. This means that our analysis can identify the effect on commercial insurance payments that are distinct from any mechanical contractual negotiation, since the prospective Medicare payment on which the private insurance contract may be negotiated remains unchanged.}  Since penalty amounts vary from year to year, and because not all hospitals are penalized, HRRP/HVBP generate both cross-sectional and temporal variation in Medicare payments.  Our baseline empirical specification is a hospital fixed effects estimator in which we estimate the difference in average payments between those hospitals with a net penalty under the HRRP/HVBP relative to those not penalized, discussed in detail in Section \ref{sec:initial}. Our baseline results reveal an increase in average payments of 1.4$\%$ for penalized hospitals, equivalent to a $\$$167 increase in the average private payer payment from 2013 through 2015. We also find evidence that penalty size matters with respect to payment changes, with a 2.4$\%$ increase in payments for the most heavily penalized hospitals relative to those hospitals receiving no penalty or a bonus. As a back-of-the-envelope calculation, our estimated increase of 1.4\% equates to a total increase in private payments of \$183,700 per hospital, based on an average relative reduction in Medicare payments of \$271,000.\footnote{The total relative reduction in Medicare payments incorporates bonus payments made to some hospitals, such that the relative reduction is larger than the average penalty amount.}

We argue that two central features of the HRRP/HVBP allow for a causal interpretation of our findings. First, the institutional details of the HRRP/HVBP are such that hospitals had little, if any, opportunity to adjust their penalties \textit{ex ante}. In particular, penalties from the HRRP/HVBP were calculated based on lagged quality, such that penalties incurred in Fiscal Year (FY) 2013 are based on claims from July 2008 through June 2011. The set of quality metrics underlying the penalty formulas also changed over time. For example, the set of conditions covered by the HRRP/HVBP increased in FY 2015, but the new conditions were not announced until FY 2014, at which point the data underlying the new conditions were already collected.

Second, there is evidence that hospital performance under HRRP/HVBP does not reveal new information on the underlying quality of a hospital. Part of this claim again derives from the timing of the data collection and penalty calculation, wherein any deficiency in quality should be known to private insurers prior to the final calculations from the Centers for Medicare and Medicaid Services (CMS). Indeed, we show that readmission rates among penalized hospitals was persistently low even before the introduction of HRRP/HVBP. Moreover, recent studies document substantial idiosyncracies in HRRP/HVBP penalty designations, such that whether hospitals are penalized in a given year is predominately noise or due to poor risk-adjustment methods from CMS \citep{friedson2016,wilcock2018}.

While we argue that HRRP/HVBP penalties are exogenous over the short term, we also acknowledge that our time period of study was one of significant change in the U.S. health care system, and there are likely unobserved, time-varying factors correlated with penalty status and also correlated with private insurer payments. To address these concerns empirically, we consider a series of alternative specifications in Section \ref{sec:robust}. First, we demonstrate that allowing trends in payments to vary by whether a hospital ever experiences an HRRP/HVBP payment reduction does not change our conclusions. We also show that our results are not driven by regional differences, the 2014 Medicaid expansion, changes in patient severity mix, changes in overall hospital quality, or the inclusion/exclusion of FY2012 (when the HRRP/HVBP only affected a subset of hospitals). 

Collectively, our initial estimates and alternative specifications provide support for the notion that hospitals can negotiate higher commercial insurance payments following a reduction in Medicare payments; however, there exist several observationally-equivalent mechanisms consistent with this result. The first is that hospitals penalized under HRRP/HVBP subsequently improve the quality of care among their private insurance patients, for which private insurers are willing to increase payments.\footnote{We emphasize that in the context of bilateral bargaining between a private insurer and a hospital, the connection between quality and payments should be with respect to the quality of care received by the relevant \textit{privately} insured population. Using data from Florida and California, \cite{demiralp2017} do not find that the HRRP generated any improvements in readmissions among the non-Medicare population.}  Another potential mechanism is that penalized hospitals change the intensity of their services such that total payments from private insurers increases while payments for identical procedures remains constant. We consider each of these mechanisms in detail in Section \ref{sec:alt}, and while it is inherently difficult to decompose changes in spending between quality, intensity, and prices \citep{garthwaite2018}, we generally find no evidence that quality or intensity of care increased for private insurance patients at penalized hospitals.

Without a change in underlying quality or intensity, another explanation for our estimated effects is that payments increased for otherwise identical services (i.e., cost-shifting). Our analysis then proceeds with a simple theoretical extension of the cost-shifting model in \cite{dranove1988} to the context of hospital-insurer bargaining, where we develop a series of empirically testable predictions implied by a theoretical model of hospital cost-shifting as well as the institutional details of the HRRP/HVBP. As discussed in detail in Section \ref{sec:Ext}, we find that hospitals appear to behave in a way that is consistent with cost-shifting.

Our findings therefore contribute to the large literature on hospital cost-shifting.\footnote{We use cost-shifting in the usual ``dynamic'' sense, reflecting a change in private payments following a change in public payments. As emphasized in \cite{hay1983}, higher payments from private payers at a given point in time may simply be evidence of price discrimination, which is reasonable in a era of increasing hospital concentration \citep{gaynor2015jel}.} This argument is ubiquitous in health policy debates and often cited by insurers, hospitals, and policy makers. For example, in the debate over the Affordable Care Act, President Obama said:\footnote{For additional examples, see the many excerpts in \cite{dranove2017}, including additional statements from President Obama and the U.S. Supreme Court regarding the Affordable Care Act.}
\begin{quote}
\textit{``You and I are both paying 900 bucks on average - our families - in higher premiums because of uncompensated care.''} - Barack Obama
\end{quote}
There is also some existing empirical evidence consistent with hospital cost-shifting.\footnote{See \cite{zwanziger2006}, \cite{lee2003}, \cite{zwanziger2000}, and \cite{cutler1998costshift} for prominent examples.}  On the other hand, more recent studies have not found strong evidence of hospital cost-shifting \citep{white2013, dranove1998}.\footnote{See \citet{frakt2011} for a review of the evidence on cost-shifting.}  We argue that much of the ambiguity in the literature is due to measurement error in charge-based proxies for private payments.  Indeed, in our data, the correlation between actual payments and a payment proxy estimated from the Healthcare Cost Report Information System (HCRIS) is 0.435, suggesting that charge-based estimates of payments may contain significant measurement error.\footnote{Notable exceptions include \cite{clemens2017}, who study the market for physician services and find that private payments decreased following a reduction in Medicare payment rates, and \cite{dranove2017}, who find little evidence of changes in private payments to hospitals following the 2008 stock market collapse.} With payment data on a balanced panel of hospitals from 2010-2015, our analysis demonstrates that hospital fixed effects may not adequately control for the mean payment-to-charge ratio within a hospital in a model of log charges \citep{cutler2000}.

Our paper also contributes to a theoretical literature on the hospital objective function. Because theoretical predictions suggest that profit-motivated (and risk-neutral) firms are unlikely to cost-shift, \cite{dranove1988} models a hospital as a utility maximizer, where utility is defined over both profit and quantity.  A hospital may directly value quantity of care for reasons of altruism or prestige, or simply because a non-profit hospital must provide some form of ``community benefit'' to maintain its tax-exempt status.\footnote{The Congressional Budget Office defines community benefits as services geared toward ``promoting the health of any broad class of persons'' \citep{cbo2006}.}  That paper argues that cost-shifting should be isolated to non-profit hospitals. We embed the model of \cite{dranove1988} within a bargaining model \citep{ho2017} and show that cost-shifting can be predicted for any hospital with diminishing marginal utility of profits. A risk-averse hospital, for example, would still be predicted to cost-shift even if they are otherwise for-profit. The model would similarly predict cost-shifting for a hospital with some overall target profit margin and for whom the margin drops below the target margin due to the penalty. When we break our empirical analysis by profit status, we find no economically meaningful difference in effects among for-profit and non-profit hospitals, although our estimates are only significantly different than zero in the case of non-profits.  Regardless of the underlying hospital objective function, recent evidence from \citet{sacarny2018} suggests that frictions in hospital-physician integration often prevent hospitals from immediately capturing revenue made available by Medicare. Therefore, hospitals may ``leave money on the table'' in the short- to medium-term simply due to rigidities in the payment renegotiation process and the complexity of the environment. We discuss these issues in more detail in Section \ref{sec:Ext}.

\section{Background}
\label{sec:Background}

\subsection{The HRRP and HVBP Policies}
The adoption of the Medicare prospective payment system (PPS) in 1983, in which Medicare payments changed from pure fee-for-service to a capitated amount for each inpatient stay depending on diagnosis, generated incentives for hospitals to cut ``excessive'' procedures. PPS also created incentives for hospitals to discharge patients quickly.  By 2011, Medicare paid $\$$24 billion per year for 1.8 million hospital \textit{readmissions} -- admissions to any hospital within 30-days of discharge for the same condition.  While some readmissions are unavoidable, the HRRP was a cost containment in the ACA designed to levy penalties on hospitals with ``excessive'' readmissions.

Starting in FY 2013 (October 2012-September 2013), the HRRP penalized hospitals for which 30-day readmissions for acute myocardial infarction (AMI), heart failure (HF), and pneumonia (PN) exceeded risk-adjusted thresholds constructed as a function of national averages. Recall that this assessment was based on data collected from July 2008 through June 2011. In this first year of the program, hospitals faced a maximum cut in Medicare payments of 1\% across all DRGs. In FY 2015, the maximum penalty increased to 3\%, total penalties rose to \$420m \citep{rau2015}, and applicable conditions were expanded to include chronic obstructive pulmonary disease (COPD) and total hip and knee replacements. The \cite{cbo2010} estimates that HRRP would reduce hospital payments from Medicare by \$113 billion through 2019.

By contrast, the HVBP program is rooted in a standard principal-agent model in which the principal (CMS in this case) contracts with agents (hospitals) to provide quality care to Medicare enrollees. The HVBP program scores hospitals based on their achievement (comparison to other hospitals) as well as their improvement (comparison to their own previous performance).  Similar to the HRRP, the HVBP bases changes in payments on past quality, with data collected over the same lagged time period as in the HRRP. However, unlike the HRRP, the HVBP program is funded by reducing all hospitals' base operating Medicare severity diagnosis-related group (MS-DRG) payments and creating rebate incentives depending on defined quality metrics. The percentage reduction increased annually up to 2\%. The program defines several quality domains and converts measures of quality within each domain to points, which are aggregated and mapped to a total point score.  The total point score determines the magnitude of the payment change, which may be positive or negative depending on if a hospital generates a rebate large enough to offset the reduction.

Since the goal of both the HRRP and HVBP is to improve hospital quality, a recent literature examines the effects of the HRRP/HVBP on hospital readmission rates and other quality metrics. We discuss this literature in the context of our analysis of hospital quality in Section \ref{sec:alt}.

\subsection{Debate over Hospital Responses to Payment Reductions}
The debate over whether, and the extent to which, hospitals can shift reductions in public payments onto private insurers has been ongoing for decades. While private insurers are naturally averse to higher private prices, hospitals have emphasized the need to cost-shift in an attempt to lobby for larger public payments. For example:
\begin{quote}
\textit{``Cost shifts have been a fact of hospital financial survival for decades.... The data show ...  how private payment is a mirror image of public payment over time and that the cost shift occurs. Hospitals must make up for shortfalls through a combination of approaches and cost-shifting is among them.'' -Rich Umbdenstock, Former President and CEO of American Hospital Association}\footnote{\href{http://blog.aha.org/post/costshifting-in-hospitals-}{``Cost-shifting in Hospitals,'' American Hospital Association Blog Post (AHA STAT), March 26, 2015.}}
\end{quote}

Indeed, the argument that cost-shifting occurs is easily motivated by observed trends.  In 2015, 55 million Americans were enrolled in Medicare, up from 37.5 million in 1995, and from 1980 to 2014, the share of hospital costs attributable to Medicare rose from 34.6$\%$ to 40.2$\%$.  Meanwhile, in 2014, hospitals endured a shortfall of $\$$35 billion in Medicare payments relative to Medicare patient costs, as compared with a $\$$5 billion surplus relative to costs in 1997.  During this same period, patients insured by private payers became increasingly lucrative: in 2014, the payment-to-cost ratio of privately insured patients was roughly 140$\%$.\footnote{All statistics from the American Hospital Association Trendwatch Chartbook, 2016} Consistent with the trend in profitability of private insurance patients to hospitals, average premiums for covered workers with family coverage more than tripled from 1999 through 2017.\footnote{\href{https://www.kff.org/interactive/premiums-and-worker-contributions/?coverageGroup=family}{``Premiums and Worker Contributions Among Workers Covered by Employer-Sponsored Coverage, 1999-2017,'' Kaiser Family Foundation.}}

While the current conditions for cost-shifting appear to be ripe, much of the evidence of significant cost-shifting comes from the 1980s and 1990s.  For example, \cite{cutler1998costshift} studies cost-shifting during the phase-in of Medicare prospective payments during the 1980s, which resulted in an average 2$\%$ per year reduction in Medicare payments.  He found evidence of dollar-for-dollar cost-shifting.  More recently, \cite{zwanziger2006} study the late 1990s and found that, between 1997 and 2001, cost-shifting was responsible for roughly 12$\%$ of the observed increase in total private payer prices.

In contrast, the simplest argument \textit{against} consistent cost-shifting as a significant mechanism in the hospital market is one of basic microeconomics.  A for-profit and risk-neutral firm with market power who sells to two groups should not respond to an exogenous decline in the price to one group by raising prices to the second group.  \cite{hay1983} shows that, even when the government commits to reimbursing the full average cost of Medicare patients, hospitals will: 1) still charge a higher price to privately insured patients; and 2) respond to lower Medicare payments with \textit{lower} private prices.\footnote{\cite{dor1996} demonstrate that payer-specific marginal costs may be evidence of differential treatment by hospitals.}

Despite the evidence presented by \cite{cutler1998costshift} and \cite{zwanziger2006}, the recent empirical evidence of cost-shifting is notably weak.\footnote{In a systematic review of the literature, \citet{frakt2011} concludes that cost-shifting, if it exists, is not widespread and is not a main driver of increased health care costs.}  Numerous studies have found zero or negative overall price effects, including \cite{dranove2008impact}, \cite{wu2010}, and \cite{dranove2017}, but potentially important positive price effects for certain subgroups.\footnote{See \cite{clemens2017} for a study of cost-shifting in the physician services market.} For example, \cite{wu2010} shows that hospitals with large shares of private patients (relative to Medicare patients) were able to cost-shift following the 1997 Balanced Budget Act, perhaps due to a stronger bargaining position.

We argue that identification of cost-shifting behavior is inherently difficult for three reasons.  First, the hospital market is incredibly complex.  In addition to many different types of payers, the industry is heavily regulated, and policy changes occur frequently.  We study two sources of public payment variation in which complexity is arguably a benefit to identification -- a common complaint among hospitals prior to the implementation of HRRP and HVBP was that the policies were opaque with respect to the payment reduction calculation.  Furthermore, because payment reductions were based on past quality metrics, hospitals could only hope to influence \textit{future} penalties once the programs were in place, and even then a hospital could only fully influence penalties four years into the future. Such a strategy is further complicated by the regular changes introduced to the programs vis-\`a-vis conditions and procedures being evaluated and the formulas by which HRRP/HVBP payments are calculated. Indeed, \cite{friedson2016} find that, within a relatively wide range of the HVBP penalty thresholds, whether a hospital ultimately underperforms in a given metric is largely random.

Second, measurement error in private payments may be severe. Because private payments are typically not observed, many of the referenced papers above must proxy for private payments, often with charges or costs, but we observe the actual dollar amount of payments from three large private insurers to hospitals.

Finally, heterogeneity in hospital responses to public payment cuts may muddle instances of important cost-shifting. Indeed, as we emphasize below, our bargaining model of hospital payments predicts that cost-shifting will be largest in markets in which a hospital has greater bargaining position. That is, the market power of a hospital in the provider market must be large relative to the market power of any given insurer in the insurance market.  While our data do not include good measures of insurance market concentration at a local level, we attempt to examine this heterogeneity with a series of alternative specifications and supplemental analyses by hospital payer mix and by different vertical relationships between physicians and hospitals.

\section{Empirical Analysis}
\label{sec:Empirical}

\subsection{Data}
Our primary data come from three large health insurance firms and account for roughly 28$\%$ of all individuals under the age of 65 with employer sponsored health insurance over the period of 2010 through 2015.  To these data, we merge information on HRRP and HVBP penalties/rewards and other cost information from the Healthcare Cost Report Information System (HCRIS); hospital-level characteristics such as bed count, for-profit status, and system membership from the American Hospital Association (AHA) annual surveys; data on a hospital's payer mix (i.e., the number and share of Medicare, Medicaid, or private insurance patients) also from HCRIS; and county-level demographic characteristics from the American Community Survey (ACS).  We restrict our sample to community hospitals in urban areas and in the contiguous United States, with at least 30 staffed beds, at least 25 admissions in a given year in the HCCI data, and observed HRRP/HVBP from HCRIS. Our final sample consists of 1,386 hospitals and 8,316 hospital/year observations.\footnote{We also consider alternative samples in which we allow for missing net penalty values from HCRIS or where we arbitrarily set missing HRRP/HVBP values to 0 (e.g., under the assumption that missing values indicate that the hospital was excluded for the program in that year). Results from these samples are provided in the supplemental appendix.}

Because hospital payments are often bundled across services, we follow \citet{gowrisankaran2015}, who use similar payment data from Northern Virginia, and aggregate payments to the hospital level by dividing the total payment for each claim by the appropriate DRG weight and regressing this amount on individual (claimant) characteristics and hospital fixed effects.  Using the estimated regression results, we predict the risk-adjusted mean hospital payment for a given year, which reflects the mean bargained payment. Table \ref{tab:summarystats} presents mean payments across hospitals over time. While average risk-adjusted payments received by hospitals increase roughly 5$\%$ annually between 2010 to 2015, shares of public (Medicare \& Medicaid) and private patients remain relatively stable over time. Importantly, while shares remain stable, within-hospital patient mix may vary considerably over time as a function of public payments, which is why we treat payer-specific discharges as a separate dependent variable. The last column of Table \ref{tab:summarystats} shows the fraction of hospitals subject to a net Medicare payment reduction. Note that the CMS fiscal year runs from October through the following September. Because of discrepancies between the fiscal year of the hospital and that of CMS, 32$\%$ of hospitals faced a penalty in their 2012 FY. By FY 2015, 79$\%$ of hospitals faced some payment reduction. Beginning FY 2013, the average penalty amount among hospitals ever penalized was \$204,711, which increased from \$171,279 in 2013 to \$272,438 in FY 2015. With non-penalized hospitals receiving an average bonus of just over \$66,000, the average relative payment reduction among penalized hospitals was around \$271,000.

Since our baseline empirical specification depends on within-hospital variation, we split our sample by whether a hospital ever faced a payment reduction under HRRP and HVBP during our sample period.  Table \ref{tab:bypenalty} presents summary statistics of our main dependent variable and some independent variables by ever-penalized status.  Payments to never-penalized hospitals are marginally higher than those to penalized hospitals over the 2010--2015 period.  Non-profit hospitals (public and private) constituted a much larger share of never-penalized hospitals, suggesting that non-profit hospitals may be of higher quality, at least in terms of HRRP and HVBP.  However, case mix is significantly more severe in the ever-penalized hospitals, which suggests that CMS risk-adjustment in HRRP and HVBP may not perfectly adjust penalty thresholds (consistent with \cite{wilcock2018}).  Ever-penalized hospitals tend to be in more competitive markets, have lower Medicare share, and come from more heavily populated counties.  Evidence from Table \ref{tab:bypenalty} suggests that controlling for hospital fixed effects is important in models of hospital payments because of persistent differences between ever-penalized and never-penalized hospitals.

The log of the annual, within-hospital mean of private insurance payments constitutes our primary dependent variable of interest. For brevity, we refer to this variable simply as the log mean payment. For comparison with the literature, we also follow \cite{dafny2009} in estimating hospital payments using the average net revenue for non-Medicare inpatient discharges. Specifically, we convert inpatient gross charges to inpatient net revenue by multiplying the hospital's total net revenues by the total gross charge ratio. Payments for Medicare inpatient services are then subtracted from inpatient net revenue to arrive at inpatient revenues from all non-Medicare patients, which we divide by the hospitals' discharges to derive the per-discharge net revenue amount. Since Medicaid revenues are not provided in HCRIS, the measure is a weighted average of net revenue per discharge for commercially insured and Medicaid patients where the weights equal the share of inpatient discharges belonging to each payer. This same measure has been used in recent studies examining hospital pricing behavior, including \cite{schmitt2017} and \cite{lewis2015}. To eliminate outliers, we trim the lower and upper tails at the 5th and 95th percentile of the resulting payment distribution, and we normalize this estimated payment based on the hospital's observed case mix index (CMI) from the inpatient prospective payment system (IPPS) final rule files. To differentiate this measure of payments from our observed payments from the HCCI data, we refer to this measure as the log mean net charge.

Finally, since standard theory of a for-profit firm suggests that the number of public insurance patients decreases following a reduction in the administrative price, we also include measures of payor mix as an additional set of outcomes. These measures include the log number of Medicare discharges, the log number of Medicaid discharges, and the log number of other discharges (non-Medicare and non-Medicaid). We also considered the Medicare, Medicaid, and other insurer shares (rather than log counts). Those results are excluded for brevity but qualitatively similar to the analysis of log counts.

\subsection{Regression Analysis}
\label{sec:initial}
Our baseline empirical specification isolates within-hospital variation in private payments over time by whether a hospital faced a net penalty from the HRRP and HVBP. This analysis therefore focuses on the extensive margin of penalties.  Equation 1 presents our main empirical model:
\begin{equation}
\label{eq: reg}
y_{ht} = \alpha_{h} + x^{'}_{ht}\beta + \delta1[Penalty_{t}]  + \theta_{t}  +  \epsilon_{ht},
\end{equation}
where outcome $y_{ht}$ at hospital $h$ in fiscal year $t$ is a function of a hospital specific intercept, $\alpha_{h}$; a vector of time-varying hospital and market-level exogenous characteristics, $x_{ht}$; an indicator for a net penalty under the combination of HRRP/HVBP policies; controls for year effects, $\theta_t$; and an i.i.d. error term $\epsilon_{ht}$.  Because the penalty indicator is zero for all hospitals in 2010 and 2011, and because we include hospital fixed effects, Equation 1 represents an unscaled difference-in-differences estimator. Our parameter of interest, $\delta$, captures the extent to which hospitals penalized under the HRRP/HVBP receive differential private payments relative to hospitals with no penalty (which includes hospitals that received a bonus).  For a causal interpretation of $\delta$, the underlying assumption in Equation 1 is that there are no time-varying unobserved characteristics that differentially affect payments in penalized hospitals relative to non-penalized hospitals, an assumption that we address in the next subsection.

Table \ref{tab:baselineresults} presents results from Equation 1 for the log of mean payments, the log of mean net charges, and several (logged) payer-specific discharge variables. The first column of Table \ref{tab:baselineresults} demonstrates that hospitals that faced payment reductions increased payments by 1.4\% over the period of 2012-2015.  This represents a roughly \$167 increase in payments among penalized hospitals, on average.\footnote{This interpretation is based on the average private insurance payment of \$12,100 among penalized hospitals after FY 2012. Assuming this average payment reflects a 1.4\% increase in the average payment in the absence of the penalty, we calculate the effect in dollar terms as $\$12,100 - \frac{\$12,000}{1+0.014}$.} Column 2 presents estimates from a similar model in which we replace negotiated payments with the log of mean net charges as discussed previously \citep{dafny2009,lewis2015,schmitt2017,dranove2017}. Results in column 2 suggest a smaller and statistically insignificant change in log mean net charges for penalized hospitals, which we argue demonstrates the importance of using actual payment data.  Columns 3 and 4 of Table \ref{tab:baselineresults} show movement \textit{away} from Medicaid and Medicare patients for penalized hospitals, with discharges decreasing by 4.5\% and 2.7\%, respectively.

To investigate the intensive margin effect of HRRP/HVBP penalties on hospital payments, we calculate the mean penalty per Medicare discharge at the hospital level.  For those hospitals with a net penalty, we break the distribution of penalty size into quartiles.  We replace the indicator for net penalty in Equation \ref{eq: reg} with indicators for each of the four penalty quartiles, where the omitted category is those hospitals which either saw no penalty or a net bonus in Medicare reimbursements. Results are presented in Table \ref{tab:int}.  Consistent with our results in Table \ref{tab:baselineresults}, we find that average payments are significantly higher in penalized hospitals relative to those receiving no change or a small bonus.  We find no effect on payments for hospitals in the first (smallest) quartile of penalties, defined as a per Medicare discharge penalty of between $\$$0.01 and $\$$12.59; however, we find a 2.4$\%$ increase in mean payments for hospitals in the highest quartile of penalties (between $\$$57.10 and $\$$291.60 per Medicare discharge).  Results in Table \ref{tab:int} therefore suggest that private payment increases are larger as the HRRP/HVBP penalty increases. Furthermore, we find monotonically more negative effects of a penalty on Medicaid and Medicare discharges in the size of the penalty.  Taken together, results from Tables \ref{tab:baselineresults} and \ref{tab:int} suggest meaningful increases in average hospital payments from private payers for those hospitals penalized under the combination of HRRP and HVBP.

\subsection{Robustness}
\label{sec:robust}
Results in Table \ref{tab:baselineresults} reflect the causal effects of the HRRP and HVBP penalties if there are no unobserved, time-varying factors that influence our outcomes and are also correlated with penalty status.  While we cannot completely rule out this possibility, we estimate a variety of alternative specifications in order to examine the influence of several potential confounders.  First, we include a set of time dummies interacted with penalty status to allow the trend in outcomes to vary by whether a hospital is ever-penalized. Differential trends conditional on penalty status and other controls would be suggestive of time-varying unobserved heterogeneity and would generally bias our estimate of $\delta$ towards zero.  These results are summarized in panel 1 of Table \ref{tab:robustness}. The estimate for log mean payments decreases from 1.4\% to 1.0\% when allowing for differential trends, but nonetheless remains economically meaningful and statistically significant. The remaining results for other outcomes are broadly consistent with those in Table \ref{tab:baselineresults}. We also present the p-values of a joint test of the null that the time trend dummies between ever-penalized and never-penalized hospitals is the same. For our log mean payment outcome, this test fails to reject the null of common trends between the ever-penalized and never-penalized hospitals (p-value $=$ 0.497). We reject the null in the case of log mean net charges and in the case of log Medicare discharges, which suggests the presence of some time-varying unobserved heterogeneity for these outcomes. This result makes sense given that the net penalty directly affects the Medicare market and therefore also affects our calculation of mean net charges by construction.

Second, we are concerned that unobserved differences across markets (e.g., with regard to insurer market power) may influence our estimates. We therefore include a set of county-level fixed effects, with results summarized in panel 2 of Table \ref{tab:robustness}. Here, we continue to find positive and significant effects on private insurance payments, as well as significant reductions in the log number of Medicare discharges. These results suggest that local area variation in provider or insurer markets is not driving our results.

Third, we remain concerned that other changes in the hospital-insurer relationship may drive our estimated increase in payments, particularly with respect to the implementation of the ACA. We therefore consider an alternative specification in which we include an indicator for whether the hospital was in a Medicaid expansion state as of 2014. These results are presented in panel 3 of Table \ref{tab:robustness} and are largely unchanged from our initial estimates.

Fourth, since the HRRP and HVBP are intended to reward and/or punish hospitals based in-part on quality of care, a hospital's ability to translate HRRP and HVBP penalties into higher private insurer payments may depend on whether such penalties reveal new quality information to the market. Existing findings from \cite{dranove2008} and others tend to find relatively small effects of quality reporting on hospital choice. As \cite{dranove2008} state, ``report cards do not always convey `news' about quality; in some cases the rankings confirm with prior beliefs about quality.'' To the extent that penalties from the HVBP and HRRP do not reveal any new information to the market, then the penalty acts simply as a reduction in public payments and the standard arguments for cost-shifting apply. The distribution of readmission rates across hospitals before the HRRP/HVBP suggest this is the case, as penalized hospitals already displayed higher readmission rates relative to other hospitals in the years prior to 2012. These distributions are presented in Figure \ref{fig:pre_readmits}. We also examine this issue with an alternative specification in which we control for a hospital's overall quality as measured by patients' overall hospital rating from the Hospital Consumer Assessment of Healthcare Providers and Systems (HCAHPS).  Panel 4 of Table \ref{tab:robustness} reports results from this model, with estimates almost identical to those in Table \ref{tab:baselineresults}.

Fifth, because of discrepancies in the timing of hospital fiscal years (both across hospitals and with CMS), the exact timing of the realization of Medicare payment cuts varies across our sample. In panel 5 of Table \ref{tab:robustness}, we report results from a model in which we drop fiscal year 2012 from our analysis. Again, our results are similar to those in Table \ref{tab:baselineresults}.

Sixth, it may be that other changes introduced through the ACA (e.g., expansion of insurance on the individual market) may have changed the ``typical'' patient being admitted to the hospital. In panel 6 of Table \ref{tab:robustness}, we demonstrate that our results are again unchanged when conditioning on the hospital's average case mix.

Finally, the novel aspect of our data is that, for a given acute care claim, we observe the actual payment from a private insurer to the hospital. Private insurance payments reflect some endogenously bargained discount on the charge or markup relative to Medicare payment rates and are therefore fundamentally different from charges, which reflect a hospital's list price for a given service. As noted above, the correlation in our data between mean payments and charge-based payments is 0.435, which suggests that measurement error in a model of log mean net charges could be significant. Many studies of hospital pricing proxy for payments with hospital charges and argue that hospital fixed effects control for mean differences between charges and payments \citep{cutler2000}. The last panel of Table \ref{tab:robustness} presents results when estimating Equation \ref{eq: reg} without hospital fixed effects.  Estimates of $\delta$ for log mean payments are negative, large, and significant. Relative to our initial results, these findings suggest that: 1) persistent and unobserved hospital-level heterogeneity is an important driver of outcomes in our setting; and 2) hospital fixed effects may in fact go a long way toward controlling for mean differences between charges and payments. However, we emphasize the importance of payment data with respect to the precision and measurement of hospital-insurer bargaining, noting the lack of statistical significance in our model of log mean net charges presented in Table \ref{tab:baselineresults}. Ultimately, these results offer some assurance that findings of a significant effect using charge-based estimates of prices are indeed reflective of a true price increase, while findings of an insignificant effect may be driven by incorrect inference (e.g., due to measurement error) or due to a true underlying null effect.

\section{Mechanisms for Payment Increases}
\label{sec:alt}
The results in Section \ref{sec:Empirical} show that penalized hospitals were able to increase private insurance payments by 1.4\% on average, and this effect is robust to a series of alternative specifications including allowing for differential time trends among penalized versus non-penalized hospitals. In this section, we consider different mechanisms that may have facilitated such an increase.

\subsection{Changes in Hospital Quality}
Since both the HRRP and HVBP are intended to incentivize hospitals to improve quality, a natural question is whether our estimated payment increases are simply a reflection of higher hospital quality, for which private insurers should be willing to pay a higher price. The existing literature in this area is mixed. \cite{gupta2018} find that the HRRP was associated with a 1.6 percentage-point reduction in 30-day Medicare readmissions for heart failure but a 1.4 percentage-point \textit{increase} in 30-day mortality. \cite{gupta2016}, however, finds evidence of a reduction in Medicare hospital mortality rates (a decrease of about 3\%, significant at the 10\% level) from the HRRP, which may account for as much as 60$\%$ of the reduction in readmissions. \cite{mellor2016} similarly find that the HRRP led to a decline in Medicare AMI 30-day readmission rates; however, new evidence from \cite{Ibrahim2017} suggest that observed decreases in readmissions may have been driven by hospitals coding patients more severely and not by ``real'' quality improvements. Consistent with this result, \cite{wilcock2018} find that the majority of HRRP penalties are a reflection of poor risk adjustment in the penalty calculation and not of true, underlying hospital quality.

Regarding the HVBP, the literature generally finds little or no effect on hospital quality \citep{ryan2015,doran2017,norton2017,ryan2017}. Examining data from 2015 to 2016, \cite{norton2017} did find some hospital response to the HVBP, but this response was in specific areas with the greatest marginal revenue rather than those areas with larger quality benefits. Conversely, based on quality data from 2005 through 2014, \cite{gao2015} found no effect of HVBP on quality. This study also interviewed a handful of hospital officials and concluded ``the HVBP program generally reinforced ongoing quality improvement efforts, but did not lead to major changes in focus.'' \cite{friedson2016} offer an explanation for these findings, where the authors find that the HVBP does not sufficiently discriminate between hospitals, and whether hospitals are penalized or rewarded by the HVBP program is largely a matter of chance rather than a reflection of true underlying quality.

Importantly, most of the existing studies of the HRRP/HVBP tend to focus on the Medicare population, but to the extent that HRRP/HVBP may have improved quality of care, such improvements should extend to private insurance patients in order to translate into higher private insurance payments. We are not aware of any evidence in the literature suggesting that quality in the private insurance market improved due to the HRRP or HVBP programs. Indeed, in a study of private insurance patients in Florida and California, \cite{demiralp2017} find no evidence that the HRRP reduced the readmission rate among the non-Medicare population. To test this empirically, we directly investigate whether penalized hospitals improved quality (as measured by readmissions) in the commercial insurance market.\footnote{Our data do not have a reliable measure of mortality. We therefore focus the analysis on readmissions. We also note that our data include inpatient stays in which the patient may have died in the hospital or soon after; however, given the age composition of the commercial sample, death is likely to be less frequent than in the Medicare population.} We estimate the effect of hospital penalty status on the probability of readmission using a linear probability model with data at the individual admission level. Following the Agency for Healthcare Research and Quality definition, we classify a readmission to be any admission to any inpatient prospective payment hospital within 30 days of a discharge.\footnote{See \href{https://www.hcup-us.ahrq.gov/reports/statbriefs/sb230-7-Day-Versus-30-Day-Readmissions.jsp?utm_source=ahrq&utm_medium=en1&utm_term=&utm_content=1&utm_campaign=ahrq_en11_7_2017}{2017 AHRQ Statistical Brief \#230} for additional details on the readmission calculations.} Our sample again excludes newborns and transfers, and we limit the analysis to all patients with 12 months of private insurance coverage in a calendar year.

Our linear probability model includes all controls from our main specification plus patient controls such as age range, gender, length of stay, DRG weight, insurance product type (HMO, PPO, POS, EPO), and DRG fixed effects. As summarized in column 1 of Table \ref{tab:other_results}, the results demonstrate that, even with a sample of over 3 million observations, we find an economically and statistically insignificant effect of penalty status on the probability of readmission.\footnote{We also estimated the model using the lagged net penalty, where we again find an economically and statistically insignificant effect of penalty status on the probability of readmission.} To the extent that penalized hospitals are investing in quality to lower Medicare readmissions among the indicated areas, we find no evidence that such quality improvements are changing readmissions on average for the commercially insured population.

\subsection{Changes in Services or Treatment Intensity}
Using our data on private payments, we are also able to directly address the extent to which hospitals respond to public penalties by changing treatment patterns or reallocating resources towards more profitable services. Indeed, since our outcome is calculated as an average payment per patient, our results could simply reflect increases in the intensity of treatment rather than an increase in the payment received for an otherwise identical service.  To investigate, we first estimate the effects of Medicare payment reductions on charges among the commercial insurance population. This analysis uses within-hospital variation in charges as a general proxy for changes in intensity of treatment, with results presented in column 2 of Table \ref{tab:other_results}. Here, we find no economically or statistically significant increase in charges among penalized hospitals.

We also follow \cite{horwitz2009} in constructing a set of indicators for ``profitable'' (e.g., angioplasty or neonatal intensive care) versus ``unprofitable'' (e.g., alcohol dependency services or hospice care) hospital services.\footnote{A full list of relatively profitable and relatively unprofitable services is provided in Table 2 of \cite{horwitz2009}. Following their analysis, we identify whether a hospital offers these services based on responses from the AHA annual surveys.} We then constructed a ``profitable services index'' calculated as the ratio of profitable services to all profitable and unprofitable services identified by \cite{horwitz2009}. For example, if the hospital offered 2 profitable services and 2 unprofitable services, then the ratio for this hospital would be 50\%. Treating this profitable services index as an additional outcome and repeating our analysis from Section \ref{sec:Empirical}, column 3 of Table \ref{tab:other_results} demonstrates that we find small and insignificant effects of being penalized. These insignificant effects persist when examining for-profit and non-profit hospitals separately as well as across all robustness checks presented in Table \ref{tab:robustness}.

A similar pattern emerges in Table \ref{tab:other_results} when we consider average DRG weights and average length of stay (among our commercial insurance population) as separate outcomes, with insignificant effects of HRRP/HVBP penalties on these outcomes in all specifications considered.  Finally, it may be that penalized hospitals incurred some costly investments, perhaps with the aim of improving quality of care. While our data are limited in these areas, we also estimated the effect of hospital penalty status on the log of cost per discharge (hospital-wide).\footnote{We calculate costs per discharge based on data available in HCRIS.} Here, we again find no significant or economically meaningful effects of being penalized on the costs per discharge for the hospital.

Collectively, we find little empirical evidence that HRRP/HVBP penalties induced hospitals to increase quality in the commercial insurance population, increase intensity of treatment, adjust service offerings toward more profitable areas, or otherwise increase costs per discharge. The results in Table \ref{tab:other_results} therefore support the hypothesis that our estimated increase in payments derives from some underlying increase in private insurance payments for otherwise similar services.

\section{Heterogeneity in Price Increases}
\label{sec:Ext}
With no apparent changes in underlying hospital services, treatment intensity, or quality of care, one explanation for our estimated increase in private hospital payments is that of hospital cost-shifting. In this section, we test a set of hypotheses that would be consistent with cost-shifting in the context of hospital-insurer bargaining and the HRRP/HVBP.

\subsection{Theoretical Framework}
As initially examined in \cite{dranove1988}, a hospital may pursue a cost-shifting strategy if the hospital's objective function includes something other than pure profit (e.g., if the hospital receives direct utility from the quantity of services provided). For this reason, cost-shifting is thought to more likely occur among non-profit hospitals, if at all. Indeed, to maintain their tax exempt status, non-profit hospitals are required by the IRS to provide community benefits.\footnote{Of course, this does not mean that non-profit hospitals are fully altruistic. In fact, evidence on non-profit hospital behavior relative to for-profit hospital behavior is mixed. For example, \cite{silverman2004} and \cite{dafny2005} find evidence that non-profits ``upcode'' less frequently, while \cite{gaynor2003} find that non-profit hospitals have lower marginal costs but higher markups than for-profit hospitals.} Since over 80\% of hospitals in our sample are non-profit, this implies that the objective function of the majority of hospitals in our analysis extends beyond pure profit-maximization.  

The model posited in \cite{dranove1988} assumes that hospitals set payments unilaterally, and it is not immediately clear whether this prediction extends to a modern managed care market in which hospitals and private insurers negotiate over private prices. To more formally examine the presence of cost-shifting in a bargaining context, we embed the hospital cost-shifting model from \cite{dranove1988} in a hospital-insurer bargaining model similar to that in \cite{ho2017} (HL), \cite{gowrisankaran2015}, \cite{lewis2015}, and \cite{dor2004}. Specifically, we consider a hospital whose objective is to maximize a function of profits and quantity of care provided, denoted by
\begin{equation}
 U\left( \pi_{j} = \sum_{i=1}^{N_{j}} \pi_{i,j}^{h} + \pi_{g,j}^{h}, \sum_{i=1}^{N_{j}} D_{i,j}^{h}, D_{g,j}^{h} \right),
\label{eqn:nfp_objective}
\end{equation}
where $\pi_{j}$ denotes total profits for hospital $j$ and $D_{i,j}^{h}$ denotes hospital demand from insurer $i$. Following HL, we assume $$\pi_{i,j}^{h}=D_{i,j}^{h}(p_{i,j}-c_{i}),$$ where $p_{i,j}$ denotes the negotiated payment between insurer $i$ and hospital $j$. We also follow HL in assuming that patients are ``unaware or unable to determine their [financial] liability prior to choosing their provider.'' In other words, the negotiated payment $p_{i,j}$ does not affect demand for a specific hospital.\footnote{This assumption has implications for the traditional model of a profit maximizing firm operating in a two-price market. In that model, which is often cited as theoretical support for a prediction of lower private prices following a reduction in public payments, firms lower private prices in order to attract more private market patients on the margin. But in a bargaining context in which the hospital remains in the insurer's network, there should be no such marginal response from patients.} The subscript $g$ denotes public (or government) insurers, for which the payment is administratively set at $p_{g}$. Finally, again following HL, we assume that profits for insurer $i$ are
\begin{equation}
\pi_{i}^{M} = D_{i} \left( \theta_{i} - \eta_{i} \right) - \sum_{j=1}^{N_{i}} D_{i,j}^{h} p_{i,j},
\label{eqn:ins_profit}
\end{equation}
where $D_{i}$ denotes the number of enrollees for insurer $i$, $\theta_{i}$ denotes the insurer's premiums, $\eta_{i}$ denotes insurer costs per-enrollee other than inpatient hospital care, and $D_{i,j}^{h} p_{i,j}$ reflects payments to hospitals for care provided to the insurer's enrollees.

The negotiated price between hospital $j$ and insurer $i$ is such that
\begin{equation}
 p_{ij}= \arg \max_{p_{ij}} \left(\triangle U_{j} \right)^{b_{j}} \times \left(\triangle \pi^{M}_{i} \right)^{1-b_{j}},
 \label{eqn:neg_price}
\end{equation}
where $\triangle U_{j}$ denotes the change in hospital $j$'s utility from reaching an agreement with insurer $i$, and similarly $\triangle \pi^{M}_{i}$ denotes the change in insurer profits from an agreement with hospital $i$. $b_{j}$ denotes the bargaining weight of hospital $j$, expressed as the weight to which the hospital's payoffs are given in the overall net value.

The first order condition for Equation \ref{eqn:neg_price} can be simplified to
\begin{equation}
 b_{j} \triangle \pi_{i}^{M} \pderiv{U_{j}}{\pi_{ij}^{h}} - (1-b_{j}) \triangle U_{j} = 0,
\label{eqn:price_foc}
\end{equation}
and appyling the implicit function theorem yields the relevant comparative static:
\begin{equation}
\deriv{p_{ij}}{p_{g}} = \frac{- b_{j} \triangle \pi_{i}^{M} \pderiv{^{2}U_{j}}{\pi_{j}^{2}}D_{g}^{h}}{D_{ij}^{h}\left(b_{j} D_{ij}^{h} \pderiv{^{2}U_{j}}{\pi_{j}^{2}} - (1-b_{j}) \pderiv{U_{j}}{\pi_{j}} \right)}.
\label{eqn:comp_static}
\end{equation}
We can see immediately from Equation \ref{eqn:comp_static} that $\deriv{p_{ij}}{p_{g}}<0$ whenever $\pderiv{^{2}U_{j}}{\pi_{j}^{2}}<0$. This means that hospitals must have some diminishing marginal utility of profits for cost-shifting to occur. Importantly, we obtain a prediction of cost-shifting without hospitals deriving utility from something other than profits, which is necessary for cost-shifting to occur in \cite{dranove1988}. If we interpret diminishing marginal utility simply as a reflection of risk-aversion (e.g., in the context of uncertain demand or uncertain ``exposure'' to the HVBP/HRRP penalties), then this model predicts any risk-averse hospital to potentially cost shift, regardless of whether the hospital is for-profit or non-profit.\footnote{While it is commonly assumed that for-profit firms are risk-neutral, there is an influential literature examining the role of risk aversion in the context of demand uncertainty. See, for example, \cite{sandmo1971}, \cite{holthausen1979}, \cite{mcdonald1985}, \cite{guiso1999}, \cite{chavas1996}, \cite{asplund2002}, and many others. Intuitively, risk aversion could be introduced through the presence of risk-averse shareholders or managers/administrators. In the case of physician-owned hospitals, diminishing marginal utility of profits is analogous to diminishing marginal utility of income to the physicians, since they are the residual claimants for hospital profits. Note that the costs of additional tests, longer inpatient stays, or other complications incurred outside of the initial surgery are generally borne by the hospital rather than the physician. In this way, the presence of risk aversion is less of an issue in the market for physician services since physicians are not generally exposed to the risk of higher costs for the same patient.} A related explanation could be something akin to a target income hypothesis, in which hospitals maintain some target profit margin and work to negotiate increases from private insurers when forced below the target margin due to a public payment reduction. Our model also predicts that cost-shifting should be largest for hospitals with more bargaining power, $b_{j}$, or a better bargaining position.

There is a perception in the literature that hospitals are indeed risk-averse \citep{cooper2017};\footnote{As discussed in \cite{cooper2017}, risk aversion is a natural reason that a hospital may prefer charge-based contracts as opposed to a prospective payment, since the hospital is exposed to uncertainty underlying a given inpatient stay in a prospective payment model.} however, in the case of a risk-neutral profit maximizing firm, the presence of cost-shifting requires that the firm must have some unused bargaining power in order to increase commercial insurance prices following a reduction in public payments. The nature of the hospital-insurer negotiation process suggests that this could occur. For example, public descriptions of hospital-insurer negotiations depict a relatively informal process.\footnote{See, for example, a recent guide from Becker's Hospital Review on \href{https://www.beckershospitalreview.com/finance/the-ins-and-outs-of-successful-hospital-insurance-negotiations-and-service-pricing.html}{The Ins and Outs of Successful Hospital Insurance Negotiations and Service Pricing.}} This depiction is consistent with several discussions we have had with insurance and hospital executives, as well as consultants hired specifically to assist with hospital-insurer contract negotiations. Indeed, experts cited several examples where seemingly sophisticated negotiating parties (i.e., large hospitals systems and insurers) had not previously examined publicly available information on hospital costs and revenues as part of their research leading up to contract renegotiations.

While this evidence is anecdotal, it suggests an informal process in which hospitals incorporate general changes in their environment over prior periods to argue in favor of a larger rate increase in any given negotiation. \cite{sacarny2018} also provides direct empirical evidence that, even when the hospital has full control over potential revenue increases, they may not fully maximize profits and thus effectively ``leave money on the table'' in any given period. 

If cost-shifting does occur, there are at least three dimensions by which we expect some heterogeneity in the extent of cost-shifting across hospitals. First, as in \cite{dranove1988}, cost-shifting may be isolated among not-for-profit hospitals. Our result in Equation \ref{eqn:comp_static} suggests this is not necessarily the case in a bargaining context, but we can test this empirically in our data. Second, from Equation \ref{eqn:comp_static}, a hospital's ability to cost-shift should be larger if they have a better bargaining position. Third, since hospitals are penalized under the HRRP based on readmissions among specific DRGs, it may be that hospitals are able to increase private prices differentially across service lines. We explore each of these areas in more detail throughout the remainder of this section.

\subsection{Heterogeneity by Hospital Objective Function}
To empirically investigate the role of for-profit versus not-for-profit status, we re-estimate Equation 1 separately for non-profit and for-profit hospitals.  The results are presented in Table \ref{tab:byprofit}. Although our estimates among non-profits are more precisely estimated, we otherwise see no meaningful difference in the effects of HRRP and HVBP penalties on the magnitude of cost-shifting between for-profit and non-profit hospitals. In panels 2 and 4 of Table \ref{tab:byprofit}, we allow for differential trends by penalty status (analogous to the overall results in panel 1 of Table \ref{tab:robustness}), again with little change relative to the initial results and panels 1 and 3, respectively.

To the extent that these results reveal information on the underlying objective function of the hospital, out estimates suggest that attitudes toward risk may be similar among for-profit and non-profit hospitals. These findings therefore contribute the broader literature on the theory of the hospital and how, if at all, hospital behaviors differ according to profit status \citep{sloan2001,duggan2002,horwitz2005,horwitz2009,david2009}.

\subsection{Heterogeneity by Bargaining Position}
From Equation \ref{eqn:comp_static}, a hospital's incentive to cost-shift is larger as the insurer's outside option decreases (i.e., $\triangle \pi_{i}^{M}$ increases). Practically, this suggests that hospitals will be more likely to cost-shift if they have some \textit{relative} market power, where the insurer is heavily dependent on the hospital but where the hospital does not receive a large number of patients from any given insurer. In the parlance of \cite{lewis2015}, relative market power describes the hospital's ``bargaining position,'' while a hospital's ``bargaining power'' is reflected by $b_{j}$.\footnote{While bargaining power and bargaining position may seem to be related, \cite{lewis2015} find that bargaining power is not significantly affected by the hospital's bargaining position or direct market share.}

To investigate, we attempt to proxy for a hospital's bargaining position by constructing the quartile of the hospital's share of public patients relative to total patients, and we interact our penalty variable with indicators for each quartile.\footnote{We also tested for differential effects of the penalty among hospitals operating as a monopoly, duopoly, or triopoly. Here, we find a relatively large and positive effect of the interaction between a monopoly indicator and the penalty indicator, with a point estimate of 0.013; however, the effect is statistically insignificant with a p-value of 0.23. We estimate smaller and statistically insignificant effects on other interaction terms between penalty status and duopoly or triopoly indicators. This pattern of results persists for different measures of the hospital market. For brevity, the full results from these specifications are excluded from the paper but are available upon request.} This analysis is similar to that of \cite{wu2010}, who intuits that a hospital with a large share of private payers represents a more important client for the insurance market, and the hospital may leverage this power when public payments are cut. Applying this intuition to a study of hospital cost-shifting following the Balanced Budget Act of 1997, \cite{wu2010} finds that hospitals with larger shares of private patients were more able to pass Medicare payment reductions on to private payers.

Results are presented in Table \ref{tab:publicshare} and suggest that our initial estimate of cost-shifting is driven by hospitals with the smallest share of public patients. Indeed, the first column of Table \ref{tab:publicshare} demonstrates that payments increased by 3.9\% for hospitals with the smallest share of public patients. This increase was nullified for hospitals in the third and fourth quartile of public patient shares.

Another proxy for bargaining position is whether a hospital is aligned with its network of physicians. \cite{lewis2015}, for example, find that hospitals that are affiliated with a physician group are able to negotiate a larger share of surplus. Vertical integration with physicians may therefore put some hospitals in a more favorable bargaining position, and thus facilitate a larger increase in private payments, consistent with Equation \ref{eqn:comp_static}. To investigate, we estimate our preferred empirical model on data from only those hospitals that already owned a physician group or physician practice \textit{prior} to 2012.\footnote{The AHA surveys provide information at the hospital-level on whether a hospital currently has an ``integrated salary model.'' This measure unfortunately does not capture \textit{how many} physicians are employed by a hospital, but instead only captures if there is any integrated model reported between the hospital and any of its physicians.} We also estimate our model on hospitals never observed to be vertically integrated.  As shown in Table \ref{tab:VI}, amongst those hospitals already vertically integrated, the effect of a net penalty on payments is 2.3$\%$ and strongly significant; a penalty is associated with a small and statistically insignificant effect on payments among those hospitals never observed to be vertically integrated.\footnote{We also considered whether the penalty itself led to more integrated salary models by treating the binary integration measure as an additional outcome. Here, we estimate a very small and insignificant negative effect of being penalized on the probability of reporting an integrated salary model, suggesting that penalized hospitals were not integrating with physicians due to the penalty. These results are excluded for the paper but available upon request.}

We note that predictions involving a hospitals' bargaining position are less clear when we incorporate the insurer's choice of premiums in the insurance market. If the insurance market is heavily concentrated, then insurers can pass health care price changes on to their plan enrollees \citep{trish2015,ho2017}. This intuition leads to conflicting conclusions: cost-shifting is likely to occur when insurers have a particularly small market share (such that hospitals can leverage their bargaining position), but perhaps also when insurers have a particularly large market share (such that insurers can pass price increases on to plan enrollees). The role of insurance markets on the prevalence or magnitude of cost-shifting is therefore empirically difficult to measure without detailed data on insurance premiums and insurer market shares (at a local level). Because we lack reliable information on local area insurance concentration, we leave as an open question the extent to which cost-shifting is more prominent in markets with both provider market power and a concentrated insurance market.

\subsection{Heterogeneity by Service Area}
As discussed previously, the reduction in public payments used in our analysis derives from a lower-than-expected performance on some set of quality metrics. How could hospitals translate this signal into higher prices? We presented evidence in Table \ref{tab:robustness} that our estimates are robust to any reputation effects from the HRRP and HVBP penalties as measured by patients' hospital ratings; however, the quality signal may be uninformative to patients while potentially informative to managed care insurers. In this case, penalized hospitals may instead target other service areas where they may maintain a comparative advantage in the market. To investigate this further, we estimate models of the log of mean payments within acute care admission service categories.

Estimates for $\delta$ are presented in Table \ref{tab:eachcondition} for several categories of acute admissions. For each admission category, we restrict our sample to hospitals with at least 25 admissions in that category in each year of our sample. Table \ref{tab:eachcondition} demonstrates significant increases in payments for nervous and circulatory admissions, suggesting that cost-shifting may occur for multiple types of admissions. Because two of the three original conditions rated by the HRRP (AMI and heart failure) were circulatory system conditions, an open question remains as to how hospitals may negotiate higher prices for these conditions. It could be that the average increase in circulatory system prices is driven by conditions other than AMI or heart failure (e.g., coronary artery disease or stroke), or it could be that the penalty among hospitals that ultimately negotiated higher circulatory system prices was driven by lower-than-expected performance in pneumonia patients rather than AMA or heart failure patients. Because of limited sample sizes for such hospitals and conditions, we cannot examine these questions directly in our data (both due to restrictions on dissemination of small sample size results and due to large variability in payments for infrequent procedures).

\section{Conclusion}
\label{sec:Conclusion}
This paper uses novel payment data from a large, multi-payer database to investigate how hospital payments from private insurers change following public-sector payment cuts. We use variation in Medicare payments generated by two cost-containment policies within the ACA -- the hospital readmissions reduction program and the hospital value based purchasing program -- to estimate the role of a net public payment reduction on average hospital payments. Our initial analysis estimates a 1.4\% increase in private insurance prices for hospitals that were penalized under the HRRP/HVBP programs. Subsequent analysis finds that this estimate is robust to a variety of alternative specifications, including differential trends among penalized and non-penalized hospitals, and we find no evidence of changes in underlying quality or intensity of treatment.

Our results therefore suggest that hospitals were able to negotiate higher private insurance prices for otherwise identical hospital services. One theory that is potentially consistent with these results is that of hospital cost-shifting. Motivated by a simple extension of \cite{dranove1988} to a bargaining framework, we generate a series of predictions implied by the cost-shifting model and test these predictions in the data. First, our theoretical model suggests that cost-shifting will only occur in the presence of diminishing marginal utility of profits, which we argue need not require the hospital to be non-profit. Consistent with this prediction, the magnitudes of our estimates among for-profit and non-profit hospitals are similar; however, our estimates are only statistically significant for non-profit hospitals due to the larger sample size relative to for-profits.

We then considered the mechanisms for cost-shifting specifically in the context of our data. In particular, our empirical analysis identifies cost-shifting from cross-sectional and temporal variation in penalties levied under the HRRP and HVBP programs. The presence of such a penalty suggests that the hospital has under-performed in some way relative to an average hospital, and it is unclear how a hospital could translate this underperformance into \textit{higher} prices. We intuit two potential mechanisms: 1) by increasing prices for services unrelated to the areas in which the hospital was evaluated and ultimately penalized; and 2) by leveraging its bargaining position as predicted in our theoretical model. We find some evidence in favor of each mechanism, where hospitals do appear to increase prices in areas unrelated to the HRRP and HVBP (but not exclusively in such areas) and where cost-shifting is also largest among hospitals with a more advantageous bargaining position (as proxied by the hospital's share of public insurance patients and the presence of some vertical integration with physicians).

Collectively, our analysis offers three central findings: 1) private insurance payments increased among hospitals penalized by the HRRP and HVBP; 2) the payment increases do not appear to be explained by changes in hospital services or quality of care; and 3) our empirical extensions tend to support heterogeneity in price effects along the same dimensions as suggested by a model of cost-shifting in the context of the HRRP/HVBP. In this way, our results support the notion of some degree of cost-shifting in the modern healthcare environment. To quantify this effect, note that our estimated 1.4\% increase in payments implies an increase of \$167 per inpatient stay based on an average private insurance payment of approximately \$12,100 among penalized hospitals. As a back-of-the-envelope calculation, if one assumes that this payment increase applies to around 1,100 inpatient stays per year, then we estimate a total increase in private insurance payments of up to \$183,700 per hospital per year.\footnote{Our price data are based on just over 550 inpatient stays per year per hospital and reflect nearly 30\% of all commercial insurance claims. Extrapolating to 1,100 assumes that some but not all commercial insurers captured in our data would have experienced the same price increase as estimated in our analysis.} To put this in context, penalized hospitals saw an average penalty of around \$205,000, while non-penalized hospitals received an average bonus of just over \$66,000. This yields a differential payment between penalized and non-penalized hospitals of approximately \$271,000. An estimated increase of \$183,700 in private insurance payments therefore translates to cost-shifting at a rate of \$0.68 per \$1 reduction in Medicare payments.

We stress, however, that our estimated payment increases follow from a relatively small penalty, and there is intuitively some ceiling by which hospitals will be restricted in their ability to negotiate higher private insurance payments. As the reduction in Medicare payments increases, hospitals are more likely to confront this constraint, at which point the rate of any cost-shifting would decrease. We therefore caution against using this estimate of cost-shifting to predict specific commercial insurance payment increases in other contexts.

We also stress that these results should not be interpreted to suggest that pay for performance in health care is inherently bad. Instead, we interpret our results as highlighting the importance of how the pay for performance program is designed. In the case of the HRRP, hospitals need only be below average in one area in order to incur some percent penalty levied on all Medicare payments. Most hospitals are not better than average in every dimension, and indeed, as the number of conditions in the HRRP has grown, so too has the percentage of hospitals penalized in a given year. Intuitively, the HRRP is a relatively blunt instrument that penalizes most hospitals in a given year. Subsequently, HRRP penalties may serve as a poor quality signal. The HVBP may similarly suffer from some basic design problems. For example, in tracking a hospital's performance across 20-plus metrics, it becomes difficult to discern a true quality signal from each hospital. When applied to a highly concentrated private industry, our results suggest that such pay for performance programs may have important unintended consequences.

\newpage
\bibliographystyle{authordate1}
\bibliography{BibTeX_Library}


\clearpage
\newpage


\newsavebox{\gfxbox}
\newpage
\section*{Figures}
\savebox{\gfxbox}{
\centering
\includegraphics[scale=1]{Readmit_Graphs.pdf}
}
\setlength{\captionmargin}{.5 \textwidth} \addtolength{\captionmargin}{-.5\wd\gfxbox}
\begin{figure}[htbp!]
\centering
\caption{Pre-HRRP/HVBP Readmission Rates}
\label{fig:pre_readmits}
\usebox{\gfxbox}
\par
\begin{minipage}{\wd\gfxbox}
\footnotesize
Notes:  Kernel density estimates for readmission rates prior to HRRP/HVBP among hospitals ultimately penalized versus those not penalized. Readmission rates reflect reported rates in 2010 and 2011, which are constructed from rates in 2006-2009 and 2007-2010, respectively.
\end{minipage}
\end{figure}
\newpage


\section*{Tables}


\savebox{\gfxbox}{
\footnotesize
\begin{tabular}{ccccccc}
\hline \hline
%\multicolumn{9}{c}{}\\
Fiscal & Sample 	&  Payment $\$$				& Medicare   & Medicaid  	& Other & Percent \\
Year   &  Size    	&  Mean (St. Dev.) 				& Discharges    	& Discharges       	& Discharges   & Penalized \\
 \hline
2010 &      1,386		& 	10,729.22   (4,936.50)	 &   4,614.62  &   2,010.11    &  7,898.18  & 0.00  \\
2011 &      1,386		& 	11,602.74   (5,076.45)	&  4,618.93    & 1,960.05    &  7,892.21  & 0.00   \\
2012 & 	1,386 		& 	12,079.46   (5,477.37) 	  &  4,493.31  &  1,810.27   &   8,019.04  & 0.32   \\
2013 & 	1,386		& 	12,668.44   (5,567.76)	  & 4,396.32    & 1,783.81    &  7,996.10  & 0.74  \\
2014 & 	1,386		&	12,795.83   (5,444.21)	 &    4,260.43   &   1,726.25  &    7,852.71 & 0.76 \\
2015 &    1,386			& 	13,397.63   (5,921.74)	&    4,311.41   &  1,578.86    &   8,261.74 & 0.79 \\
\hline
Total & 	8,316		& 12,212.22   (5,481.55)	  &    4,449.17   &  1,811.56  &   7,986.67 & 0.43 \\
\hline
\end{tabular}
}
\setlength{\captionmargin}{.5 \textwidth} \addtolength{\captionmargin}{-.5\wd\gfxbox}
\begin{table}[htbp!]
\centering
\caption{Characterization of Research Sample over Time}
\label{tab:summarystats}
\usebox{\gfxbox}
\par
\begin{minipage}{\wd\gfxbox}
\footnotesize
Notes: Balanced panel of hospitals over time between 2010 and 2015.  Payment represents the mean dollar amount paid to a hospital in a year over all acute care admissions.  Penalty is a binary variable for whether the combination of HRRP and HVBP resulted in a net payment reduction. Other discharges denotes all discharges other than Medicare and Medicaid.
\end{minipage}
\end{table}



\newpage
\savebox{\gfxbox}{
\scriptsize
\begin{tabular}{lccc}
\hline \hline
Variable 				& Never 			& Ever  				&   	  \\
		   			&  Penalized    		& Penalized			&    p-value   				\\
 \hline
Log(Payment)			&	9.423	&	9.300	&	0.000	\\
Log(Charge)			&      8.843	&	8.726	& 	0.000	\\
System  Membership      	&	0.768	&	0.784	&	0.352	\\
Non-profit     			&	0.790	&	0.692	&	0.000	\\
Log(Case Mix Index)        	&	0.437	&	0.447	&	0.090	\\
\multicolumn{4}{l}{Local Hospital}\\							
\hspace{0.05in} Monopoly  	&	0.133	&	0.113	&	0.110	\\
\hspace{0.05in} Duopoly    	&	0.282	&	0.156	&	0.000	\\
\hspace{0.05in} Triopoly   	&	0.139	&	0.108	&	0.012	\\
\multicolumn{4}{l}{Market Share}\\							
\hspace{0.05in} Medicare  	&	0.338	&	0.330	&	0.056	\\
\hspace{0.05in} Medicaid    	&	0.110	&	0.125	&	0.000	\\
\hspace{0.05in} Medicare+Medicaid      	&	0.447	&	0.455	&	0.086	\\
\hspace{0.05in} Other     	&	0.553	&	0.545	&	0.086	\\
Total Pop. (1000s)     	&	714	&	1,190	&	0.000	\\
\multicolumn{4}{l}{County Age Distribution}\\							
\hspace{0.05in}[18, 34]      	&	0.240	&	0.239	&	0.504	\\
\hspace{0.05in}[35, 64]   	&	0.393	&	0.393	&	0.947	\\
\hspace{0.05in}$>$65    	&	0.133	&	0.130	&	0.101	\\
\multicolumn{4}{l}{County Race Distribution}\\							
\hspace{0.05in}White     	&	0.795	&	0.734	&	0.000	\\
\hspace{0.05in}Black    	&	0.096	&	0.134	&	0.000	\\
\multicolumn{4}{l}{County Income Distribution}\\							
\hspace{0.05in}$<$ \$50k   	&	0.185	&	0.180	&	0.000	\\
\hspace{0.05in}[\$50k, 75k]   	&	0.126	&	0.123	&	0.000	\\
\hspace{0.05in}[\$100k, 150k]       	&	0.132	&	0.132	&	0.820	\\
\hspace{0.05in}$>$ \$150k   	&	0.095	&	0.101	&	0.007	\\
\multicolumn{4}{l}{County Education Distribution}\\							
\hspace{0.05in}High School Only  	&	0.270	&	0.270	&	0.925	\\
\hspace{0.05in}Bachelor's Only    	&	0.197	&	0.191	&	0.005	\\
\hline
\end{tabular}
}
\setlength{\captionmargin}{.5 \textwidth} \addtolength{\captionmargin}{-.5\wd\gfxbox}
\begin{table}[htbp!]
\centering
\caption{Hospital Characteristics by Penalties}
\label{tab:bypenalty}
\usebox{\gfxbox}
\par
\begin{minipage}{\wd\gfxbox}
\footnotesize
Notes:  $n=8,316$ Summary statistics are split by whether a hospital is ever observed to receive a net penalty in 2012-2015. Payment represents the mean dollar amount paid to a hospital in a year over all acute care admissions.   County level characteristics are from the American Community Survey.
\end{minipage}
\end{table}

\newpage
\savebox{\gfxbox}{
\scriptsize
\begin{tabular}{llllll}
\hline\hline
 			& Log Mean		& Log Mean			& Log Medicaid 	   	& Log Medicare   		& Log Other  			\\
			& Payment		& 	Net Charge		& Discharges      		& Discharges       	& Discharges        	\\
\hline
Net Penalty  					&	0.014***	&	0.008	&	-0.045**	&	-0.027***	&	-0.004	\\
							&	(0.005)	&	(0.008)	&	(0.021)	&	(0.007)	&	(0.011)	\\
\multicolumn{6}{l}{Hospital Characteristics}\\											
\hspace{0.15in} Monopoly			&	-0.008	&	0.004	&	-0.025	&	0.003	&	-0.012	\\
							&	(0.012)	&	(0.011)	&	(0.055)	&	(0.025)	&	(0.029)	\\
\hspace{0.15in} Duopoly			&	-0.005	&	0.010	&	0.036	&	0.030	&	0.013	\\
							&	(0.010)	&	-(0.010)	&	(0.044)	&	(0.019)	&	(0.023)	\\
\hspace{0.15in} Triopoly			&	0.000	&	0.003	&	-0.000	&	0.002	&	0.006	\\
							&	(0.009)	&	(0.008)	&	(0.039)	&	(0.015)	&	(0.019)	\\
\hspace{0.10in} Large Market		&	-0.041	&	0.001	&	-0.063	&	0.049**	&	0.179***	\\
							&	(0.028)	&	(0.013)	&	(0.050)	&	(0.020)	&	(0.043)	\\
\hspace{0.10in} Any Teaching		&	-0.018	&	-0.022	&	-0.047	&	-0.021	&	-0.013	\\
							&	(0.012)	&	(0.014)	&	(0.039)	&	(0.016)	&	(0.022)	\\
\hspace{0.10in} Major Teaching		&	0.003	&	-0.001	&	0.008	&	0.009	&	0.011	\\
							&	(0.006)	&	(0.004)	&	(0.026)	&	(0.010)	&	(0.012)	\\
\hspace{0.10in} System			&	0.019	&	-0.002	&	-0.091**	&	-0.066***	&	-0.083***	\\
							&	(0.015)	&	(0.011)	&	(0.041)	&	(0.019)	&	(0.020)	\\
\hspace{0.10in} Nonprofit			&	0.020 &	-0.009	&	0.073	&	0.036	&	0.016	\\
							&	(0.026)	&	(0.016)	&	(0.058)	&	(0.028)	&	(0.032)	\\
\multicolumn{6}{l}{County Age Share}\\											
\hspace{0.1in}[18,34]			&	-1.132*	&	-0.896*	&	2.902	&	-3.163***	&	-1.418	\\
							&	(0.681)	&	(0.543)	&	(2.327)	&	(0.853)	&	(0.880)	\\
\hspace{0.1in}[35,64]			&	-0.402	&	-1.182*	&	2.923	&	-3.428***	&	-0.044	\\
							&	(0.910)&	(0.656)	&	(2.781)	&	(1.171)	&	(1.295)	\\
\hspace{0.1in} $>$64			&	-0.488	&	0.281	&	-1.440	&	0.361	&	-0.838	\\
							&	(0.797)	&	(0.671)	&	(2.765)	&	(1.245)	&	(1.359)	\\
\multicolumn{6}{l}{County Share in Income Group}\\											
\hspace{0.1in} 50k-75k			&	-0.288	&	-0.034	&	1.518	&	-0.173	&	0.420	\\
							&	(0.386)	&	(0.286)	&	(1.439)	&	(0.548)	&	(0.790)	\\
\hspace{0.1in} 75k-100k			&	-0.279	&	0.649*	&	0.281	&	-0.319	&	-0.286	\\
							&	(0.479)	&	(0.352)	&	(1.736)	&	(0.623)	&	(0.791)	\\
\hspace{0.1in} 100k-150k			&	-0.736	&	0.290	&	-1.847	&	-0.017	&	0.072	\\
							&	(0.457)	&	(0.313)	&	(1.533)	&	(0.625)	&	(0.776)	\\
\hspace{0.1in}$>$150k			&	0.891**	&	-0.139	&	0.814	&	0.997*	&	-1.767***	\\
							&	(0.402)	&	(0.314)	&	(1.375)	&	(0.511)	&	(0.671)	\\
\hline
\end{tabular}
}
\setlength{\captionmargin}{.5 \textwidth} \addtolength{\captionmargin}{-.5\wd\gfxbox}
\begin{table}[htbp!]
\centering
\caption{Baseline Results}
\label{tab:baselineresults}
\usebox{\gfxbox}
\par
\begin{minipage}{\wd\gfxbox}
\footnotesize
Notes: $n=8,316$.  All regressions include hospital and year fixed effects, and other hospital level controls include bed count, number of nurses, and number of other non-medical staff.  Market power variables are constructed using the overall hospital service area.  Large market is a binary variable for a hospital in the top half of the market size distribution.  In cases in which independent variables are missing, we recode them and control for missing variable indicators to ensure a balanced panel.  Standard errors are clustered at the hospital level.  *** p-value$<$0.01, ** p-value$<$0.05, * p-value$<$0.1.
\end{minipage}
\end{table}






\newpage
\savebox{\gfxbox}{
\scriptsize
\begin{tabular}{lc|lllll}
\hline\hline
Penalty 		& Mean Penalty Per Medicare 	& Log Mean		& Log Mean			& Log Medicaid 	   	& Log Medicare   		& Log Other  			\\
Quartile		&  Discharge [Range]	& Payment		& 	Net Charge		& Discharges      		& Discharges       	& Discharges        	\\
\hline
\multicolumn{7}{c}{Reference Category $=$ No Penalty or Bonus} \\

1	&$\$$6.00 &	0.004	&	0.01	&	-0.007	&	0.001	&	0.006	\\
	&[$\$$0.01, $\$$12.59] &	(0.006)	&	(0.009)	&	(0.025)	&	(0.008)	&	(0.012)	\\
2	&$\$$20.21&	0.020***	&	0.007	&	-0.053**	&	-0.018**	&	0.005	\\
	&[$\$$12.59, $\$$29.08]&	(0.006)	&	(0.009)	&	(0.024)	&	(0.008)	&	(0.013)	\\
3& $\$$41.77	&	0.014**	&	0.001	&	-0.061**	&	-0.035***	&	-0.006	\\
	& [$\$$29.15, $\$$57.06]&	(0.006)	&	(0.011)	&	(0.027)	&	(0.009)	&	(0.013)	\\
4& $\$$94.25&	0.024***	&	0.016	&	-0.085***	&	-0.085***	&	-0.036**	\\
	& [$\$$57.10, $\$$291.60]&	(0.008)	&	(0.013)	&	(0.030)	&	(0.012)	&	(0.015)	\\
\hline
\end{tabular}
}
\setlength{\captionmargin}{.5 \textwidth} \addtolength{\captionmargin}{-.5\wd\gfxbox}
\begin{table}[htbp!]
\centering
\caption{Intensive Margin Results}
\label{tab:int}
\usebox{\gfxbox}
\par
\begin{minipage}{\wd\gfxbox}
\footnotesize
Notes: $n=8,316$.  Results derived from breaking the size of the per Medicare discharge penalty into quartiles, with the omitted category as those hospitals receiving no penalty or a bonus.  All regressions include hospital and year fixed effects, and other hospital level controls include bed count, number of nurses, and number of other non-medical staff.  Market power variables are constructed using the overall hospital service area.  In cases in which independent variables are missing, we recode them and control for missing variable indicators to ensure a balanced panel.  Standard errors are clustered at the hospital level.   *** p-value$<$0.01, ** p-value$<$0.05, * p-value$<$0.1.
\end{minipage}
\end{table}





\newpage
\savebox{\gfxbox}{
\footnotesize
\begin{tabular}{llllll}
\hline	
\hline
 			& Log Mean 				& Log Mean			& Log Medicaid 	   	& Log Medicare   		& Log Other  			\\
			& Payment		& 	Net Charge	& Discharges      		& Discharges       	& Discharges    \\
	\hline
\multicolumn{6}{c}{1. Penalty Specific Trends} 											\\
\hline											

Net Penalty 	&	0.010**	&	0.019**	&	-0.038	&	-0.026***	&	-0.011	\\
			&	(0.005)	&	(0.008)	&	(0.023)	&	(0.007)	&	(0.012)	\\
p-value & 0.497 & 0.041 & 0.250 & 0.005 & 0.446 \\
\hline											
\multicolumn{6}{c}{2. Hospital, Year, and County Fixed Effects} 											\\
\hline											
Net Penalty 	&	0.015***	&	0.009	&	-0.048**	&	-0.027***	&	-0.003	\\
			&	(0.005)	&	(0.008)	&	(0.022)	&	(0.007)	&	(0.011)	\\
\hline											
\multicolumn{6}{c}{3. Controlling for Medicaid Expansion States} 											\\
\hline											
Net Penalty 	&	0.014***	&	0.008	&	-0.044**	&	-0.027***	&	-0.005	\\
			&	(0.005)	&	(0.008)	&	(0.021)	&	(0.007)	&	(0.010)	\\
\hline											
\multicolumn{6}{c}{4. Controlling for Overall HCAHPS Hospital Rating} 											\\
\hline											
Net Penalty 	&	0.014***	&	0.008	&	-0.045**	&	-0.026***	&	-0.003	\\
			&	(0.005)	&	(0.008)	&	(0.021)	&	(0.007)	&	(0.010)	\\
\hline											
\multicolumn{6}{c}{5. Dropping Fiscal 2012} 											\\
\hline											
Net Penalty 	&	0.012**	&	0.010	&	-0.045*	&	-0.028***	&	-0.007	\\
			&(0.005)		&	(0.009)		&(0.023)		&(0.007)	&	(0.012)	\\
\hline											
\multicolumn{6}{c}{6. Controlling for Case Mix} 		\\								
\hline											
Net Penalty 	&	0.014***	&	0.004	&	-0.044**	&	-0.026***	&	-0.005	\\
			&(0.005)		&	(0.008)		&(0.021)		&(0.007)	&	(0.011)	\\
\hline
\multicolumn{6}{c}{7. Omitting Hospital Fixed Effects} 		\\								
\hline
Net Penalty 	&	-0.061***	&	-0.049***	&	0.220***	&	0.094***	&	0.069***	\\
			& (0.015)		&	(0.018)		&(0.045)		&(0.026)	&	(0.022)	\\
\hline

\end{tabular}
}
\setlength{\captionmargin}{.5 \textwidth} \addtolength{\captionmargin}{-.5\wd\gfxbox}
\begin{table}[htbp!]
\centering
\caption{Robustness Checks}
\label{tab:robustness}
\usebox{\gfxbox}
\par
\begin{minipage}{\wd\gfxbox}
\footnotesize
Notes: Further controls include those in our baseline specification for mean payments.  The p-value in the first row of results is in reference to the null hypothesis that trends in the outcome of interest are the same between ever-penalized and never-penalized hospitals conditional on the model covariates.  In cases in which independent variables are missing, we recode them and control for missing variable indicators to ensure a balanced panel.  Standard errors are clustered at the hospital level.   *** p-value$<$0.01, ** p-value$<$0.05, * p-value$<$0.1.
\end{minipage}
\end{table}


\newpage
\savebox{\gfxbox}{
\footnotesize
\begin{tabular}{lllllll}
\hline\hline
 				&  Patient-Level & Log & Profit Index 	& Average DRG & Average  & Log Cost per     	\\
 				& 	Readmission & Charge & 			& Weight 		& LOS & 	Discharge 	\\
\hline
\hline							
Net Penalty  		& -0.001 & 0.004 &	0.002	&	0.004	& 	0.015  & -0.001 \\
				& (0.001) & (0.004) &	(0.001)	&	(0.004)	&	(0.012) & (0.001)  	\\
n				& 3,345,641 & 8,316 & 8,316& 8,316 & 8,316 & 8,238  \\
															
\hline
\end{tabular}
}
\setlength{\captionmargin}{.5 \textwidth} \addtolength{\captionmargin}{-.5\wd\gfxbox}
\begin{table}[htbp!]
\centering
\caption{Changes in Quality or Treatment Intensity}
\label{tab:other_results}
\usebox{\gfxbox}
\par
\begin{minipage}{\wd\gfxbox}
\footnotesize
Notes: All regressions include hospital and year fixed effects, and other hospital level controls include bed count, number of nurses, and number of other non-medical staff. In cases in which independent variables are missing, we recode them and control for missing variable indicators to ensure a balanced panel.  Standard errors are clustered at the hospital level.   *** p-value$<$0.01, ** p-value$<$0.05, * p-value$<$0.1.
\end{minipage}
\end{table}

\newpage
\savebox{\gfxbox}{
\footnotesize
\begin{tabular}{llllll}
\hline	
\hline
 			& Log Mean			& Log Mean				& Log Medicaid 	   	& Log Medicare   		& Log Other  			\\
			& Payment			& Net Charge				& Discharges      		& Discharges       	& Discharges    \\
	\hline
\multicolumn{6}{c}{Non-profit Hospitals}\\											
\hline											
Net Penalty 	&	0.015***	&	0.008	&	-0.046*	&	-0.029***	&	-0.011	\\
			&	(0.005)	&	(0.009)	&	(0.024)	&	(0.007)	&	(0.012)	\\
\hline											
\multicolumn{6}{c}{Non-Profit Hospitals with Penalty Specific Trends} 											\\
						\hline					
Net Penalty 	&	0.012**	&	0.015*	&	-0.039	&	-0.023***	&	-0.015	\\
			&	(0.005)	&	(0.009)	&	(0.026)	&	(0.007)	&	(0.014)	\\
p-value & 0.805 & 0.205 & 0.241 & 0.001 & 0.849 \\
\hline											
\multicolumn{6}{c}{For-profit Hospitals}\\											
\hline											
Net Penalty 	&	0.020	&	0.023	&	-0.018	&	-0.008	&	0.026	\\
			&	(0.014)	&	(0.021)	&	(0.050)	&	(0.018)	&	(0.020)	\\
\hline											
\multicolumn{6}{c}{For-Profit Hospitals with Penalty Specific Trends} 											\\
\hline											
Net Penalty	&	0.011	&	0.043*	&	0.002	&	-0.028	&	0.007	\\
			&	(0.014)	&	(0.023)	&	(0.050)	&	(0.017)	&	(0.020)	\\
p-value & 0.417 & 0.025 & 0.885 & 0.013 & 0.003 \\
\hline
\end{tabular}
}
\setlength{\captionmargin}{.5 \textwidth} \addtolength{\captionmargin}{-.5\wd\gfxbox}
\begin{table}[htbp!]
\centering
\caption{Results by Profit Status}
\label{tab:byprofit}
\usebox{\gfxbox}
\par
\begin{minipage}{\wd\gfxbox}
\footnotesize
Notes: All regressions include hospital and year fixed effects.  Further controls include those in our baseline specification for mean payments.  The p-values are in reference to the null hypothesis that trends in the outcome of interest are the same between ever-penalized and never-penalized hospitals conditional on the model covariates.  In cases in which independent variables are missing, we recode them and control for missing variable indicators to ensure a balanced panel.  Standard errors are clustered at the hospital level.   *** p-value$<$0.01, ** p-value$<$0.05, * p-value$<$0.1.
\end{minipage}
\end{table}





%\newpage
%\savebox{\gfxbox}{
%\footnotesize
%\begin{tabular}{lllllll}
%\hline	
%\hline
% 			& Log 				& Log				& Medicaid 	   	& Medicare   		& Private  			& Profit  \\
%			& Payment		& Charge			& Discharges      		& Discharges       	& Discharges        	& Index  \\
%\hline
%\multicolumn{7}{c}{Small Markets} \\
%\hline
%Net Penalty	&	0.003	&	-0.007	&	-0.049	&	-0.034 	&	-0.029	&	0.002	\\
%	&	(0.010)	&	(0.015)	&	(0.043)	&	(0.018)	&	(0.022)	&	(0.003)	\\
%\hspace{0.1in}  Med. Mkt. Share 	&	-0.004	&	0.030 	&	0.031	&	0.008	&	0.035	&	-0.003	\\
%	&	(0.012)	&	(0.018)	&	(0.046)	&	(0.020)	&	(0.024)	&	(0.004)	\\
%\hspace{0.1in}  High Mkt. Share 	&	0.009	&	-0.004	&	-0.002	&	0.014	&	0.030	&	0.000	\\
%	&	(0.012)	&	(0.019	)&	(0.051)	&	(0.022)	&	(0.024)	&	(0.004)	\\	
%Market Share	&	0.009	&	-0.056  	&	0.167  	&	0.216   	&	0.321   	&	0.004	\\
%\hspace{0.1in} Medium	&	(0.011)	&	(0.026)	&	(0.080)	&	(0.038)	&	(0.049)	&	(0.004)	\\
%Market Share	&	0.012	&	-0.047	&	0.294   	&	0.288   	&	0.428   	&	0.004	\\
%\hspace{0.1in} High	&	(0.017)	&	(0.035)	&	(0.102)	&	(0.054)	&	(0.063)	&	(0.006)	\\
%\hline
%\multicolumn{7}{c}{Large Markets} \\
%\hline
%Net Penalty	&	0.036   	&	-0.001	&	-0.072 	&	-0.022	&	0.010	&	0.000	\\
%	&	(0.012)	&	(0.018)	&	(0.040)	&	(0.018)	&	(0.025)	&	(0.003)	\\
%\hspace{0.1in}  Med. Mkt. Share 	&	-0.013	&	0.018	&	0.046	&	0.018	&	0.019	&	0.006 	\\
%	&	(0.013)	&	(0.018)	&	(0.038)	&	(0.018)	&	(0.022)	&	(0.003)	\\
%\hspace{0.1in}  High Mkt. Share	&	-0.022 	&	0.038 	&	0.051	&	0.024	&	-0.003	&	0.002	\\
%	&	(0.013)	&	(0.020)	&	(0.039)	&	(0.019)	&	(0.025)	&	(0.003)	\\	
%Market Share	&	-0.001	&	-0.049  	&	0.364   	&	0.360   	&	0.395   	&	-0.003	\\
%\hspace{0.1in} Medium	&	(0.016)	&	(0.021)	&	(0.057)	&	(0.047)	&	(0.060)	&	(0.005)	\\
%Market Share	&	0.009	&	-0.138   	&	0.515   	&	0.513   	&	0.688   	&	-0.005	\\
%High 	&	(0.019)	&	(0.032)	&	(0.082)	&	(0.065)	&	(0.097)	&	(0.007)	\\
%
%\hline
%\end{tabular}
%}
%\setlength{\captionmargin}{.5 \textwidth} \addtolength{\captionmargin}{-.5\wd\gfxbox}
%\begin{table}[!h]
%\centering
%\caption{Triple Differences by Market Share}
%\label{tab:bymktshare}
%\usebox{\gfxbox}
%\par
%\begin{minipage}{\wd\gfxbox}
%\footnotesize
%Notes: All regressions include hospital and year fixed effects. Further controls include those in our baseline specification for mean payments.  In cases in which independent variables are missing, we recode them and control for missing variable binary variables to ensure a balanced panel.  We split the sample by market size because of the strong negative correlation between market size and market share. Standard errors are clustered at the hospital level.
%\end{minipage}
%\end{table}




\newpage
\savebox{\gfxbox}{
\footnotesize
\begin{tabular}{lll}
\hline	
 		& Log Mean & Log Mean   				 \\
		& Payment & Net Charge\\
\hline
Net Penalty				&	0.039***	&	0.043***	\\
						&	(0.010)	&	(0.013)	\\
\hspace{0.1in}* Public Share 2 	&	-0.020*	&	-0.014	\\
						&	(0.012)	&	(0.014)	\\
\hspace{0.1in}* Public Share 3	&	-0.033**	&	-0.043***	\\
						&	(0.013)	&	(0.015)	\\
\hspace{0.1in}* Public Share 4	&	-0.044***	&	-0.070***	\\
						&	(0.013)	&	(0.016)	\\
Public Share 2				&	0.007	&	0.049***	\\
						&	(0.010)	&	(0.013)	\\
Public Share 3				&	0.016	&	0.087***	\\
						&	(0.011)	&	(0.016)	\\
Public Share 4				&	0.023*	&	0.157***	\\
						&	(0.012)	&	(0.018)	\\
\hline
\end{tabular}
}
\setlength{\captionmargin}{.5 \textwidth} \addtolength{\captionmargin}{-.5\wd\gfxbox}
\begin{table}[htbp!]
\centering
\caption{Triple Differences by Public Share}
\label{tab:publicshare}
\usebox{\gfxbox}
\par
\begin{minipage}{\wd\gfxbox}
\footnotesize
Notes: All regressions include hospital and year fixed effects.  Further controls include those in our baseline specification for mean payments.  The share of a hospital's patients insured by the public sector is broken into quartiles and interacted with penalty variables.  In cases in which independent variables are missing, we recode them and control for missing variable indicators to ensure a balanced panel.  Standard errors are clustered at the hospital level.  We restrict the sample to include at least 25 admissions per hospital per year.  *** p-value$<$0.01, ** p-value$<$0.05, * p-value$<$0.1.
\end{minipage}
\end{table}




\newpage
\savebox{\gfxbox}{
\footnotesize
\begin{tabular}{llllll}
\hline	
\hline
 & Log Mean 				& Log Mean			& Log Medicaid 	   	& Log Medicare   		& Log Other  			\\
			& Payment		& 	Net Charge	& Discharges      		& Discharges       	& Discharges    \\
\hline
\multicolumn{6}{c}{Hospitals Integrated Vertically with Physician Groups Prior to 2012} 											\\
\hline											
Net Penalty 	&	0.023***		&	0.017***	&	-0.036	&	-0.026**	&	0.008	\\
			&	(0.008)		&	(0.006)		&(0.032)		&(0.009)	&	(0.016)	\\
			\hline
\multicolumn{6}{c}{Hospitals Never Observed to be Vertically Integrated with a Physician Group} 											\\
\hline											
Net Penalty 	&	0.008		&	0.021***	&	-0.063**	&	-0.024**	&	-0.005	\\
			&	(0.007)		&	(0.012)		&(0.031)		&(0.010)	&	(0.015)	\\
			\hline
\end{tabular}
}
\setlength{\captionmargin}{.5 \textwidth} \addtolength{\captionmargin}{-.5\wd\gfxbox}
\begin{table}[htbp!]
\centering
\caption{Vertical Integration and Penalties}
\label{tab:VI}
\usebox{\gfxbox}
\par
\begin{minipage}{\wd\gfxbox}
\footnotesize
Notes: Empirical models are identical to those in Table \ref{tab:baselineresults}.  Standard errors are clustered at the hospital level.  *** p-value$<$0.01, ** p-value$<$0.05, * p-value$<$0.1.
\end{minipage}
\end{table}

\pagebreak


\savebox{\gfxbox}{
\footnotesize
\begin{tabular}{ccccccc}
\hline	
\hline
 							& Nervous  	& Respiratory  	 & Circulatory    & Musculoskeletal   		& Labor and & Neonatal \\
							&  System		&  System      	&  System     	&  System        			& Delivery   &	\\
\hline
Net Penalty 					& 0.021***		&	0.001 	&	0.019**	&	0.004	&	-0.001	&	0.016	\\
							& (0.010)		&	(0.011)	&	(0.008)	&	(0.007)	&	(0.005)	&	(0.010)	\\
\hline
n							& 1,410		&	1,758 	&	2,754  	&	3,060  	&	5,226	&	3,204	\\
Mean 						&13,762.86	&	12,015.13	&	13,071.17	&	12,981.58	&	11,308.56	&	8,911.19	\\
\hline
\end{tabular}
}
\setlength{\captionmargin}{.5 \textwidth} \addtolength{\captionmargin}{-.5\wd\gfxbox}
\begin{table}[htbp!]
\centering
\caption{Log Payments for Condition Specific Admissions}
\label{tab:eachcondition}
\usebox{\gfxbox}
\par
\begin{minipage}{\wd\gfxbox}
\footnotesize
Notes: All regressions include hospital and year fixed effects.  The dependent variable is the log of average payments for each associated acute care admission.  Further controls include those in our baseline specification for mean payments.  In cases in which independent variables are missing, we recode them and control for missing variable indicators to ensure a balanced panel.  Standard errors are clustered at the hospital level.  We restrict the sample to include at least 25 admissions per hospital per year.  *** p-value$<$0.01, ** p-value$<$0.05, * p-value$<$0.1.
\end{minipage}
\end{table}
















\end{document}
