\documentclass[12pt]{article}
\usepackage{graphicx,amssymb,amsmath,setspace,comment,verbatim,titling,pgf,lscape}
\usepackage[left=2cm,right=2cm,top=2.5cm,bottom=2cm]{geometry}
\usepackage[round]{natbib}
\usepackage{hyperref}
\usepackage{array}
\usepackage{bbm}
\usepackage{csquotes}


\usepackage[justification=centering]{caption}
\newcommand{\deriv}[2]{\frac{\mathrm{d}#1}{\mathrm{d}#2}}
%%\usepackage{breqn}
\newcommand{\pderiv}[2]{\frac{\partial#1}{\partial#2}}
%\usepackage{siunitx}
\newcolumntype{P}[1]{>{\raggedright\arraybackslash}p{#1}}
\hypersetup{colorlinks,%
						citecolor=black,%
						filecolor=black,%
						linkcolor=black,%
						urlcolor=blue,%
						}


\setlength{\droptitle}{-50pt}

\begin{document}

\title{Hospital Pricing and Public Payments}
\author{%
  Michael Darden, Ian McCarthy, and Eric Barrette\thanks{Darden: George Washington University. McCarthy: Emory University and NBER. Barrette: Health Care Cost Institute.  Correspondence: \href{mailto:darden@gwu.edu}{darden@gwu.edu}; \href{mailto:ian.mccarthy@emory.edu}{ian.mccarthy@emory.edu}; \href{mailto:ebarrette@healthcostinstitute.org}{ebarrette@healthcostinstitute.org}. We acknowledge the Health Care Cost Institute (HCCI), along with companies providing data (Aetna, Humana, and UnitedHealthcare) used in this analysis.}
}
\date{January 2018}

\maketitle

\begin{abstract}
A longstanding debate in health economics and health policy concerns how hospitals adjust prices with private insurers following reductions in public funding. A common argument is that hospitals engage in some degree of ``cost-shifting,'' wherein hospitals increase prices with private insurers in response to a reduction in public payments; however, evidence of significant cost-shifting is mixed, and the rationale for such behavior is unclear.  We enter this debate by examining plausibly exogenous variation in Medicare payment rates generated by two policies under the Affordable Care Act: the Hospital Readmission Reduction Program (HRRP) and the Hospital Value Based Payment (HVBP) program.  We merge rich hospital-level information to actual private-payer payment data from a large, multi-payer database. Our data include roughly 50\% of inpatient prospective payment hospitals in the United States from 2010 to 2015. We find that hospitals that faced net payment reductions from HRRP and HVBP were able to negotiate 1.6\% higher average private payments - approximately \$165 extra for the average acute care claim, or \$86,500 per hospital, based on an average hospital penalty of nearly \$153,000.  Consistent with a bargaining model in which hospitals are not solely profit maximizing, this result is completely driven by non-profit hospitals.  We find the largest increases in payments for circulatory system (2.5\%), nervous system (2.2\%), and neonatal (2.1\%) claims.  We also find significant heterogeneity by payer mix, where cost-shifting is largest for hospitals with higher shares of private insurance patients.
\end{abstract}
\noindent \textit{JEL Classification:} I11; I18; L2 \\\\
\noindent \textit{Keywords:} Hospital Behavior; Healthcare Prices; Medicare Payments; Cost-Shifting; Affordable Care Act\\\\
\setstretch{1.3}

\newpage
\section{Introduction}
A longstanding debate in health economics and health policy concerns a hospital's response to public payment reductions. Essentially two predictions are posited in the literature: 1) standard economic theory for a profit-maximizing firm with market power predicts that a reduction in public payments should put downward pressure on hospital prices as hospitals attempt to attract a larger share of private insurance patients; and 2) an alternative argument, termed ``cost-shifting'' and formalized in \cite{dranove1988}, is that hospitals pass public payment cuts to privately insured patients by negotiating for higher payments from private insurance companies. These arguments offer opposite predictions in terms of changes to hospital prices following a reduction in public payments. Identifying when, if at all, cost-shifting occurs is therefore critical to understanding the effects of future public payment reductions and in guiding future health policy more generally.

Insurance companies have long maintained that Medicare in particular is not paying its fair share \citep{frakt2011}, and the assumption of cost-shifting as common hospital practice is ubiquitous in health care policy debates. For example, in the debate over the Affordable Care Act, President Obama said:
\begin{quote}
\textit{``You and I are both paying 900 bucks on average - our families - in higher premiums because of uncompensated care.''\footnote{For additional examples, see the many excerpts in \cite{dranove2017}, including additional statements from President Obama and the U.S. Supreme Court regarding the Affordable Care Act.}- Barack Obama}
\end{quote}
And there is some evidence to support the notion of cost-shifting. Studying California hospitals from 1993-2001, \cite{zwanziger2006} estimate large effects on private payments due to reductions in Medicare and Medicaid payment rates, mirroring the findings of \cite{lee2003} and \cite{zwanziger2000}. \cite{zwanziger2006} estimate that cost-shifting can explain 12.3\% of the total increase in private payers' payments from 1997 to 2001.  Yet, in a systematic review of the literature, \citet{frakt2011} concludes that cost-shifting, if it exists, is not widespread and is not a main driver of increased health care costs.   While a simple model of monopoly pricing with price discrimination would suggest different prices for public and private patients \citep{hay1983}, that model would also suggest a \textit{decrease} in the private price for a decrease in public payments.  Furthermore, for both for-profit and non-profit firms, it is unclear why a hospital with the power to raise prices would not have already done so.\footnote{Several papers find zero or even negative price effects given public payment cuts. \cite{white2013} examines market level pricing from the Truven MarketScan data and finds that private prices decrease following reductions in Medicare payment rates. Exploiting a change in Medicaid payment policies in California, \citet{dranove1998} similarly found little evidence of cost-shifting. More recently, using the 2008 stock market collapse as an exogenous change to hospital endowments, \cite{dranove2017} find that the average hospital does not appear to cost-shift, with some evidence of cost-shifting among hospitals with sufficient market share.}

We enter the cost-shifting debate by exploiting the 2013 adoption of the hospital readmission reduction program (HRRP) and hospital value-based purchasing program (HVBP), which are both components of the Affordable Care Act's (ACA) cost containment strategy and which both serve as plausibly exogenous changes in Medicare payment rates.  Initially, the HRRP penalized hospitals for which 30-day readmissions for acute myocardial infarction (AMI), heart failure (HF), and pneumonia (PN) exceeded risk-adjusted thresholds constructed as a function of national averages.  Starting in Fiscal Year (FY) 2013 (October 2012-September 2013), hospitals faced a maximum cut in Medicare payments of 1\% across all Diagnosis Related Groups (DRG); the maximum potential penalty increased to 3$\%$ by 2015.  The \cite{cbo2010} estimates that HRRP will reduce hospital payments from Medicare by \$113 billion through 2019.  \citet{mellor2016} finds that HRRP has been associated with declines in AMI 30-day readmission rates, and \citet{gupta2016} finds a 5\% reduction in overall readmissions and 3\% reduction in all-cause mortality, mostly driven by quality improvements.\footnote{However, recent evidence from \citet{gupta2017} suggests that the HRRP may have increased heart failure mortality despite decreasing 30-day readmissions.}  By contrast, starting in FY 2013, HVBP reduces Medicare payments to all hospitals by 2\%, but it rewards hospitals with incentive payments for their quality of care over a variety of quality domains. As a result of these potential rebates, hospitals have the opportunity to receive a net payment increase if the rebates exceed the initial 2$\%$ reduction.  \citet{norton2016} provide evidence that quality of care improved as a result of HVBP, but only for services with the highest marginal incentives to improve quality of care.

An important contribution of our paper is the use of data on actual payments from private insurance firms to hospitals.\footnote{Throughout, rather than use the term ``price'', we refer to the financial transfer for a given procedure as the ``payment'' from a private insurance firm to a hospital.  A payment is distinctly different than a hospital ``charge,'' which effectively represents a hospital's list price for a give procedure. Private insurance firms negotiate substantial discounts from charges.}  Our data, maintained by the Health Care Cost Institute (HCCI), contain all claims made for acute care hospital admissions to three national commercial insurers. \cite{cooper2015} also uses HCCI data to examine broad trends in hospital pricing from 2007 through 2011, but to our knowledge, we are the first to use actual hospital-level payment data in a study of cost-shifting. These unique data include payments for every claim, which capture the negotiated payments between hospitals and insurers and which may differ substantially from charge-based estimates of payments often used in the literature \citep{dafny2009,dranove2017}. Indeed, in our data, the correlation between actual payments and a payment proxy estimated from the Healthcare Cost Report Information System (HCRIS) is 0.435, suggesting that charge-based estimates of payments may contain significant measurement error.  Furthermore, with payment data and a balanced panel of hospitals, we are able to investigate the extent to which hospital fixed effects adequately control for the mean payment to charge ratio within a hospital in a model of log charges \citep{cutler2000}. Our private insurance payment data cover approximately 28\% of individuals under the age of 65 who have employer-sponsored insurance (ESI). When merged with several other datasets on hospital characteristics and HRRP/HVBP penalties, our final analytic data constitute a balanced panel of 50\% of all inpatient prospective payment hospitals in the United States between 2010 and 2015.

Because payment changes under HVBP may be positive or negative (as opposed to HRRP, which are always zero or negative), we calculate the total change in payments from both policies and construct a binary variable that equals one if the net change is negative.  Our baseline empirical specification is a hospital fixed effects estimator in which we estimate the difference in average payments between those hospitals with a net penalty under the HRRP and HVBP relative to those not penalized before and after 2013. Our results reveal an increase in average payments of 1.6\% for hospitals penalized by these policies relative to those not penalized, equivalent to a roughly \$165 increase at the mean from 2013 through 2015. This equates to a total increase in private payments of approximately \$86,500 per hospital, based on an average hospital penalty of nearly \$153,000. Consistent with payment reductions, we find that penalized hospitals decreased Medicare discharges by 2.2\%, while discharges for private patients did not significantly change.

Our results are unbiased if there are no unobserved, time-varying factors that influenced payments that were also correlated with penalty status.  While we cannot directly test this assumption, three facts provide supporting evidence for a causal interpretation of our results.  First, the quality metrics that enter into construction of the HRRP and HVBP payment changes in a given year are calculated based on data from the previous three years.  Thus, in 2014 and 2015, our net penalty variable was largely pre-determined.  Even still, a competing explanation for our results may be that hospitals penalized under the HRRP and HVBP may have been able to negotiate higher private prices not because of penalties, but because of anticipated differential impacts of the full ACA;\footnote{There is evidence that hospitals located in low-SES areas were more likely to be penalized under HRRP, despite risk-adjustment at the patient level.} however, controls for whether a hospital operates in a state that expanded Medicaid, as well as the inclusion of county fixed effects, do not change our main results. Second, we demonstrate that allowing trends in payments to vary by whether a hospital ever experiences an HRRP/HVBP payment reduction does not significantly change our conclusions.  Finally, our results are robust to variety of specification checks, including dropping fiscal year 2012, when the policies affected only a subset of hospitals.  Collectively, our baseline results provide support for the notion of a small degree of cost-shifting in the modern health care environment.

Because theoretical predictions suggest that purely profit-motivated firms are unlikely to cost-shift \citep{hay1983}, economic rationales for cost-shifting have focused on a hospital's objective function.  \cite{dranove1988} models a hospital as a utility maximizer, where utility is defined over both profit and quantity.  A hospital may directly value quantity of care for reasons of altruism or prestige, or simply because a non-profit hospital must provide some form of ``community benefit'' to maintain its tax-exempt status.\footnote{The Congressional Budget Office defines community benefits as services geared toward ``promoting the health of any broad class of persons'' \citep{cbo2006}.}  Thus, if cost-shifting exists, evidence is anticipated to be isolated primarily among non-profit hospitals. We embed the model of \cite{dranove1988} within a bargaining context and show that this prediction extends to the modern managed care environment where hospitals negotiate prices with private insurers. Consistent with this prediction, when we break our analysis by profit status, we find that our baseline result of a 1.6$\%$ increase in private insurance payments is driven by non-profit hospitals.  We find an insignificant and economically smaller effect of a penalty on payments to for-profit hospitals.

Complicating matters, hospital payments are not merely set by hospitals, but rather they are negotiated with private insurance firms. In the context of such a negotiation, the mechanism for cost-shifting in response to \textit{penalties} incurred from lower-than-expected quality is particularly unclear. How can a penalty for low quality enable the hospital to negotiate a higher payment? We argue that three potential scenarios may allow for such an outcome: 1) if the quality information revealed by the penalty is not new information to the market, then the penalty is simply a reduction in public payment and the underlying source of the penalty is irrelevant; 2) if the information is new, hospitals may exploit the penalty in other service areas (i.e., those not tied to quality scores) where they have a comparative quality advantage in the market; or 3) hospitals with sufficient bargaining power may be able to translate public payment reductions into higher private insurance payments regardless of a negative quality signal. We address each of these mechanisms with a series of alternative specifications and hospital samples.

To address the first scenario, we add a control variable for the overall hospital quality ranking given annually by Hospital Consumer Assessment of Healthcare Providers and Systems (HCAHPS).  We find no difference in the effect of Medicare payment cuts on private insurance payments when comparing hospitals with identical overall quality rankings relative to omitting this control.  To address the second scenario, we construct measures of average payments within specific acute care admission categories. We re-estimate our preferred specification separately for each admission category,\footnote{We identify ``admission categories'' based on the major diagnostic category classifications.} and we find marked increases in payments for circulatory system (2.5$\%$), nervous system (2.2$\%$), and neonatal (2.1$\%$) claims, which suggest that cost-shifting may occur regardless of whether a specialty is directly related to the payment reductions cuts (such as AMI and HF within HRRP).  For the final scenario, given the lack of data on private insurance market shares at a local level, we follow \cite{wu2010} and proxy for hospital bargaining power using the hospital's share of private insurance patients. The intuition is that hospitals with the largest shares of private insurance patients may represent a more important client to private insurers, and hospitals may leverage this power when faced with public payment cuts. Consistent with this intuition, we find that hospitals with a larger share of private insurance patients do engage in more cost-shifting.

Because of the confidential nature of hospital/insurer bargaining, we cannot completely rule out that higher prices for penalized hospitals where due to other, unobserved factors that are correlated with penalties under HRRP and HVBP.  However, since researchers generally do not have access to the specific bargaining process, this limitation exists for much of the literature. Our results are also consistent across a number of specification checks. Taken together, our results demonstrate that a small but economically significant degree of cost-shifting occurred following the HRRP and HVBP.

Following a discussion of the setting and a detailed explanation of each policy in Section \ref{sec:Background}, we present our data, empirical methods, and baseline findings in Section \ref{sec:Empirical}. Section \ref{sec:Ext} considers several extensions and testable hypotheses implied by the standard theoretical model of cost-shifting, and Section \ref{sec:Conclusion} concludes.

\section{Background}
\label{sec:Background}

\subsection{Existing Evidence of Cost-shifting}
The debate over whether, and the extent to which, hospitals engage in cost-shifting has been ongoing for decades. While private insurers are naturally averse to higher private prices, hospitals have emphasized the need to cost-shift in an attempt to lobby for larger public payments. For example:
\begin{quote}
\textit{``Cost shifts have been a fact of hospital financial survival for decades.... The data show ...  how private payment is a mirror image of public payment over time and that the cost shift occurs. Hospitals must make up for shortfalls through a combination of approaches and cost-shifting is among them.'' -Rich Umbdenstock, Former President and CEO of American Hospital Association}\footnote{\href{http://blog.aha.org/post/costshifting-in-hospitals-}{``Cost-shifting in Hospitals,'' American Hospital Association Blog Post (AHA STAT), March 26, 2015.}}
\end{quote}

The argument that cost-shifting occurs is easily motivated by observed trends.  In 2015, 55 million Americans were enrolled in Medicare, up from 37.5 million in 1995, and from 1980 to 2014, the share of hospital costs attributable to Medicare rose from 34.6$\%$ to 40.2$\%$.  Meanwhile, in 2014, hospitals endured a shortfall of $\$$35 billion of Medicare payments relative to Medicare patient costs, as compared with a $\$$5 billion surplus relative to costs in 1997.  During this same period, patients insured by private payers became increasingly lucrative: in 2014, the payment-to-cost ratio of privately insured patients was roughly 140$\%$.\footnote{All statistics from the American Hospital Association Trendwatch Chartbook, 2016} Consistent with the trend in profitability of private insurance patients to hospitals, average premiums for covered workers with family coverage more than tripled from 1999 through 2017.\footnote{\href{https://www.kff.org/interactive/premiums-and-worker-contributions/?coverageGroup=family}{``Premiums and Worker Contributions Among Workers Covered by Employer-Sponsored Coverage, 1999-2017,'' Kaiser Family Foundation.}}

While the current conditions for cost-shifting appear to be ripe, much of the evidence of significant cost-shifting comes from the 1980s and 1990s.  For example, \cite{cutler1998costshift} studies cost-shifting during the phase-in of Medicare prospective payments during the 1980s, which resulted in an average 2$\%$ per year reduction in Medicare payments.  He found evidence of dollar-for-dollar cost-shifting.  More recently, \cite{zwanziger2006} study the late 1990s and found that, between 1997 and 2001, cost-shifting was responsible for roughly 12$\%$ of the observed increase in total private payer prices.

In contrast, the simplest argument \textit{against} cost-shifting as a significant mechanism in the hospital market is one of basic microeconomics.  A for-profit firm with market power who sells to two groups should not respond to an exogenous decline in the price to one group by raising prices to the second group.  \cite{hay1983} shows that, even when the government commits to reimbursing the full average cost of Medicare patients, hospitals will: 1) still charge a higher price to privately insured patients; and 2) respond to lower Medicare payments with \textit{lower} private prices.\footnote{\cite{dor1996} demonstrate that payer-specific marginal costs may be evidence of differential treatment by hospitals.}

In spite of the evidence presented by \cite{cutler1998costshift} and \cite{zwanziger2006}, the empirical evidence of cost-shifting is notably weak.  In a 2011 review of this literature, Austin Frakt states:
\begin{quote}
\textit{``In fact, as a whole, the evidence does not support the notion that cost-shifting is both large and pervasive. Instead, it reveals that cost-shifting can occur but may not always do so. When it has occurred, it has generally been measured at a rate far below dollar-for-dollar''}.
\end{quote}

Indeed, numerous studies have found zero or negative overall price effects, including \cite{dranove2008impact}, \cite{wu2010}, \cite{frakt2014}, and \cite{dranove2017}, but potentially important positive price effects for certain subgroups.  For example, \cite{wu2010} shows that hospitals with large shares of private patients (relative to Medicare patients) were able to cost-shift following the 1997 Balanced Budget Act, perhaps due to greater bargaining power.

We argue that identification of cost-shifting behavior is inherently difficult for three reasons.  First, the hospital market is incredibly complex.  In addition to many different types of payers, the industry is heavily regulated, and policy changes occur frequently.  We study two exogenous sources of public payment variation in which complexity is arguably a benefit to identification -- as discussed below, a common complaint among hospitals prior to the implementation of HRRP and HVBP was that the policies were opaque with respect to the payment reduction calculation.  Furthermore, because payment reductions were based on past quality metrics, hospitals could only hope to influence \textit{future} penalties once the programs were in place. Such a strategy is further complicated by the regular changes introduced to the programs vis-\`a-vis conditions and procedures being evaluated and the formulas by which HRRP/HVBP payments are calculated. Indeed, \cite{friedson2016} find that, within a relatively wide range of the HVBP penalty thresholds, whether a hospital ultimately underperforms in a given metric is largely random. Second, measurement error in private payments may be severe.  Because private payments are typically not observed, many of the referenced papers above must proxy for private payments, often with charges or costs, but we observe the actual dollar amount of paymentt from three large private insurers to hospitals.  Finally, heterogeneity in hospital responses to public payment cuts may muddle instances of important cost-shifting.  Indeed, as we emphasize below, our bargaining model of hospital payments predicts that cost-shifting will be largest in markets in which a hospital has the greatest \textit{relative} market power.  That is, the market power of a hospital in the provider market is large relative to the market power of any given insurer in the insurance market.  Yet our data do not include good measures of insurance market concentration at a local level.  We attempt to examine this heterogeneity with a series of alternative specifications and supplemental analyses.

\subsection{Policy Environment: HRRP and HVBP}
The adoption of the Medicare prospective payment system (PPS) in 1983, in which Medicare payments changed from pure fee-for-service to a capitated amount for each inpatient stay depending on diagnosis, generated incentives for hospitals to cut ``excessive'' procedures. PPS also created incentives for hospitals to discharge patients quickly.  By 2011, Medicare paid $\$$24 billion per year for 1.8 million hospital \textit{readmissions} -- admissions to any hospital within 30-days of discharge for the same condition.  While some readmissions are unavoidable, the HRRP was a cost containment in the ACA designed to levy penalties on hospitals with ``excessive'' readmissions.  Starting in 2013, hospitals with risk-adjusted readmissions in acute myocardial infarction, heart failure, and pneumonia that exceeded national comparison averages saw overall Medicare payment cuts of up to 1$\%$.   In 2015, the maximum penalty increased to 3\%, total penalties rose to \$420m \cite{rau2015}, and applicable conditions were expanded to include chronic obstructive pulmonary disease and total hip and knee replacements.  Evidence has been suggestive that HRRP has reduced readmissions in the tested conditions.  For example, \cite{mellor2016} found that HRRP was associated with declines in AMI readmission, which were not due to delay of treatment, changes in intensity, or selective patient mix. And \cite{gupta2017} find a 5$\%$ reduction in overall readmissions and 3$\%$ reduction in all-cause mortality, which was mostly driven by quality improvement.

By contrast, the Hospital Value-Based Purchasing (HVBP) program is rooted in a standard principal-agent model in which the principal (CMS in this case) contracts with agents (hospitals) to provide quality care to Medicare enrollees. The HVBP program scores hospitals based on their achievement (comparison to other hospitals) as well as their improvement (comparison to their own previous performance).  Similar to the HRRP, the HVBP bases changes in payments on past quality.  However, unlike the HRRP, the HVBP program is funded by reducing all hospitals' base operating Medicare severity diagnosis-related group (MS-DRG) payments by 2$\%$ and creating rebate incentives depending on defined quality metrics.  The program defines several quality domains and converts measures of quality within each domain to points, which are aggregated and mapped to a total point score.  The total point score determines the magnitude of the payment change, which may be positive or negative depending on if a hospital generates a rebate large enough to offset the 2$\%$ reduction.  Not surprisingly, \cite{norton2016} show that HVBP generated quality improvements in the quality domains with the highest marginal incentive to improve care.



\section{Initial Empirical Analysis}
\label{sec:Empirical}


\subsection{Data}
Our primary data come from three large health insurance firms that insure roughly 28$\%$ of all individuals under the age of 65 with employer sponsored health insurance over the period of 2010 through 2015.  To these data, we merge information on HRRP and HVBP penalties/rewards and other cost information from the Healthcare Cost Report Information System (HCRIS); hospital-level characteristics such as bed count, for-profit status, and system membership from the American Hospital Association (AHA) annual surveys; data on a hospital's payer mix (i.e., the number and share of Medicare, Medicaid, or private insurance patients) also from HCRIS; and county-level demographic characteristics from the American Community Survey (ACS).  We restrict our sample to community hospitals in urban areas and in the contiguous United States, with at least 30 staffed beds, at least 25 admissions in a given year in the HCCI data, and observed cost-reports from HCRIS.  Our final sample consists of 1,644 hospitals and 9,864 hospital/year observations.

Following \citet{gowrisankaran2015}, who use similar payment data from Northern Virginia, we aggregate payments to the hospital level by dividing the total payment for each claim by the appropriate DRG weight and regressing this amount on individual (claimant) and hospital fixed effects.  Using the estimated regression results, we predict the risk-adjusted mean hospital payment for a given year, which reflects the mean bargained payment. Table \ref{tab:summarystats} presents mean payments across hospitals over time. While average risk-adjusted payments received by hospitals increase roughly 5$\%$ between 2010 to 2015, shares of public (Medicare \& Medicaid) and private patients remain relatively stable over time.  Importantly, while shares remain stable, within-hospital patient mix may vary considerably over time as a function of public payments, which is why we treat payer-specific discharges as a separate dependent variable.  The last column of Table \ref{tab:summarystats} shows the fraction of hospitals subject to a net Medicare payment reduction.  Because the CMS fiscal year is from October through the following September, 30$\%$ of hospitals faced a penalty in calendar year 2012 (after October) because of discrepancies between the fiscal year of the hospital and that of CMS.  By 2015, 80$\%$ of hospitals faced some payment reduction. Among ever-penalized hospitals, the overall average penalty amount was \$152,800, which increased from \$135,150 in 2013 to \$231,408 in 2015.

Since our baseline empirical specification depends on within-hospital variation, we split our sample by whether a hospital ever faced a payment reduction under HRRP and HVBP during our sample period.  Table \ref{tab:bypenalty} presents summary statistics of our main dependent variable and some independent variables by ever-penalized status.  Log payments to never-penalized hospitals are marginally higher than to those penalized when averaged over the period of 2010 to 2015.  Non-profit hospitals constituted a much larger share of never-penalized hospitals, suggesting that non-profit hospitals may be of higher quality, at least in terms of HRRP and HVBP.  However, case mix is significantly more severe in the ever-penalized hospitals, which suggest that CMS risk-adjustment in HRRP and HVBP may not perfectly adjust penalty thresholds.  Ever-penalized hospitals tend to be in more competitive markets, have lower Medicare share and higher profits, and come from more heavily populated counties.  Evidence from Table \ref{tab:bypenalty} suggests that controlling for hospital fixed effects is important in models of hospital payments because of persistent differences between ever-penalized and never-penalized hospitals.

The log of the within-hospital mean of private insurance payments constitutes our primary dependent variable of interest. For brevity, we refer to this variable simply as the log mean payment. For comparison with the literature, we also follow \cite{dafny2009} in estimating hospital prices using the average net revenue for non-Medicare inpatient discharges. Specifically, we convert inpatient gross charges to inpatient net revenue by multiplying the hospital's total net revenues by the total gross charge ratio. Payments for Medicare inpatient services are then subtracted from inpatient net revenue to arrive at inpatient revenues from all non-Medicare patients, which we divide by the hospitals discharges to derive the per discharge net revenue amount. Since Medicaid revenues are not provided in HCRIS, the measure is a weighted average of net revenue per discharge for commercially insured and Medicaid patients where the weights equal the share of inpatient discharges belonging to each payer. This same measure has been used in recent studies examining hospital pricing behavior, including \cite{schmitt2017} and \cite{lewis2015}. To eliminate outliers, we trim the lower and upper tails at the 5th and 95th percentile of the resulting price distribution, and we normalize this estimated price based on the hospital's observed case mix index (CMI) from the inpatient prospective payment system (IPPS) final rule files. To differentiate this measure of price from our observed payments from the HCCI data, we refer to this measure as the log charge-based payment.

Finally, since standard theory of a for-profit firm suggests that the number of public insurance patients decreases following a reduction in the administrative price, we also include measures of payor mix as an additional set of outcomes. These measures include the log number of Medicare discharges, the log number of Medicaid discharges, and the log number of private insurance discharges. We also considered the Medicare, Medicaid, and private insurance shares (rather than log counts). Those results are excluded for brevity but qualitatively similar to the analysis of log counts.

\subsection{Regression Analysis}
Our baseline empirical specification isolates within-hospital variation in private payments over time by whether a hospital faced a net penalty from the HRRP and HVBP. This analysis therefore focuses on the extensive margin of penalties.  Equation 1 presents our main empirical model:
\begin{equation}
\label{eq: reg}
y_{ht} = \alpha_{h} + x^{'}_{ht}\beta + \delta1[Penalty]  + \theta_{t}  +  \epsilon_{ht},
\end{equation}
where outcome $y_{ht}$ at hospital $h$ in fiscal year $t$ is a function of a hospital specific intercept, $\alpha_{h}$; a vector of time-varying hospital and market-level exogenous characteristics, $x_{ht}$; an indicator for a net penalty under both of the policies; controls for year effects, $\theta_t$; and an i.i.d. error term $\epsilon_{ht}$.  Because the penalty indicator is zero for all hospitals in 2010 and 2011, and because we include hospital fixed effects, Equation 1 represents a difference-in-differences estimator. Our parameter of interest, $\delta$, captures the extent to which hospitals penalized under the ACA receive differential private payments relative to hospitals with no penalty.  For a causal interpretation of $\delta$, the underlying assumption in Equation 1 is that there are no time-varying unobserved characteristics that differentially affect payments in penalized hospitals relative to non-penalized hospitals, an assumption that we address in the next subsection.

Table \ref{tab:baselineresults} presents results from Equation 1 for the log of mean payments, the log of charge-based payments, and several (logged) payer-specific discharge variables. The first column of Table \ref{tab:baselineresults} demonstrates that hospitals that faced payment reductions increased payments by 1.6$\%$ over the period of 2012-2015.  This represents a roughly $\$$165 increase in the average payment at the mean payment reported in Table \ref{tab:summarystats}.  Column 2 presents estimates from a similar model in which we replace negotiated payments with the log of charge-based payments as discussed previously \citep{dafny2009,lewis2015,schmitt2017,dranove2017}. Results in column 2 suggest a smaller and statistically insignificant change in log charge-based payments for penalized hospitals, which we argue demonstrates the importance of using actual payment data.  Columns 3 and 4 of Table \ref{tab:baselineresults} show movement \textit{away} from Medicare and Medicaid patients for penalized hospitals, with discharges decreasing by 3.7$\%$ and 2.2$\%$, respectively.  Both of these effects are significant at the 90$\%$ and 99\% level of confidence, respectively.

\subsection{Robustness}
Results in Table \ref{tab:baselineresults} reflect the causal effects of the HRRP and HVBP penalties if there are no unobserved, time-varying factors that influence our outcomes and are also correlated with penalty status.  While we cannot completely rule out this possibility, we estimate a variety of alternative specifications in order to examine the influence of several potential confounders.

First, we allow the trend in outcomes to vary by whether a hospital is ever-penalized. Differential trends conditional on penalty status and other controls would be suggestive of time-varying unobserved heterogeneity and would generally bias our estimate of $\delta$ towards zero.  These results are summarized in row 1 of Table \ref{tab:robustness}, where the results are broadly consistent with Table \ref{tab:baselineresults}.

Second, we are concerned that unobserved differences across markets (e.g., with regard insurer market power) may influence our estimates. We therefore include a set of county-level fixed effects, with results summarized in row 2 of Table \ref{tab:robustness}. Here, we continue to find positive and significant effects on private insurance payments, as well as significant reductions in the log number of Medicare discharges. These results suggest that local area variation in provider or insurer markets are not driving our results.

Third, we remain concerned that other changes in the hospital-insurer relationship may drive payments after the full implementation of the ACA in 2015.\footnote{A model in which we dropped 2014 and 2015, relying on only 2013 as our ``post'' period found an estimate of $\delta$ of 0.08, which was not significant.  Given that, to cost-shift, a hospital must renegotiate its payments, we argue that one year after the policy is insufficient to detect a significant effect.  Furthermore, because hospitals complained about uncertainty with respect to payment cuts associated with the ACA, we argue that it was unlikely that hospitals anticipated the full payment cut prior to the implementation of HRRP and HVBP}. We therefore consider an alternative specification in which we include an indicator for whether the hospital was in a Medicaid expansion state as of 2014. These results are presented in row 3 of Table \ref{tab:robustness} and are largely unchanged from our initial estimates.

Fourth, since the HRRP and HVBP are intended to reward and/or punish hospitals based in-part on quality of care, a hospital's ability to translate HRRP and HVBP penalties into higher private insurer payments may depend on whether such penalties reveal new quality information to the market. Existing findings from \cite{dranove2008} and others tend to find relatively small effects of quality reporting on hospital choice. As \cite{dranove2008} state, ``report cards do not always convey `news' about quality; in some cases the rankings confirm with prior beliefs about quality.'' To the extent that penalties from the HVBP and HRRP do not reveal any new information to the market, then the penalty acts simply as a reduction in public payments and the standard arguments for cost-shifting apply. We examine this issue with an alternative specification in which we control for a hospital's overall quality as measured by patients' overall hospital rating from the Hospital Consumer Assessment of Healthcare Providers and Systems (HCAHPS).  Row 4 of Table \ref{tab:robustness} reports results from this model, with estimates almost identical to those in Table \ref{tab:baselineresults}.

Fifth, because of discrepancies in the timing of hospital fiscal years (both across hospitals and with CMS), the exact timing of the realization of Medicare payment cuts varies across our sample. In row 5 of Table \ref{tab:robustness}, we report results from a model in which we drop fiscal year 2012 from our analysis. Again, our results are similar to those in Table \ref{tab:baselineresults}.

Finally, the novel aspect of our data are that, for a given acute care claim, we observe the actual payment from a private insurer to the hospital. Private insurance payments reflect some endogenously bargained discount on the charge or markup relative to Medicare payment rates and are therefore fundamentally different from charges, which reflect a hospital's list price for a given service. As noted above, the correlation in our data between mean payments and charge-based payments is 0.435, which suggests that measurement error in a model of log charge-based payments could be significant. Many studies of hospital pricing proxy for payments with hospital charges and argue that hospital fixed effects control for mean differences between charges and payments \citep{cutler2000}. The last row of Table \ref{tab:robustness} presents results of Equation 1 without hospital fixed effects for each of our dependent variables.  Estimates of $\delta$ for both log mean payments and log charge-based payments are negative and significant. Relative to our initial results, these findings suggests that: 1) permanent and unobserved hospital-level heterogeneity is an important driver of outcomes in our setting; and 2) hospital fixed effects may in fact go a long way toward controlling for mean differences between charges and payments. However, we emphasize the importance of payment data with respect to the precision and measurement of hospital-insurer bargaining, noting the lack of statistical significance in our model of log charge-based payments presented in Table \ref{tab:baselineresults}. Ultimately, these results offer some assurance that findings of a significant effect using charge-based estimates of prices are indeed reflective of a true price increase, while findings of an insignificant effect may be driven by incorrect inference (e.g., due to measurement error) or due to a true underlying null effect.

\section{Empirical Extensions}
\label{sec:Ext}

\subsection{For-profit versus Not-for-profit}
As initially examined in \cite{dranove1988}, a hospital may pursue a cost-shifting strategy if the hospital's objective function includes something other than pure profit (e.g., if the hospital receives direct utility from the quantity of services provided). For this reason, cost-shifting is thought to more likely occur among non-profit hospitals, if at all. Indeed, to maintain their tax exempt status, non-profit hospitals are required by the IRS to provide community benefits.\footnote{Of course, this does not mean that non-profit hospitals are fully altruistic. In fact, evidence on non-profit hospital behavior relative to for-profit hospital behavior is mixed. For example, \cite{silverman2004} and \cite{dafny2005} find evidence that non-profits ``upcode'' less frequently, while \cite{gaynor2003} find that non-profit hospitals have lower marginal costs but higher markups than for-profit hospitals.} Since over 80$\%$ of hospitals in our sample are non-profit, this implies that the objective function of the majority of hospitals in our analysis extends beyond pure profit-maximization.  Importantly, the model posited in \cite{dranove1988} assumes that hospitals set payments unilaterally, and it is not immediately clear whether this prediction extends to a modern managed care market in which hospitals and private insurers negotiate over private prices.

To more formally examine the presence of cost-shifting in a bargaining context, we embed the hospital cost-shifting model from \cite{dranove1988} in a hospital-insurer bargaining model similar to that in \cite{ho2017} (HL), \cite{gowrisankaran2015}, \cite{lewis2015}, and \cite{dor2004}. Specifically, we consider a non-profit hospital whose objective is to maximize a function of profits and quantity of care provided, denoted by
\begin{equation}
 U\left( \pi_{j} = \sum_{i=1}^{N_{j}} \pi_{i,j}^{h} + \pi_{g,j}^{h}, \sum_{i=1}^{N_{j}} D_{i,j}^{h}, D_{g,j}^{h} \right),
\label{eqn:nfp_objective}
\end{equation}
where $\pi_{j}$ denotes total profits for hospital $j$ and $D_{i,j}^{h}$ denotes hospital demand from insurer $i$. Following HL, we assume $$\pi_{i,j}^{h}=D_{i,j}^{h}(p_{i,j}-c_{i}),$$ where $p_{i,j}$ denotes the negotiated payment between insurer $i$ and hospital $j$. We also follow HL in assuming that patients are ``unaware or unable to determine their [financial] liability prior to choosing their provider.'' In other words, the negotiated payment $p_{i,j}$ does not affect demand for a specific hospital. The subscript $g$ denotes public (or government) insurers, for which the payment is administratively set at $p_{g}$. Finally, again following HL, we assume that profits for insurer $i$ are
\begin{equation}
\pi_{i}^{M} = D_{i} \left( \theta_{i} - \eta_{i} \right) - \sum_{j=1}^{N_{i}} D_{i,j}^{h} p_{i,j},
\label{eqn:ins_profit}
\end{equation}
where $D_{i}$ denotes the number of enrollees for insurer $i$, $\theta_{i}$ denotes the insurer's premiums, $\eta_{i}$ denotes insurer costs per-enrollee other than inpatient hospital care, and $D_{i,j}^{h} p_{i,j}$ reflects payments to hospitals for care provided to the insurer's enrollees.

The negotiated price between hospital $j$ and insurer $i$ is such that
\begin{equation}
 p_{ij}= \arg \max_{p_{ij}} \left(\triangle U_{j} \right)^{b_{j}} \times \left(\triangle \pi^{M}_{i} \right)^{1-b_{j}},
 \label{eqn:neg_price}
\end{equation}
where $\triangle U_{j}$ denotes the change in hospital $i$'s utility from reaching an agreement with insurer $i$, and similarly $\triangle \pi^{M}_{i}$ denotes the change in insurer profits from an agreement with hospital $i$. $b_{j}$ denotes the bargaining weight of hospital $j$, expressed as the weight to which the hospital's payoffs are given in the overall net value.

The first order condition for equation \ref{eqn:neg_price} can be simplified to
\begin{equation}
 b_{j} \triangle \pi_{i}^{M} \pderiv{U_{j}}{\pi_{ij}^{h}} - (1-b_{j}) \triangle U_{j} = 0.
\label{eqn:price_foc}
\end{equation}
Applying the implicit function theorem yields the relevant comparative static:
\begin{equation}
\deriv{p_{ij}}{p_{g}} = \frac{- b_{j} \triangle \pi_{i}^{M} \pderiv{^{2}U_{j}}{\pi_{j}^{2}}D_{g}^{h}}{D_{ij}^{h}\left(b_{j} D_{ij}^{h} \pderiv{^{2}U_{j}}{\pi_{j}^{2}} - (1-b_{j}) \pderiv{U_{j}}{\pi_{j}} \right)}.
\label{eqn:comp_static}
\end{equation}
We can see immediately from Equation \ref{eqn:comp_static} that $\deriv{p_{ij}}{p_{g}}<0$ whenever $\pderiv{^{2}U_{j}}{\pi_{j}^{2}}<0$. Hospitals must therefore have some diminishing marginal utility of profits for cost-shifting to occur. Interestingly, this result exists without hospitals deriving utility directly from quantity of care provided, which is necessary for cost-shifting to occur in \cite{dranove1988}. Nonetheless, this condition is intuitively more likely to hold for not-for-profit hospitals relative to for-profit hospitals.

To investigate the extent to which cost-shifting varies by profit status, we re-estimate Equation 1 separately for non-profit and for-profit hospitals.  The results are presented in Table \ref{tab:byprofit}.  Consistent with our theoretical prediction, the results show that the effects of HRRP and HVBP penalties on cost-shifting are isolated among non-profit hospitals, with a statistically insignificant (and smaller in magnitude) effect among for-profit hospitals. The estimated coefficients between for-profit and non-profit hospitals are also significantly different from each other at all standard levels of significance. In rows 2 and 4 of Table \ref{tab:byprofit}, we allow for differential trends by penalty status -- analogous to the overall results in row 2 of Table \ref{tab:robustness}. These results are even more telling in terms of the magnitude of the difference in the estimates between for-profit and non-profit hospitals, with point estimates among for-profit hospitals dropping by more than half. Effects on private insurance payments to non-profit hospitals, however, remain positive and significant.


\subsection{Differential Effects by Service Area}
As discussed previously, the underlying mechanisms for cost-shifting are unclear in the context of hospital-insurer bargaining. In particular, the reduction in public payments that serve to identify the presence of cost-shifting in our analysis derive from a lower-than-expected performance on some set of quality metrics. For cost-shifting to occur, hospitals must translate this signal into higher prices. We presented evidence in Table \ref{tab:robustness} that our estimates are robust to any reputation effects from the HRRP and HVBP penalties as measured by patients' hospital ratings; however, the quality signal may be uninformative to patients while potentially informative to managed care insurers. In this case, penalized hospitals may instead target other, unrated service areas where they may maintain a comparative quality advantage in the market. To investigate this further, we estimate models of the log of payments within acute care admission service categories.

Results for $\hat{\delta}$ are presented in Table \ref{tab:eachcondition} for several categories of acute admissions. For each admission category, we restrict our sample to hospitals with at least 25 admissions in that category in each year of our sample. Table \ref{tab:eachcondition} demonstrates significant increases in payments for nervous, circulatory, and neonatal admissions, suggesting that cost-shifting may occur for multiple types of admissions. Because two of the three original conditions rated by the HRRP (AMI and heart failure) were circulatory system conditions, an open question remains as to how hospitals may negotiate higher prices for these conditions. It could be that the average increase in circulatory system prices is driven by conditions other than AMI or heart failure (e.g., coronary artery disease or stroke), or it could be that the penalty among hospitals that ultimately negotiated higher circulatory system prices was driven by lower-than-expected performance in pneumonia patients rather than AMA or heart failure patients. Because of limited sample sizes for such hospitals and conditions, we cannot examine these questions directly in our data (both due to restrictions on dissemination of small sample size results and due to large variability in prices for infrequent procedures).


\subsection{Differential Effects by Market Power}
From Equation \ref{eqn:comp_static}, a hospital's incentive to cost-shift is larger as the insurer's outside option decreases (i.e., $\triangle \pi_{i}^{M}$ increases) and as the number of public-payer patients increases, $D_{g}^{h}$, while the incentive to cost-shift is reduced if the hospital receives a larger number of patients from insurer $i$. Practically, this suggests that hospitals will be more likely to cost-shift if they have some \textit{relative} market power, where the insurer is heavily dependent on the hospital but where the hospital does not receive a large number of patients from any given insurer. In the parlance of \cite{lewis2015}, relative market power describes the hospital's ``bargaining position,'' while a hospital's ``bargaining power'' is reflected by $b_{j}$.\footnote{While bargaining power and bargaining position may seem to be related, \cite{lewis2015} finds that bargaining power is not significantly affected by the hospital's relative market power or direct market share.}

To investigate, we attempt to proxy for a hospital's relative market power by constructing the quartile of the hospital's share of public patients relative to total patients, and we interact our penalty variable with indicators for each quartile. This analysis is similar to that of \cite{wu2010}, who intuits that a hospital with a large share of private payers represents a more important client for the insurance market, and the hospital may leverage this power when public payments are cut. Applying this intuition to a study of hospital cost-shifting following the Balanced Budget Act of 1997, \cite{wu2010} finds that hospitals with larger shares of private patients were more able to pass Medicare payment reductions on to private payers.

Results are presented in Table \ref{tab:publicshare} and suggest that our initial estimate of cost-shifting is driven by hospitals with the smallest share of public patients. Indeed, the first column of Table \ref{tab:publicshare} demonstrates that payments increased by 3.9$\%$ for hospitals with the smallest share of public patients. This increase was nullified for hospitals in the third and fourth quartile of public patient shares.

We note that predictions involving a hospitals' relative market power are less clear when we incorporate the insurer's choice of premiums in the insurance market. If the insurance market is heavily concentrated, then insurers can pass health care price changes to their plan enrollees \citep{trish2015,ho2017}. This intuition leads to conflicting conclusions: cost-shifting is likely to occur when insurers have a particularly small market share (such that hospitals can leverage market power), but perhaps also when insurers have a particularly large market share (such that insurers can pass price increases on to insureds). The role of insurance markets on the prevalence or magnitude of cost-shifting is therefore empirically difficult to measure without detailed data on insurance premiums and insurer market shares (at a local level). Because we lack reliable information on local area insurance concentration, we leave as an open question the extent to which cost-shifting is more prominent in markets with both provider market power and a concentrated insurance market.

\section{Conclusion}
\label{sec:Conclusion}
This paper uses novel payment data from a large, multi-payer database to investigate the extent to which hospitals, faced with public-sector payment cuts, compensate by negotiating for higher payments from private insurers.  We use variation in Medicare payments generated by two cost-containment policies within the ACA -- the hospital readmissions reduction program and the hospital value base purchasing program -- to estimate the role of a net public payment reduction on average hospital payments.  Our results suggest support for a modest degree of ``cost-shifting,''  where the change in payments for hospitals facing a net payment reduction was 1.6$\%$ higher than that for non-penalized hospitals. This result is robust to differential time trends among penalized versus non-penalized hospitals, reputation effects as measured by HCAHPS responses, and Medicaid expansion in 2014 as part of the ACA.

Motivated by a simple extension of \cite{dranove1988} to a bargaining framework, we extended our empirical analysis to consider the underlying mechanisms allowing for cost-shifting. First, our theoretical model suggests that cost-shifting will only occur in the presence of diminishing marginal utility of profits. Similar to \cite{dranove1988}, we argue that this is more likely to be the case for non-profit hospitals than for-profit hospitals. Consistent with theoretical predictions, the presence of cost-shifting is driven by non-profit hospitals, and we find no significant effect in for-profit hospitals.

We then considered the mechanisms for cost-shifting specifically in the context of our data. In particular, our empirical analysis identifies cost-shifting from variation across hospitals (on the extensive margin) in penalties levied under the HRRP and HVBP programs. The presence of such a penalty suggests that the hospital has under-performed in some way relative to an average hospital, and it is unclear how a hospital could translate this underperformance into \textit{higher} prices. We intuit two potential mechanisms: 1) by increasing prices for services unrelated to the areas in which the hospital was evaluated and ultimately penalized; and 2) by leveraging market power as predicted in our theoretical model. We find some evidence in favor of each mechanism, where hospitals do appear to increase prices in areas unrelated to the HRRP and HVBP (but not exclusively in such areas) and where cost-shifting is also largest among hospitals with higher relative market power.

Since our outcome is calculated as an average payment per patient, a remaining question concerns whether our estimates reflect true price increases or changes in the intensity of treatment. While we cannot examine this question directly, we find little evidence that hospitals are increasing treatment intensity. For example, we followed \cite{horwitz2009} in constructing a set of indicators for ``profitable'' (e.g., angioplasty or neonatal intensive care) versus ``unprofitable'' (e.g., alcohol dependency services or hospice care) hospital services.\footnote{A full list of relatively profitable and relatively unprofitable services is provided in Table 2 of \cite{horwitz2009}. Following their analysis, we identify whether a hospital offers these services based on responses from the AHA annual surveys.} We then constructed a ``profitable services index'' calculated as the ratio of profitable services to all profitable and unprofitable services identified by \cite{horwitz2009}. For example, if the hospital offered 2 profitable services and 2 unprofitable services, then the ratio for this hospital would be 50\%. Treating this profitable services index as an additional outcome and repeating our analysis from Section \ref{sec:Empirical}, we find small and insignificant effects of being penalized. These insignificant effects persist when examining for-profit and not-for-profit hospitals separately as well as across all robustness checks presented in Table \ref{tab:robustness}. A similar pattern emerges if we consider average length-of-stay and average DRG weights (among our commercial insurance population) as separate outcomes, with insignificant effects of HRRP/HVBP penalties on these outcomes in all specifications considered. These results therefore suggest that hospitals do not respond to HRRP/HVBP penalties by shifting service offerings disproportionately toward more profitable services or by treating patients more intensively. Indirectly, the results therefore support the hypothesis that our estimated increase in payments derives from some underlying increase in actual prices.


\newpage
\bibliographystyle{authordate1}
\bibliography{BibTeX_Library}


\clearpage
\newpage
\appendix
\section{Tables}
\label{app:tables}

\newsavebox{\gfxbox}
\savebox{\gfxbox}{
\begin{tabular}{cccccccc}
\hline \hline
%\multicolumn{9}{c}{}\\
Fiscal & Sample 		&  Payment $\$$					& Public  	   & Medicare   & Medicaid  	& Private & Percent \\
Year   &  Size    		&  Mean (St. Dev.) 				& Share        & Share     	& Share        	& Share   & Penalized \\
 \hline
2010 & 1,644			& 	10,250.60   (8,940.35)	& 0.48   &   0.34   &   0.14    &  0.52  & 0.00  \\
2011 & 1,644 		& 	10,420.89   (4,965.80)		& 0.47   &   0.34    &  0.13    &  0.53  & 0.00   \\
2012 & 	1,644 		& 	10,147.18   (4,468.36) 	&  0.46  &    0.33   &   0.13   &   0.54  & 0.30   \\
2013 & 	1,644		& 	10,118.09   (6,370.39)	& 0.45   &   0.33    &  0.12   &   0.55  & 0.74  \\
2014 & 	1,644 	&	10,343.40   (4,742.95)		& 0.45  &    0.33   &   0.12   &   0.55 & 0.76 \\
2015 &    1,644 & 10,782.29   (5,670.77)		&	0.45  &    0.33   &   0.12   &   0.55& 0.80 \\
\hline
Total & 	9,864		& 10,338.67   (6,083.84)	&	0.45  &    0.33   &   0.12   &   0.55& 0.42 \\
\hline
\end{tabular}
}
\setlength{\captionmargin}{.5 \textwidth} \addtolength{\captionmargin}{-.5\wd\gfxbox}
\begin{table}[!h]
\centering
\caption{Characterization of Research Sample over Time}
\label{tab:summarystats}
\usebox{\gfxbox}
\par
\begin{minipage}{\wd\gfxbox}
\footnotesize
Notes: Balanced panel of hospitals over time between 2010 and 2015.  Payment represents the mean dollar amount paid to a hospital in a year over all acute care admissions.  Share variables are measured at the hospital level.  Penalty is a binary variable for whether the combination of HRRP and HVBP resulted in a net payment reduction.
\end{minipage}
\end{table}



\newpage
\savebox{\gfxbox}{
\begin{tabular}{lccc}
\hline \hline
Variable 	& Never 				& Ever  				&   	  \\
		   		&  Penalized    		& Penalized			&    p-value   				\\
 \hline
Log(Payment)					& 9.323			& 9.299  			& 0.040\\
System  Membership      	&       0.669    	&     0.765   	&     0.000\\
Non-profit     &       0.839    &    0.706    &     0.000\\
Log(Case Mix Index)        &       0.422      &   0.446      &   0.000\\
\multicolumn{4}{l}{Local Hospital}\\
\hspace{0.05in} Monopoly  &      0.150    &     0.113    &     0.000 \\
\hspace{0.05in} Duopoly    &     0.226    &     0.154   &      0.000\\
\hspace{0.05in} Triopoly    &    0.112     &     0.108     &     0.655\\
\multicolumn{4}{l}{Market Share}\\
\hspace{0.05in} Medicare  &      0.347  &       0.329    &     0.000\\
\hspace{0.05in} Medicaid    &     0.117      &    0.126    &     0.004\\
\hspace{0.05in} Public      &    0.465     &    0.456     &    0.016\\
\hspace{0.05in} Private     &      0.535    &     0.544    &     0.016\\
Total Pop. (1000s)     &     710.362   &  1166.051   &     0.000 \\
\multicolumn{4}{l}{County Age Distribution}\\
\hspace{0.05in}[18, 34]      &    0.240   &     0.239    &    0.406\\
\hspace{0.05in}[35, 64]   &      0.396    &    0.392   &     0.000\\
\hspace{0.05in}$>$65    &   0.132    &    0.131     &   0.189\\
\multicolumn{4}{l}{County Race Distribution}\\
\hspace{0.05in}White     &     0.743   &     0.733    &    0.026\\
\hspace{0.05in}Black    &     0.151  &      0.137   &     0.000\\
\multicolumn{4}{l}{County Income Distribution}\\
\hspace{0.05in}$<$ \$50k   &    0.182   &     0.180      &  0.000\\
\hspace{0.05in}[\$50k, 75k]   &   0.126   &     0.123    &    0.000\\
\hspace{0.05in}[\$100k, 150k]       & 0.138     &   0.132  &      0.000\\
\hspace{0.05in}$>$ \$150k   &    0.108    &    0.101 &       0.000\\
\multicolumn{4}{l}{County Education Distribution}\\
\hspace{0.05in}High School Only   &     0.272   &     0.272   &     0.924\\
\hspace{0.05in}Bachelor's Only      &     0.196    &    0.190   &     0.000\\
\hline
\end{tabular}
}
\setlength{\captionmargin}{.5 \textwidth} \addtolength{\captionmargin}{-.5\wd\gfxbox}
\begin{table}[!h]
\centering
\caption{Hospital Characteristics by Penalties}
\label{tab:bypenalty}
\usebox{\gfxbox}
\par
\begin{minipage}{\wd\gfxbox}
\footnotesize
Notes: Summary statistics are split by whether a hospital is ever observed to receive a net penalty in 2012-2015. Payment represents the mean dollar amount paid to a hospital in a year over all acute care admissions.   County level characteristics are from the American Community Survey.
\end{minipage}
\end{table}

\newpage
\savebox{\gfxbox}{
\scriptsize
\begin{tabular}{llllll}
\hline\hline
 			& Log Mean				& Log Charge-				& Log Medicaid 	   	& Log Medicare   		& Log Private  			\\
			& Payment		& based Payment			& Discharges      		& Discharges       	& Discharges        	\\
\hline
Penalty  										&	0.016***	&	0.010	&	-0.037*	&	-0.022***	&	-0.001	\\
													&	(0.005)	&	(0.008)	&	(0.020)	&	(0.006)&	(0.010)	\\
\hline
\multicolumn{6}{l}{Hospital Characteristics}\\
\hspace{0.1in} Market Power		&	-0.003	&	-0.048**	&	0.197***	&	0.290***	&	0.379***	\\
\hspace{0.15in} Medium				&	(0.011)	&	(0.019)	&	(0.051)	&	(0.039)	&	(0.047)	\\
\hspace{0.1in} Market Power		&	-0.016	&	-0.099***	&	0.277***	&	0.479***	&	0.626***	\\
\hspace{0.15in} High					&	(0.014)	&	(0.030)	&	(0.073)	&	(0.047)	&	(0.059)	\\
\hspace{0.1in} Large Market		&	-0.061**	&	-0.088***	&	-0.051	&	0.070***	&	0.206***	\\
													&	(0.031)	&	(0.023)	&	(0.046)	&	(0.018)	&	(0.038)	\\
\hspace{0.1in}Any Teaching	&	-0.016	&	-0.066**	&	-0.039	&	-0.019	&	-0.006	\\
													&	(0.012)	&	(0.028)	&	(0.038)	&	(0.014)	&	(0.020)	\\
\hspace{0.1in}Major Teaching   	&	0.006	&	-0.004	&	-0.001	&	0.000	&	0.002	\\
													&	(0.006)	&	(0.010)	&	(0.025)	&	(0.009)	&	(0.011)	\\
\hspace{0.1in}System					&	0.032*	&	-0.004	&	-0.057	&	-0.051***	&	-0.064***	\\
													&	(0.017)	&	(0.018)	&	(0.037)	&	(0.016)	&	(0.017)	\\
\hspace{0.1in}Non-profit				&	0.020	&	-0.001	&	0.101*	&	0.032	&	0.015	\\
													&	(0.024)	& (0.026)	&	(0.054)	&	(0.026)	&	(0.029)	\\
\hline
\multicolumn{6}{l}{County Age Share}\\
\hspace{0.1in}[18,34]				&	-1.085*	&	1.133	&	3.087	&	-2.656***	&	-0.737	\\
													&	(0.635)	&	(0.727)	&	(2.348)	&	(0.874)	&	(0.969)	\\
\hspace{0.1in}[35,64]				&	-0.050	&	1.572	&	2.503	&	-3.450***	&	0.017	\\
													&	(0.864)	&	(1.016)	&	(2.744)	&	(1.100)	&	(1.228)	\\
\hspace{0.1in} $>$64	&	-0.658	&	0.261	&	-1.715	&	-0.266	&	-1.400	\\
													& (0.771)	&	(1.091)	&	(2.441)	&	(1.047)	&	(1.184)	\\
\hline
\multicolumn{6}{l}{County Share in Race Group}\\
\hspace{0.1in} Share White			&	-0.381**	&	0.092	&	-1.006	&	-0.115	&	0.497	\\
													&	(0.187)	&	(0.270)	&	(0.673)	&	(0.233)	&	(0.314)	\\
\hspace{0.1in} Share Black			&	-0.192	&	-0.073	&	-0.732	&	0.296	&	0.599	\\
													&	(0.208)	&	(0.392)	&	(0.855)	&	(0.302)	&	(0.658)	\\
\hline
\multicolumn{6}{l}{County Share in Income Group}\\
\hspace{0.1in}	50k-75k 				&	-0.398	&	-1.540**	&	1.008	&	-0.630	&	0.212	\\
													&	(0.361)	&	(0.606)	&	(1.354)	&	(0.514)	&	(0.729)	\\
\hspace{0.1in} 75k-100k			&	-0.542	&	0.918	&	-0.132	&	-0.499	&	-0.187	\\
													&	(0.445)	&	(0.721)	&	(1.657)	&	(0.560)	&	(0.725)	\\
\hspace{0.1in} 100k-150k			&	-0.935**	&	-0.334	&	-0.805	&	-0.154	&	-0.111	\\
													&	(0.420)	&	(0.630)	&	(1.455)	&	(0.552)	&	(0.708)	\\
\hspace{0.1in}$>$150k				&	0.846**	&	1.242**	&	1.880	&	1.307***	&	-1.556**	\\
													&	(0.372)	&	(0.562)	&	(1.291)	&	(0.463)	&	(0.617)	\\
\hline
\end{tabular}
}
\setlength{\captionmargin}{.5 \textwidth} \addtolength{\captionmargin}{-.5\wd\gfxbox}
\begin{table}[!h]
\centering
\caption{Baseline Results}
\label{tab:baselineresults}
\usebox{\gfxbox}
\par
\begin{minipage}{\wd\gfxbox}
\footnotesize
Notes: $n=9,425$.  All regressions include hospital and year fixed effects and other hospital level controls include bed count and labor force.  Market power variables are constructed as the overall county market share tercile.  Large market is a binary variable for a hospital in the top half of the market size distribution.  In cases in which independent variables are missing, we recode them and control for missing variable indicators to ensure a balanced panel.  Standard errors are clustered at the hospital level.  *** p-value$<$0.01, ** p-value$<$0.05, * p-value$<$0.1.
\end{minipage}
\end{table}


\newpage
\savebox{\gfxbox}{
\footnotesize
\begin{tabular}{llllll}
\hline	
\hline
 			& Log Mean 				& Log Charge-				& Medicaid 	   	& Medicare   		& Private  			\\
			& Payment		& based Payment			& Discharges      		& Discharges       	& Discharges    \\
	\hline
\multicolumn{6}{c}{Penalty Specific Trends} 		\\	
	\hline										
Penalty 	&	0.011**	&	0.017**	&	-0.030	&	-0.022***	&	-0.006	\\
				&	(0.005)	&	(0.008)	&	(0.022)	&	(0.006)&	(0.011)	\\
	\hline
\multicolumn{6}{c}{Hospital, Year, and County Fixed Effects} 			\\		
	\hline								
Penalty &	0.017***	&	0.009	&	-0.040*	&	-0.021***	&	0.000	\\
&	(0.005)	&	(0.008)	&	(0.021)	&	(0.007)	&	(0.010)	\\
	\hline
\multicolumn{6}{c}{Controlling for Medicaid Expansion States} 			\\			
	\hline							
Penalty &	0.016***	&	0.010	&	-0.037*	&	-0.022***	&	-0.001	\\
&	(0.005)	&	(0.008)	&	(0.020)	&	(0.006)	&	(0.010)	\\
	\hline
\multicolumn{6}{c}{Controlling for Overall HCAHPS Hospital Rating} 		\\
	\hline									
Penalty &	0.016***	&	0.009	&	-0.036*	&	-0.022***	&	-0.001	\\
&	(0.005)	&	(0.008)	&	(0.020)	&	(0.006)	&	(0.010)	\\
	\hline
\multicolumn{6}{c}{Dropping Fiscal 2012} 		\\		
	\hline									
Penalty &	0.015***	&	0.010	&	-0.036*	&	-0.022***	&	-0.003	\\
&	(0.005)	&	(0.008)	&	(0.022)	&	(0.007)	&	(0.011)	\\
	\hline	
\multicolumn{6}{c}{Year Fixed Effects Only} 			\\		
	\hline								
Penalty	&	-0.068***	&	-0.057***	&	0.229***	&	0.098***	&	0.070***	\\
	&	(0.014)	&	(0.017)	&	(0.042)	&	(0.024)	&	(0.021)	\\
				\hline
\end{tabular}
}
\setlength{\captionmargin}{.5 \textwidth} \addtolength{\captionmargin}{-.5\wd\gfxbox}
\begin{table}[!h]
\centering
\caption{Robustness Checks}
\label{tab:robustness}
\usebox{\gfxbox}
\par
\begin{minipage}{\wd\gfxbox}
\footnotesize
Notes: Further controls include those in our baseline specification for mean payments.  In cases in which independent variables are missing, we recode them and control for missing variable indicators to ensure a balanced panel.  Standard errors are clustered at the hospital level.  *** p-value$<$0.01, ** p-value$<$0.05, * p-value$<$0.1.
\end{minipage}
\end{table}

\newpage
\savebox{\gfxbox}{
\begin{tabular}{llllll}
\hline	
\hline
 			& Log Mean				& Log Charge-				& Medicaid 	   	& Medicare   		& Private  			\\
			& Payment		& based Payment			& Discharges      		& Discharges       	& Discharges    \\
	\hline
\multicolumn{6}{c}{Non-profit Hospitals}\\
\hline
Net Penalty & 0.016***	&	0.008	&	-0.035	&	-0.026***	&	-0.009	\\
					& (0.005)	&	(0.008)	&	(0.023)	&	(0.007)	&	(0.011)	\\
\hline
\multicolumn{6}{c}{Non-Profit Hospitals with Penalty Specific Trends} 		\\	
			\hline										
Net Penalty 	&	0.012**	&	0.012	&	-0.029	&	-0.021***	&	-0.013	\\
				&	(0.005)	&	(0.009)	&	(0.025)	&	(0.007)&	(0.012)	\\
\hline
\multicolumn{6}{c}{For-profit Hospitals}\\													
\hline													
Net Penalty & 0.013	&	0.014	&	-0.037	&	-0.005	&	0.035*	\\
				& (0.012)	&	(0.019)	&	(0.048)	&	(0.017)	&	(0.020)	\\
\hline
\multicolumn{6}{c}{For-Profit Hospitals with Penalty Specific Trends} 		\\		
			\hline									
Net Penalty &	0.005	&	0.027	&	-0.028	&	-0.018	&	0.022	\\
&	(0.013)	&	(0.021)	&	(0.048)	&	(0.017)	&	(0.021)	\\
\end{tabular}
}
\setlength{\captionmargin}{.5 \textwidth} \addtolength{\captionmargin}{-.5\wd\gfxbox}
\begin{table}[!h]
\centering
\caption{Results by Profit Status}
\label{tab:byprofit}
\usebox{\gfxbox}
\par
\begin{minipage}{\wd\gfxbox}
\footnotesize
Notes: All regressions include hospital and year fixed effects.  Further controls include those in our baseline specification for mean payments.  In cases in which independent variables are missing, we recode them and control for missing variable indicators to ensure a balanced panel.  Standard errors are clustered at the hospital level.  *** p-value$<$0.01, ** p-value$<$0.05, * p-value$<$0.1.
\end{minipage}
\end{table}



\newpage
\savebox{\gfxbox}{
\begin{tabular}{ccccccc}
\hline	
\hline
 							& Nervous  				& Respiratory  	   	& Circulatory    & Musculoskeletal   		& Labor and & Neonatal \\
							&  System						&  System      	&  System     	&  System        		& Delivery   &	\\
\hline
Net Penalty & 0.022**	&	-0.002	&	0.025***	&	0.001	&	0.000	&	0.021**	\\
& (0.010)	&	(0.010)	&	(0.007)	&	(0.007)	&	(0.005)	&	(0.009)	\\
\hline
n& 1,638	&	2,027	&	3,105	&	3,516	&	5,890	&	3,602.00	\\
Mean &13,734.44	&	11,963.87	&	13,136.28	&	12,970.62	&	11,409.36	&	8,969.000	\\
\hline
\end{tabular}
}
\setlength{\captionmargin}{.5 \textwidth} \addtolength{\captionmargin}{-.5\wd\gfxbox}
\begin{table}[!h]
\centering
\caption{Log Payments for Condition Specific Admissions}
\label{tab:eachcondition}
\usebox{\gfxbox}
\par
\begin{minipage}{\wd\gfxbox}
\footnotesize
Notes: All regressions include hospital and year fixed effects.  The dependent variable is the log of average payments for each condition.  Further controls include those in our baseline specification for mean payments.  The dependent variable in each column is the log of the payment for the associated acute care admission.  In cases in which independent variables are missing, we recode them and control for missing variable indicators to ensure a balanced panel.  Standard errors are clustered at the hospital level.  We restrict the sample to include at least 25 admissions per hospital per year.  *** p-value$<$0.01, ** p-value$<$0.05, * p-value$<$0.1.
\end{minipage}
\end{table}



%\newpage
%\savebox{\gfxbox}{
%\footnotesize
%\begin{tabular}{lllllll}
%\hline	
%\hline
% 			& Log 				& Log				& Medicaid 	   	& Medicare   		& Private  			& Profit  \\
%			& Payment		& Charge			& Discharges      		& Discharges       	& Discharges        	& Index  \\
%\hline
%\multicolumn{7}{c}{Small Markets} \\
%\hline
%Net Penalty	&	0.003	&	-0.007	&	-0.049	&	-0.034*	&	-0.029	&	0.002	\\
%	&	(0.010)	&	(0.015)	&	(0.043)	&	(0.018)	&	(0.022)	&	(0.003)	\\
%\hspace{0.1in} *Med. Mkt. Share 	&	-0.004	&	0.030*	&	0.031	&	0.008	&	0.035	&	-0.003	\\
%	&	(0.012)	&	(0.018)	&	(0.046)	&	(0.020)	&	(0.024)	&	(0.004)	\\
%\hspace{0.1in} *High Mkt. Share 	&	0.009	&	-0.004	&	-0.002	&	0.014	&	0.030	&	0.000	\\
%	&	(0.012)	&	(0.019	)&	(0.051)	&	(0.022)	&	(0.024)	&	(0.004)	\\	
%Market Share	&	0.009	&	-0.056**	&	0.167**	&	0.216***	&	0.321***	&	0.004	\\
%\hspace{0.1in} Medium	&	(0.011)	&	(0.026)	&	(0.080)	&	(0.038)	&	(0.049)	&	(0.004)	\\
%Market Share	&	0.012	&	-0.047	&	0.294***	&	0.288***	&	0.428***	&	0.004	\\
%\hspace{0.1in} High	&	(0.017)	&	(0.035)	&	(0.102)	&	(0.054)	&	(0.063)	&	(0.006)	\\
%\hline
%\multicolumn{7}{c}{Large Markets} \\
%\hline
%Net Penalty	&	0.036***	&	-0.001	&	-0.072*	&	-0.022	&	0.010	&	0.000	\\
%	&	(0.012)	&	(0.018)	&	(0.040)	&	(0.018)	&	(0.025)	&	(0.003)	\\
%\hspace{0.1in} *Med. Mkt. Share 	&	-0.013	&	0.018	&	0.046	&	0.018	&	0.019	&	0.006*	\\
%	&	(0.013)	&	(0.018)	&	(0.038)	&	(0.018)	&	(0.022)	&	(0.003)	\\
%\hspace{0.1in} *High Mkt. Share	&	-0.022*	&	0.038*	&	0.051	&	0.024	&	-0.003	&	0.002	\\
%	&	(0.013)	&	(0.020)	&	(0.039)	&	(0.019)	&	(0.025)	&	(0.003)	\\	
%Market Share	&	-0.001	&	-0.049**	&	0.364***	&	0.360***	&	0.395***	&	-0.003	\\
%\hspace{0.1in} Medium	&	(0.016)	&	(0.021)	&	(0.057)	&	(0.047)	&	(0.060)	&	(0.005)	\\
%Market Share	&	0.009	&	-0.138***	&	0.515***	&	0.513***	&	0.688***	&	-0.005	\\
%High 	&	(0.019)	&	(0.032)	&	(0.082)	&	(0.065)	&	(0.097)	&	(0.007)	\\
%
%\hline
%\end{tabular}
%}
%\setlength{\captionmargin}{.5 \textwidth} \addtolength{\captionmargin}{-.5\wd\gfxbox}
%\begin{table}[!h]
%\centering
%\caption{Triple Differences by Market Share}
%\label{tab:bymktshare}
%\usebox{\gfxbox}
%\par
%\begin{minipage}{\wd\gfxbox}
%\footnotesize
%Notes: All regressions include hospital and year fixed effects. Further controls include those in our baseline specification for mean payments.  In cases in which independent variables are missing, we recode them and control for missing variable binary variables to ensure a balanced panel.  We split the sample by market size because of the strong negative correlation between market size and market share. Standard errors are clustered at the hospital level.  *** p-value$<$0.01, ** p-value$<$0.05, * p-value$<$0.1.
%\end{minipage}
%\end{table}




\newpage
\savebox{\gfxbox}{
\begin{tabular}{lll}
\hline	
 		& Log (Payment) & Log (Charge)  				 \\
\hline
Penalty &	0.039***	&	0.045***	\\
&	(0.010)	&	(0.012)	\\
\hspace{0.1in}* Public Share 2 &	-0.016	&	-0.016	\\
&	(0.011)	&	(0.013)	\\
\hspace{0.1in}* Public Share 3&	-0.030**	&	-0.049***	\\
&	(0.012)	&	(0.014)	\\
\hspace{0.1in}* Public Share 4&	-0.048***	&	-0.070***	\\
&	(0.012)	&	(0.016)	\\
Public Share 2 &	0.009	&	0.046***	\\
&	(0.009)	&	(0.013)	\\
Public Share 3 &	0.019*	&	0.085***	\\
&	(0.011)	&	(0.015)	\\
Public Share 4 &	0.026**	&	0.146***	\\
&	(0.012)	&	(0.018)	\\
\hline
\end{tabular}
}
\setlength{\captionmargin}{.5 \textwidth} \addtolength{\captionmargin}{-.5\wd\gfxbox}
\begin{table}[!h]
\centering
\caption{Triple Differences by Public Share}
\label{tab:publicshare}
\usebox{\gfxbox}
\par
\begin{minipage}{\wd\gfxbox}
\footnotesize
Notes: All regressions include hospital and year fixed effects.  Further controls include those in our baseline specification for mean payments.  The share of a hospital's patients insured by the public sector is broked into quartiles and interacted with penalty variables.  In cases in which independent variables are missing, we recode them and control for missing variable indicators to ensure a balanced panel.  Standard errors are clustered at the hospital level.  We restrict the sample to include at least 25 admissions per hospital per year.  *** p-value$<$0.01, ** p-value$<$0.05, * p-value$<$0.1.
\end{minipage}
\end{table}





\end{document}
