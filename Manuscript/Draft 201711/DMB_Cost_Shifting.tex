\documentclass[12pt]{article}
\usepackage{graphicx,amssymb,amsmath,setspace,comment,verbatim,titling,pgf,lscape}
\usepackage[left=2cm,right=2cm,top=2.5cm,bottom=2cm]{geometry}
\usepackage[round]{natbib}
\usepackage{hyperref}
\usepackage{array}
\usepackage{bbm}
\usepackage{csquotes}


\usepackage[justification=centering]{caption}
\newcommand{\deriv}[2]{\frac{\mathrm{d}#1}{\mathrm{d}#2}}
%%\usepackage{breqn}
\newcommand{\pderiv}[2]{\frac{\partial#1}{\partial#2}}
%\usepackage{siunitx}
\newcolumntype{P}[1]{>{\raggedright\arraybackslash}p{#1}}
\hypersetup{colorlinks,%
						citecolor=black,%
						filecolor=black,%
						linkcolor=black,%
						urlcolor=blue,%
						}


\setlength{\droptitle}{-50pt}

\begin{document}

\title{Do Hospitals Cost-Shift? New Evidence from the Affordable Care Act}
\author{%
  Michael Darden, Ian McCarthy, and Eric Barrette\thanks{Darden: George Washington University. McCarthy: Emory University and NBER. Barrette: Health Care Cost Institute.  Correspondence: \href{mailto:darden@gwu.edu}{darden@gwu.edu}; \href{mailto:ian.mccarthy@emory.edu}{ian.mccarthy@emory.edu}; \href{mailto:ebarrette@healthcostinstitute.org}{ebarrette@healthcostinstitute.org}.}
}
\date{November 2017}

\maketitle

\begin{abstract}
A longstanding debate in health economics and health policy concerns the extent to which hospitals increase prices to private insurance patients following reductions in public funding.  This ``cost-shifting" hypothesis is ubiquitous in health policy debates, but evidence of significant cost-shifting is mixed, and the rationale for such behavior is unclear.  We enter this debate by examining plausibly exogenous variation in Medicare reimbursement rates generated by two policies under the Affordable Care Act: the Hospital Readmission Reduction Program (HRRP) and the Hospital Value Based Payment (HVBP) program.  We merge rich hospital level information on reimbursements to actual private-payer payment  data from a large, multi-payer database.  Our data include roughly 50$\%$ of inpatient prospective payment hospitals and 28$\%$ of all acute care admissions in the United States from 2010 to 2014.  We find that hospitals that faced reimbursement reductions from both programs were able to negotiate 2.3$\%$ higher average private payer prices - approximately $\$$274 extra for the average acute care claim.  This result is completely driven by non-profit hospitals.  We find the largest increases in payments for neonatal care (8.8$\%$) and nervous system claims (5.4$\%$).  Furthermore, we find significant heterogeneity by market share and payer mix.  Taken together, our results provide support for the cost-shifting hypothesis and are consistent with a model in which hospitals maximize utility over profits and quantity while bargaining over prices with private insurance firms.
\end{abstract}
\noindent \textit{JEL Classification:} I11; I18; L2 \\\\
\noindent \textit{Keywords:} Cost-Shifting; Hospital Behavior; Affordable Care Act.\\\\
\setstretch{1.3}

\newpage
\section{Introduction}
A longstanding debate in health economics and health policy concerns hospital ``cost-shifting'' - the ability of a hospital to pass public patient reimbursement cuts to privately insured patients by negotiating for higher prices from private insurance companies.  Insurance companies have long maintained that Medicare in particular is not paying its fair share \citep{frakt2011}.  Furthermore, the assumption of cost-shifting as common hospital practice is ubiquitous in healthcare policy circles.  For example, in the debate over the Affordable Care Act, President Obama said:
\begin{quote}
\textit{``You and I are both paying 900 bucks on average - our families - in higher premiums because of uncompensated care.''\footnote{For additional examples, see the many excerpts in \cite{dranove2017}, including additional statements from President Obama and the U.S. Supreme Court regarding the Affordable Care Act.}- Barack Obama}
\end{quote}
And there is some evidence to support the notion of cost-shifiting. Studying California hospitals from 1993-2001, \cite{zwanziger2006} estimate large effects on private prices due to reductions in Medicare and Medicaid prices, mirroring the findings of \cite{lee2003} and \cite{zwanziger2000}. \cite{zwanziger2006} estimate that cost-shifting can explain 12.3\% of the total increase in private payers' prices from 1997 to 2001.  Yet, in a systematic review of the literature, \citet{frakt2011} concludes that cost-shifting, if it exists, is not widespread and is not a main driver of increased healthcare costs.   While a simple model of monopoly pricing with price discrimination would suggest different prices for public and private patients \citep{hay1983}, that model would also suggest a \textit{decrease} in the private price for a decrease in public reimbursements.  Furthermore, for both for-profit and non-profit firms, it is unclear why a hospital with the power to raise prices would not have already done so.\footnote{Several papers find zero or even negative price effects given public reimbursement cuts. \cite{white2013} examines market level pricing from the Truven MarketScan data and finds that private prices decrease following reductions in Medicare reimbursement rates. Exploiting a change in Medicaid payment policies in California, \citet{dranove1998} similarly found little evidence of cost-shifting. More recently, using the 2008 stock market collapse as an exogenous change to hospital endowments, \cite{dranove2017} find that the average hospital does not appear to cost-shift, with some evidence of cost-shifting among hospitals with sufficient market share.}

We enter the cost-shifting debate by exploiting the 2013 adoption of the hospital readmission reduction program (HRRP) and hospital value based purchasing program (HVBP) as plausibly exogenous reductions in Medicare reimbursement rates. We use data from a large, multi-player commercial health insurance claims database to capture changes in privately negotiated prices, which implies that we avoid having to proxy for private payments with charges, costs, or other measures of hospital revenues. We merge our price data to readmission figures from the Centers for Medicare and Medicaid Services (CMS) Hospital Compare website, hospitals characteristics from the American Hospital Association (AHA) annual surveys, market demographics from the American Community Survey (ACS), and hospital financial data from the Healthcare Cost Report Information System (HCRIS). Our final hospital pricing data constitute a balanced panel of roughly 28$\%$ of all acute care claims from 50$\%$ of all inpatient prospective payment hospitals in the United States between 2010 and 2014.

The HRRP and HVBP are components of the Affordable Care Act's (ACA) cost containment strategy.  The HRRP penalizes hospitals for which 30-day readmissions for certain conditions exceed risk-adjusted thresholds constructed as a function of national averages.  Starting in Fiscal Year (FY) 2013 (October 2012-September 2013), hospitals faced a maximum cut in Medicare reimbursements of 1$\%$ across all Diagnosis Related Groups (DRG); the maximum potential penalty increased to 3$\%$ by 2015.  The \cite{cbo2010} estimates that HRRP will reduce hospital reimbursements from Medicare by $\$$113 billion through 2019.\footnote{\citet{mellor2016} finds that HRRP has been associated with declines in AMI 30-day readmission rates, and \citet{gupta2016} finds a 5$\%$ reduction in overall readmissions and 3$\%$ reduction in all-cause mortality, mostly driven by quality improvements.}  By contrast, the HVBP program reduces reimbursements to all hospitals by 2$\%$, but rewards hospitals with incentive payments for their quality of care over a variety of quality domains.  Hospitals have the opportunity to receive rebates if rewards exceed the initial 2$\%$ reduction.\footnote{\citet{norton2016} provide evidence that quality of care improved as a result of HVBP, but only for services with the highest marginal incentives to improve quality of care.  Both HRRP and HVBP generated significant and plausibly exogenous cross-sectional and temporal variation in Medicare reimbursements.}

Our baseline empirical specification is a hospital fixed effects estimator in which we estimate the difference in average payments between those hospitals with a net penalty under the HRRP and HVBP relative to those not penalized before and after 2013.  Our preliminary results reveal that hospitals penalized from these policies increased prices by 2.3$\%$, equivalent to a roughly $\$$247 increase at the mean, in the period following policy enactment.  Despite evidence of an increase in prices, we find that penalized hospitals decreased the share of public patients by 1$\%$ and increased the share of private patients by 1$\%$.  Models of prices for specific procedures suggest marked increases in prices for neonatal (8.8$\%$) and nervous system claims (5.4$\%$).

Because theoretical predictions suggest that purely profit-motivated firms are unlikely to cost-shift, economic rationales for cost-shifting have focused on a hospital's objective function.  \cite{dranove1988} models a hospital as a utility maximizer, where utility is defined over both profit and quantity. The model in \cite{dranove1988} suggests that hospitals may cost-shift if: a) utility is derived from the quantity of services for privately insured patients; and b) the hospital has some market power. Indeed, as the level of competition in a market increases, the ability to raise prices falls, so market power is a necessary condition for cost-shifting. Hospitals may directly value quantity of care for reasons of altruism or prestige, or simply because nonprofit hospitals (which represent 66\% of hospital beds in the country) must provide some form of ``community benefit'' to maintain its tax-exempt status.\footnote{The Congressional Budget Office defines community benefits as services geared toward ``promoting the health of any broad class of persons'' \citep{cbo2006}.} Thus, if cost-shifting exists, evidence is anticipated to be isolated primarily among non-profit hospitals. Consistent with this prediction, when we break our analysis by profit status, we find that our baseline result of a 2.3$\%$ increase in private payments is completely driven by non-profit hospitals.

Complicating matters, hospital prices are not merely set by hospitals, but rather they are negotiated with private insurance firms. In the context of such a negotiation, the mechanism for cost-shifting in response to \textit{penalties} incurred from lower-than-expected quality is particularly unclear. Essentially, how can a penalty for low quality allow the hospital to negotiate a higher price? We argue that three potential mechanisms may allow for such an outcome: 1) if the quality information revealed by the penalty is not new information to the market, then the penalty is simply a reduction in the public reimbursement rate and the underlying source of the penalty is irrelevant; 2) even if the information is new, hospitals may exploit the penalty in other service areas where they have a comparative quality advantage in the market; or 3) hospitals with very high shares of public patients may be particularly burdened by the HRRP and HVBP penalties, such that the hospital relies on higher private payments to operate. In this latter case, private insurers may allow higher prices so as to maintain leverage in negotiations with other hospitals in the local market. We address each of these mechanisms with a series of alternative specifications and hospital samples, where we generally find evidence of all three mechanisms at play.

%We find that our main effect of interest does not vary by relative market share, but, consistent with \cite{wu2010}, we find that hospitals with a smaller share of Medicare patients engage in more cost-shifting.

In addition to our careful examination of the different mechanisms underlying potential cost-shifting, our paper contributes to the cost-shifting debate by using a precise measure of hospital payments for private patients in the modern health care environment. \cite{gowrisankaran2015} and \cite{cooper2015} use similar data in examining hospital prices, but to our knowledge, we are the first to use actual payments to private insurers in a study of cost-shifting.  One challenge to identification in our study is the time frame: the main components of the ACA went into effect in 2014, and we cannot completely rule-out the possibility that differential patient flows and other sources of time-varying unobserved heterogeneity may bias our results.  A competing explanation for our results may be that hospitals penalized under the HRRP and HVBP may have been able to negotiate higher private prices not because of penalties, but because of differential impacts of the full ACA.\footnote{There is evidence that hospitals located in low-SES areas were more likely to be penalized under HRRP, despite risk-adjustment at the patient level.}  Because of the confidential nature of hospital/insurer bargaining, we cannot completely rule out that higher prices for penalized hospitals where due to other, unobserved factors that are correlated with penalties under HRRP and HVBP.  However, this problem is present in any study that lacks information on the bargaining process, and our results are consistent across number specification checks. 

Following a discussion of the setting and a detailed explanation of each policy in Section \ref{sec:Background}, we present our data, empirical methods, and baseline findings in Section \ref{sec:Empirical}. Section \ref{sec:Ext} considers several extensions and testable hypotheses implied by the standard theoretical model of cost-shifting, and Section \ref{sec:Discussion} puts these results in context.  Section \ref{sec:Conclusion} concludes.

\section{Background}
\label{sec:Background}

The debate over whether, and the extent to which, hospitals shift costs has been ongoing for decades. While private insurers are naturally averse to higher private prices, hospitals have emphasized the need to cost-shift in an attempt to lobby for larger public reimbursement. For example:
\begin{quote}
\textit{``Cost shifts have been a fact of hospital financial survival for decades.... The data show ...  how private payment is a mirror image of public payment over time and that the cost shift occurs. Hospitals must make up for shortfalls through a combination of approaches and cost-shifting is among them.'' -Rich Umbdenstock, Former President and CEO of American Hospital Association}\footnote{\href{$http://blog.aha.org/post/costshifting-in-hospitals-$}{American Hospital Association Stat}}
\end{quote}

In the end, the argument that cost-shifting occurs is easily motivated by observed trends.  In 2015, 55 million Americans were enrolled in Medicare, up from 37.5 million in 1995, and from 1980 to 2014, the share of hospital costs attributable to Medicare rose from 34.6$\%$ to 40.2$\%$.  Meanwhile, in 2014, hospitals endured a shortfall of $\$$35 billion of Medicare payments relative to Medicare patient costs, as compared with a $\$$5 billion surplus relative to costs in 1997.  During this same period, patients insured by private payers became increasingly lucrative: in 2014, the payment-to-cost ratio of privately insured patients was roughly 140$\%$.\footnote{All statistics from the American Hospital Association Trendwatch Chartbook, 2016} Consistent with the trend in profitability of private insurance patients to hospitals, average premiums for covered workers with family coverage increased by 69$\%$ from 2000 to 2005.\footnote{Kaiser Family Foundation}

While the current conditions for cost-shifting appear to be ripe, much of the evidence of significant cost-shifting comes from the 1980s and 1990s.  For example, \cite{cutler1998costshift} studies cost-shifting during the phase-in of Medicare prospective payments during the 1980s, which resulted in an average 2$\%$ per year reduction in Medicare reimbursements.  He found evidence of dollar-for-dollar cost-shifting.  More recently, \cite{zwanziger2006} study the late 1990s and found that, between 1997 and 2001, cost-shifting was responsible for roughly 12$\%$ of the observed increase in total private payer prices.

In contrast, the simplest argument against cost-shifting as a significant mechanism in the hospital market is one of basic microeconomics.  A for-profit firm with market power who sells to two groups should not respond to an exogenous decline in the price to one group by raising prices to the second group.  \cite{hay1983} shows that, even when the government commits to reimbursing the full average cost of Medicare patients, hospitals will: a) still charge a higher price to privately insured patients; and b) respond to lower Medicare reimbursements with \textit{lower} private prices.\footnote{\cite{dor1996} demonstrate that payer-specific marginal costs may be evidence of differential treatment by hospitals.}

In spite of the evidence presented by \cite{cutler1998costshift} and \cite{zwanziger2006}, the empirical evidence of cost-shifting is notably weak.  In a 2011 review of this literature, Austin Frakt states:
\begin{quote}
\textit{``In fact, as a whole, the
evidence does not support the notion that cost-shifting is both large and
pervasive. Instead, it reveals that cost-shifting can occur but may not
always do so. When it has occurred, it has generally been measured at a
rate far below dollar-for-dollar''}.
\end{quote}

Indeed, numerous studies have found zero or even lower overall price effects, including \cite{dranove2008impact}, \cite{wu2010}, \cite{frakt2014}, and \cite{dranove2017}, but potentially important positive price effects for certain subgroups.  For example, \cite{wu2010} shows that hospitals with large shares of private patients (relative to Medicare patients) were able to cost-shift following the 1997 Balanced Budget Act, perhaps due to greater bargaining power.

We argue that identification of cost-shifting behavior is inherently difficult for three reasons.  First, the hospital market is incredibly complex.  In addition to many different types of payers, the industry is heaviliy regulated, and policy changes occur frequently.  We study two exogenous sources of reimbursement variation in which complexity is arguably a benefit to identification -- as discussed below, the initial implementation of the hospital readmission reduction program and the hospital value based purchasing program were complex, and a common complaint from hospitals is the opaque nature of the reimbursement reduction calculation.  Second, measurement error in private payments may be severe.  Because private payments are typically not observed, many of the referenced papers above must proxy for private payments, often with charges or costs.  In our paper, as discussed below, we observe private payments -- the actual dollar amount of reimbursement from three large private insurers to hospitals for 28$\%$ of the privately insured population.  Finally, heterogeneity in hospital responses to public reimbursement cuts may muddle instances of important cost-shifting. We attempt to examine this heterogeneity with a series of alternative specifications and supplemental analyses.

\subsection{HRRP and HVBP}
The adoption of the Medicare prospective payment system (PPS) in 1998, in which hospital reimbursement changed from pure fee-for-service to a fixed amount for each inpatient stay depending on diagnosis, generated incentives for hospitals to cut ``excessive'' procedures. The PPS also created incentives for hospitals to discharge patients quickly.  By 2011, Medicare paid $\$$24 billion per year for 1.8 million hospital \textit{readmissions} - admissions to any hospital within 30-days of discharge for the same condition.  While some readmissions are unavoidable, the HRRP was a cost containment in the ACA designed to create hospital reimbursement penalties for ``excessive'' readmissions.  Starting in 2013, hospitals with risk-adjusted readmissions in AMI, HF, and PN that exceeded national comparison averages saw overall Medicare reimbursement cuts of up to 1$\%$.   In 2015, the maximum penalty increased to 3$\%$, total penalties rose to $\$$420m (Rau, 2015), and applicable conditions also included chronic obstructive pulmonary disease and total hip and knee replacements.  Evidence has been suggestive that HRRP has reduced readmissions in the tested conditions.  For example, Mellor \textit{et al.} (2016) found that HRRP was associated with declines in AMI readmission, which were not due to delay of treatment, changes in intensity, or selective patient mix. Carey and Lin (2015) used discharge data from New York hospitals in 2008 and 2012 to examine changes in readmissions and several unintended consequences. And Gupta (2016) find a 5$\%$ reduction in overall readmissions and 3$\%$ reduction in all-cause mortality, which was mostly driven by quality improvement.

In contrast, the Hospital Value-Based Purchasing (HVBP) program is rooted in a standard principal-agent model in which the principal (CMS in this case) contracts with agents (hospitals) to provide quality care to Medicare enrollees. The HVBP program scores hospitals based on their achievement (comparison to other hospitals) as well as their improvement (comparison to their own previous performance).  Similar to the HRRP, the HVBP bases changes in reimbursement on past quality.  However, unlike the HRRP, the HVBP program is funded by reducing all hospitals' base operating Medicare severity diagnosis-related group (MS-DRG) payments by 2$\%$ and creating rebate incentives depending on defined quality metrics.  The program defines several quality domains and converts measures of quality within each domain to points, which are aggregated and mapped to a total point score.  The total point score determines the size of the reimbursement rebate.  Thus, the HVBP program may result in a net penalty or reward.


\section{Empirical Methods}
\label{sec:Empirical}


\subsection{Data}
An important contribution of our paper is the use of data on actual payments from private insurance firms to hospitals.  Our data, maintained by the Healthcare Cost Institute, contain all claims made to three national commercial insurers.  The data include claims made in all 50 states plus Washington D.C., and they cover approximately 28$\%$ of individuals under the age of 65 who have employer-sponsored insurance (ESI).  These unique data include payments for every claim made for acute care hospital admissions, which capture the negotiated prices between hospitals and insurers.  These payments differ substantially from charge-based estimates of prices often used in the literature \citep{dafny2009,dranove2017}. Indeed, in our data, the correlation between negotiated prices and estimated prices from HCRIS is 0.435, suggesting that using charge-based estimates of price as the dependent variable may contain significant measurement error.  To our payment data, we merge HRRP and HVBP penalty data from the Healthcare Cost Report Information System (HCRIS) and Hospital Compare; local area characteristics from the American Community Survey (ACS); hospital level characteristics such as bed count, for-profit status, and system membership from the American Hospital Association (AHA) annual surveys; and data on a hospital's payer mix (i.e., the number and share of Medicare, Medicaid, or private insurance patients) also from HCRIS.

Table 1 presents our main dependent variables over time. While average risk-adjusted payments received by hospitals increase over 20$\%$ between 2010 to 2014, shares of public, Medicare, Medicaid, and private patients remain relatively stable over time.  Importantly, while shares remain stable, within-hospital patient mix may vary considerably over time as a function of public reimbursements, which is why we treat payer share as a dependent variable.

Because our baseline empirical specification below depends on within hospital variation, we split our sample by whether a hospital was ever observed to be penalized by the HRRP and HVBP during our sample period.  Table 2 presents summary statistics of our dependent and some independent variables by ever penalized status.  Payments to ever penalized hospitals are marginally higher than to those penalized at some point, but this difference is not statistically significant.  Nonprofit hospitals constituted a much larger share of our never penalized hospitals, suggesting that nonprofit hospitals may be of higher quality, at least in terms of HRRP and HVBP.

Our baseline empirical specification isolates within hospital variation in private payments over time by whether a hospital faced penalties from the HRRP or HVBP. This analysis therefore focuses on the extensive margin; however, we consider the intensive margin based on the dollar-value of the penalties in subsequent analyses. Our baseline empirical model is given by
\begin{equation}
\label{eq: reg}
y_{hct} = \alpha_{h} + x^{'}_{hct}\beta + \delta \times 1[Penalty] + \tau_{c} + \theta_{t}  +  \epsilon_{hct},
\end{equation}
where outcome $y_{hct}$ at hospital $h$ in county $c$ in fiscal year $t$ is a function of a hospital specific intercept, $\alpha_{h}$; a vector of time-varying exogenous characteristics, $x_{hct}$; mutually exclusive indicators for penalty under one and both of the policies; a county-level fixed effect, $\tau_c$; controls for year effects, $\theta_t$; and an i.i.d. error term $\epsilon_{hct}$.  Because indicators for policy are zero for all hospitals between 2010-2012, Equation 1 represents a difference-in-differences estimator, where our parameter of interest, $\delta$, captures the extent to which hopsitals penalized under the HRRP and HVBP policies receive differential private payments relative to hospitals with no penalty.

Table 3 presents results from Equation \ref{eq: reg} for the log of payments and for a variety of payer mix variables.  Results suggest some evidence of cost-shifting, as well as movement \textit{away} from public patients and towards private patients.  Indeed, hospitals penalized by both policies received risk-adjusted payments that were 2.3$\%$ higher than hospitals penalized by neither policy.  The share of public patients declined by one percentage point and the share of private patients increased by one percentage point at these hospitals, driven primarily by shifting way from Medicare.  Both of these effects were significant at the 99$\%$ level of confidence.

Results in Table 3 reflect the causal effect of the HRRP and HVBP on payments if there are no unobserved, time-varying factors that influence payments and are also correlated with penalty status.  While we cannot completely rule out this possibility, we estimate  a variant of Equation \ref{eq: reg} in which we allow the trend in prices to vary by whether a hospital is ever penalized.  Differential trends in prices conditional on penalty status and other controls would be suggestive of time-varying unobserved heterogeneity.  These results are yet to be finished.  We also intend to allow trends to vary by county, to account for local are level unobserved factors that vary over time.





\section{Empirical Extensions}
\label{sec:Ext}

As initially examined in \cite{dranove1988}, a hospital may pursue a cost-shifting strategy if the hospital's objective function includes something other than pure profit (e.g., if the hospital receives direct utility from the quantity of services provided). For this reason, cost-shifting is thought to more likely occur among nonprofit hospitals, if at all. Indeed, to maintain their tax exempt status, nonprofit hospitals are required by the IRS to provide community benefits.\footnote{Of course, this does not mean that nonprofit hospitals are fully altruistic. In fact, evidence on nonprofit hospital behavior relative to for-profit hospital behavior is mixed. For example, \cite{silverman2004} and \cite{dafny2005} find evidence that nonprofits ``upcode'' less frequently, while \cite{gaynor2003} find that nonprofit hospitals have lower marginal costs but higher markups than for-profit hospitals.} Since over 80$\%$ of hospitals in our sample are nonprofit, this implies that the objective function of the majority of hospitals in our analysis extends beyond pure profit-maximization.

Importantly, the model posited in \cite{dranove1988} assumes that hospitals set price unilaterally, and it is not immediately obvious whether this prediction extends to a modern managed care market in which hospitals and private insurers negotiate over private prices. In Appendix \ref{app:cs_bargaining}, we offer a minor extension to the model of \cite{dranove1988} and show that much of the same intuition applies in a bargaining context as well. One interesting difference is that the hospital may still cost-shift even if the objective function is based solely on profit - what is required is that there exists diminishing marginal utility of profits, a mechanism that we would expect to be strongest among nonprofit hospitals.

To investigate the extent to which cost-shifiting varies by profit status, we re-estimate Equation \ref{eq: reg} separately for nonprofit and for-profit hospitals.  The results are presented in Table 5.  Consistent with the literature, our results suggest both an economically and statistically insignificant effect of the HRRP and HVBP penalties on private payments for for-profit hospitals.  The point estimate for non-profit hospitals is statistically significant and nearly identical to our baseline result (2.5$\%$ vs. 2.3$\%$).

The role of private insurance markets is also an interesting factor as to whether or how much hospitals can cost-shift. As discussed in Appendix \ref{app:cs_bargaining}, we would expect cost-shifting to occur most in areas where hospitals not only have a large market share but also where a hospital's market share is large relative to an insurer's. This is the natural extension of \cite{dranove1988} to a bargaining context; however, this prediction is less clear when we incorporate the insurer's choice of premiums in the insurance market. Essentially, if the insurance market is heavily concentrated, then insurers can pass healthcare price changes to their plan enrollees \cite{ho2016}. In this sense, cost-shifting is likely to occur when insurers have a particularly small market share but perhaps also when insurers have a particularly large market share. The role of insurance markets on the prevalence or magnitude of cost-shifting is therefore empirically difficult to measure without detailed data on insurance premiums and insurer market shares (at a local level).

Perhaps more importantly, the underlying mechanisms for cost-shifting in our context are unclear if we seriously consider the role of hospital and insurer bargaining. In particular, the reduction in public payments that serve to identify the presence of cost-shifting in our analysis derive from a lower-than-expected performance on some set of quality metrics. For cost-shifting to occur, hospitals must somehow translate a signal of low quality into higher prices. This could occur in at least three ways: 1) if the quality information revealed by the penalty is not new information to the market, then the penalty is simply a reduction in the public reimbursement rate and the underlying source of the penalty is irrelevant; 2) even if the information is new, hospitals may exploit the penalty in other service areas where they have a comparative quality advantage in the market; or 3) hospitals with very high shares of public patients may be particularly burdened by the HRRP and HVBP penalties, such that the hospital relies on higher private payments to operate. We examine each of these possibilities in more detail in the following subsections.

\subsection{HRRP and HVBP Penalties \textit{Not} Informative of Quality}
Existing findings from \cite{dranove2008} and others tend to find relatively small effects of quality reporting on hospital choice. As \cite{dranove2008} state, ``report cards do not always convey `news' about quality; in some cases the rankings confirm with prior beliefs about quality.'' To the extent that penalties from the HVBP and HRRP do not reveal any new information to the market, then the penalty acts simply as a reduction in public reimbursements and the intuition from our model in Appendix \ref{app:cs_bargaining} applies. We examine this issue with an alternative specification in which we control for a hospital's overall quality as measured by patients' overall hospital rating from the Hospital Consumer Assessment of Healthcare Providers and Systems (HCAHPS).


\subsection{Differential Effects Across Service Lines}
Overall price effects may be driven by specific services which are not measured by the HRRP and HVBP.


\subsection{Increased ``Exposure'' to Penalties}
In the context of price negotiations for a single insurer-hospital pairing, insurers will generally want to pay lower prices while hospitals want to receive higher prices; however, insurers may not pursue such negotiations independently across hospitals. For example, most hospital markets are relatively concentrated with just a handful of hospitals in operation. The loss of one hospital in the market may therefore increase the remaining hospitals' bargaining power over an existing insurer. As such, insurers may be willing to increase prices for a single hospital if such a price increase prohibits the hospital from otherwise closing or merging with a competitor. In this case, hospitals most exposed to the HRRP and HVBP (i.e., hospitals more dependent on publicly insured patients) may be able to disproportionately increase private insurance prices out of insurers' willingness to maintain more competitors in the hospital market.


\section{Discussion}
\label{sec:Discussion}
\section{Conclusion}
\label{sec:Conclusion}


\newpage
\bibliographystyle{authordate1}
\bibliography{BibTeX_Library}


\clearpage
\newpage
\appendix

\section{Tables}
\label{app:tables}

\newsavebox{\gfxbox}
\savebox{\gfxbox}{
\begin{tabular}{cccccccccc}
\hline \hline
%\multicolumn{9}{c}{}\\
Fiscal & Sample 		&  Payment $\$$								& Public  	   & Medicare   & Medicaid  		& Other & One  & Two \\
Year   &  Size    		&  Mean (St. Dev.) 				& Share      & Share       	& Share        	& Share & Penalty & Penalties\\
 \hline
2010 & 1,635			& 	10,853.77 (4,899.20)		& 0.48   &   0.34   &   0.14    &  0.52  & 0.00 & 0.00 \\
2011 &   1,635 		& 	11,437.98 (4,942.05)		& 0.47   &   0.34    &  0.13    &  0.53  & 0.00 & 0.00   \\
2012 & 	1,635 		& 	12,035.04	(5,409.05)&  0.46  &    0.33   &   0.13   &   0.54  & 0.00 & 0.00   \\
2013 & 	1,635 		& 	12,378.27	(5,711.30)& 0.45   &   0.33    &  0.12   &   0.55  & 0.49 & 0.36   \\
2014 & 		1,635 	&	12,913.11	(5,658.24)& 0.45  &    0.33   &   0.12   &   0.55 & 0.48 & 0.37\\
Total & 	 8,175		& 11,914.66	(5,381.98)	&	 0.46   &   0.33   &   0.13   &   0.54  \\
\hline
\end{tabular}
}
\setlength{\captionmargin}{.5 \textwidth} \addtolength{\captionmargin}{-.5\wd\gfxbox}
\begin{table}[!h]
\centering
\caption{Characterization of Research Sample over Time}
\label{tab:summarystats}
\usebox{\gfxbox}
\par
\begin{minipage}{\wd\gfxbox}
\footnotesize
Notes: Balanced panel of hospitals over time between 2010 and 2014.  Payment represents the mean dollar amount paid to a hospital in a year over all acute care admissions.  Share variables are measured at the hospital level.  One penalty is a binary variable for whether a hospital was penalized under HRRP or HVBP, but not both.  Two penalties is a binary variable for being penalized under both policies.
\end{minipage}
\end{table}



\newpage
\savebox{\gfxbox}{
\begin{tabular}{lccc}
\hline \hline
Variable 	& Never 				& Ever  				&   	  \\
		   		&  Penalized    		& Penalized			&    p-value   				\\
 \hline
Log(Payment)					& 9.245			& 9.284 			& 0.072\\
System  Membership      	&       0.728    	&     0.747   	&     0.453\\
Nonprofit     &       0.967    &     0.700    &     0.000\\
Log(Case Mix Index)        &       0.446      &   0.435      &   0.206\\
\multicolumn{4}{l}{Local Hospital}\\
\hspace{0.05in} Monopoly  &      0.161    &     0.124    &     0.054 \\
\hspace{0.05in} Duopoly    &     0.239    &     0.172   &      0.002\\
\hspace{0.05in} Triopoly    &    0.085    &     0.105     &    0.260\\
\multicolumn{4}{l}{Market Share}\\
\hspace{0.05in} Medicare  &      0.361  &       0.331    &     0.000\\
\hspace{0.05in} Medicaid    &     0.120     &    0.126    &     0.304\\
\hspace{0.05in} Public      &    0.480     &    0.457     &    0.001\\
\hspace{0.05in} Other     &      0.520    &     0.543    &     0.001\\
\hspace{0.05in} Profit Index     &         0.576    &    0.603    &    0.000\\
Total Pop. (1000s)     &     562.147   &  1123.431   &     0.000 \\
\multicolumn{4}{l}{County Age Distribution}\\
\hspace{0.05in}[18, 34]      &    0.244   &     0.239    &    0.054\\
\hspace{0.05in}[35, 64]   &      0.399    &    0.393   &     0.001\\
\hspace{0.05in}>65    &    0.123    &    0.130     &   0.001\\
\multicolumn{4}{l}{County Race Distribution}\\
\hspace{0.05in}White     &     0.654   &     0.737    &    0.000\\
\hspace{0.05in}Black    &     0.245  &      0.135   &     0.000\\
\multicolumn{4}{l}{County Income Distribution}\\
\hspace{0.05in}<$\$$50k   &    0.181    &    0.181      &  0.773\\
\hspace{0.05in}[$\$$50k, 75k]   &    0.126   &     0.124    &    0.010\\
\hspace{0.05in}[$\$$100k, 150k]       & 0.146   &    0.131  &      0.000\\
\hspace{0.05in}>$\$$150k   &    0.117   &     0.099 &       0.000\\
\multicolumn{4}{l}{County Education Distribution}\\
\hspace{0.05in}High School Only   &     0.273    &    0.273     &   0.931\\
\hspace{0.05in}Bachelor's Only      &     0.190   &     0.189      &  0.759\\
\hline
\end{tabular}
}
\setlength{\captionmargin}{.5 \textwidth} \addtolength{\captionmargin}{-.5\wd\gfxbox}
\begin{table}[!h]
\centering
\caption{Hospital Characteristics by Penalties}
\label{tab:bypenalty}
\usebox{\gfxbox}
\par
\begin{minipage}{\wd\gfxbox}
\footnotesize
Notes: Summary statistics are split by whether a hospital is ever observed to be penalized by both policies in either 2013 or 2014, or both. Payment represents the mean dollar amount paid to a hospital in a year over all acute care admissions.   County level characteristics are from the American Community Survey.
\end{minipage}
\end{table}

\newpage
\savebox{\gfxbox}{
\scriptsize
\begin{tabular}{lllllll}
\hline	
\hline
 			& Log 				& Log				& Medicare 	   	& Public   		& Other  			& Profit  \\
			& Payment		& Charge			& Share      		& Share       	& Share        	& Index  \\
\hline
HRRP and HVBP   	&	0.022**	&	-0.004	&	-0.017***	&	-0.021***	&	0.021***	&	0.002	\\
                                                        	&	(0.010)	&	(0.018)	&	(0.003)	&	(0.006)	&	(0.006)	&	(0.003)	\\
HRRP or HVBP            	&	0.008	&	-0.013	&	-0.010***	&	-0.010***	&	0.010***	&	-0.001	\\
                                                        	&	(0.007)	&	(0.012)	&	(0.002)	&	(0.004)	&	(0.004)	&	(0.002)	\\
\multicolumn{7}{l}{Hospital Characteristics}\\                                                  													
\hspace{0.1in}Market Power      	&	0.013	&	-0.041*	&	-0.009	&	-0.005	&	0.005	&	-0.005	\\
\hspace{0.1in} Medium   	&	(0.012)	&	(0.021)	&	(0.006)	&	(0.008)	&	(0.008)	&	(0.006)	\\
\hspace{0.1in}Market Power     	&	0.010	&	-0.120***	&	-0.023**	&	-0.031**	&	0.031**	&	-0.005	\\
\hspace{0.1in} High     	&	(0.016)	&	(0.035)	&	(0.010)	&	(0.013)	&	(0.013)	&	(0.007)	\\
\hspace{0.1in} Large Market	&	-0.074**	&	-0.064**	&	-0.016***	&	-0.021**	&	0.021**	&	0.008**	\\
	&	(0.036)	&	(0.026)	&	(0.005)	&	(0.009)	&	(0.009)	&	(0.004)	\\
\hspace{0.1in}Teaching Type 1   	&	-0.018	&	-0.086*	&	0.004	&	0.003	&	-0.003	&	-0.003	\\
        	&	(0.016)	&	(0.052)	& (0.004)	&	(0.014)	&	(0.014)	&	(0.006)	\\
\hspace{0.1in}Teaching Type 2   	&	0.014*	&	-0.015	&	-0.001	&	0.004	&	-0.004	&	-0.002	\\
        	&	(0.007)	&(0.011)	&	(0.002)	&	(0.004)	&	(0.004)&	(0.002)\\
\hspace{0.1in}System    	&	0.031	&	0.005	&	-0.004	&	-0.004	&	0.004	&	0.006*	\\
        	&	(0.021)	&	(0.023)	&	(0.004)	&	(0.006)	&	(0.006)	&	(0.004)	\\
\hspace{0.1in}Nonprofit 	&	0.036	&	0.024	&	-0.011	&	-0.001	&	0.001	&	-0.001	\\
        	&	(0.059)	&	(0.071)	&	(0.009)	&	(0.011)	&	(0.011)	&	(0.005)	\\
\multicolumn{7}{l}{County Share in Age Group}\\													
\hspace{0.1in}18-34     	&	-2.691**	&	2.746	&	-0.112	&	0.581	&	-0.581	&	0.065	\\
        	&	(1.294)	&	(1.857)	&	(0.304)	&	(0.562)	&	(0.562)	&	(0.347)	\\
\hspace{0.1in}35-64     	&	-0.494	&	2.263	&	-0.749**	&	-0.738	&	0.738	&	-0.324	\\
        	&	(1.395)	&	(2.144)	&	(0.348)	&	(0.598)	&	(0.598)	&	(0.475)	\\
\hspace{0.1in}>64       	&	-1.823	&	0.032	&	0.198	&	-0.135	&	0.135	&	-0.319	\\
        	&	(1.189)	&	(2.061)	&	(0.334)	&	(0.589)&	(0.589)	&	(0.416)	\\
\multicolumn{7}{l}{County Share in Race Group}\\	
\hspace{0.1in}Share White     	&	-0.385*	&	0.048	&	-0.088*	&	-0.315***	&	0.315***	&	0.044	\\
        	&	(0.221)	&	(0.363)	&	(0.051)	&	(0.112)	&	(0.112)	&	(0.062)	\\
\hspace{0.1in}Share Black     	&	-1.057	&	-0.467	&	-0.161	&	0.706**	&	-0.706**	&	0.149	\\
        	&	(0.766)	&	(1.353)&	(0.174)	&	(0.318)	&	(0.318)	&	(0.189)	\\
\multicolumn{7}{l}{County Share in Income Group}\\												
\hspace{0.1in}50k-75k   	&	-0.319	&	-0.839	&	-0.131	&	-0.225	&	0.225	&	-0.256*	\\
        	&	(0.398)	&	(0.701)	&	(0.108)	&	(0.205)	&	(0.205)	&	(0.149)	\\
\hspace{0.1in}75k-100k  	&	-0.788	&	0.742	&	-0.112	&	-0.033	&	0.033	&	-0.395***	\\
        	&	(0.510)	&	(0.810)	&	(0.130)	&	(0.238)	&	(0.238)	&	(0.138)	\\
\hspace{0.1in}100k-150k 	&	-1.037**	&	-0.115	&	0.044	&	-0.159	&	0.159	&	-0.070	\\
        	&	(0.480)	&	(0.719)	&	(0.137)	&	(0.222)	&	(0.222)	&	(0.136)	\\
\hspace{0.1in}>150k     	&	1.357***	&	1.605**	&	0.403***	&	0.246	&	-0.246	&	0.007	\\
        	&	(0.452)	&	(0.756)	&	(0.121)	&	(0.208)	&	(0.208)	&	(0.148)	\\
        \multicolumn{7}{l}{County Share in Education Group}\\	
\hspace{0.1in} High School      	&	0.470	&	-0.073	&	-0.120	&	-0.309	&	0.309	&	0.166	\\
\hspace{0.15in}  Only 	&	(0.366)	&	(0.719)	&	(0.112)	&	(0.208)	&	(0.208)&	(0.123)	\\
\hspace{0.1in}  Bachelor's      	&	0.170	&	-1.346	&	-0.419**	&	-0.340	&	0.340	&	0.206	\\
\hspace{0.15in}  Only            	&	(0.532)	&	(1.053)	&	(0.185)	&	(0.284)	&	(0.284)	&	(0.165)	\\
\hline
\end{tabular}
}
\setlength{\captionmargin}{.5 \textwidth} \addtolength{\captionmargin}{-.5\wd\gfxbox}
\begin{table}[!h]
\centering
\caption{Baseline Results}
\label{tab:baselineresults}
\usebox{\gfxbox}
\par
\begin{minipage}{\wd\gfxbox}
\footnotesize
Notes: $n=8,175$.  All regressions include county and year fixed effects and other hospital level controls include bed count and labor force.  Market power variables are constructed as the overall market share tercile.  Large market is a binary variable for a hospital in top half of the market size distribution.  In cases in which independent variables are missing, we recode them and control for missing variable binary variables to ensure a balanced panel.  Standard errors are clustered at the hospital level.  *** p-value<0.01, ** p-value<0.05, * p-value<0.1.
\end{minipage}
\end{table}


\newpage
\savebox{\gfxbox}{
\begin{tabular}{ccccccc}
\hline	
\hline
 							& Nervous  				& Respiratory  	   	& Circulatory    & Musculoskeletal   		& Labor and & Neonatal \\
							&  System						&  System      	&  System     	&  System        		& Delivery   &	\\
\hline
HRRP and HVBP	&	0.046*	&	0.033*	&	0.040***	&	0.026*	&	0.014	&	0.085***	\\
	&	(0.025)	&	(0.018)	&	(0.015)	&	(0.014)	&	(0.012)	&	(0.029)	\\
HRRP or HVBP	&	0.023	&	0.003	&	0.016	&	0.016*	&	0.009	&	0.054***	\\
	&	(0.015)	&	(0.012)	&	(0.010)	&	(0.010)	&	(0.008)	&	(0.017)	\\
	\hline
n	&	1,615	&	2,085	&	2,985	&	3,295	&	5,278	&	3,305	\\
Mean	&	13,295.81	&	11,609.96	&	12,822.42	&	12,730.49	&	11,188.04	&	9,031.72	\\
\hline
\end{tabular}
}
\setlength{\captionmargin}{.5 \textwidth} \addtolength{\captionmargin}{-.5\wd\gfxbox}
\begin{table}[!h]
\centering
\caption{Payments for Condition Specific Admissions}
\label{tab:eachcondition}
\usebox{\gfxbox}
\par
\begin{minipage}{\wd\gfxbox}
\footnotesize
Notes: All regressions include county and year fixed effects.  Further controls include those in our baseline specification for mean payments.  The dependent variable in each column is the log of the payment for the associated acute care admission.  In cases in which independent variables are missing, we recode them and control for missing variable binary variables to ensure a balanced panel.  Standard errors are clustered at the hospital level.  We restrict the sample to include at least 25 admissions per hospital per year.  *** p-value<0.01, ** p-value<0.05, * p-value<0.1.
\end{minipage}
\end{table}

\newpage
\savebox{\gfxbox}{
\begin{tabular}{lllllll}
\hline	
\hline
 			& Log 				& Log				& Medicare 	   	& Public   		& Other  			& Profit  \\
			& Payment		& Charge			& Share      		& Share       	& Share        	& Index  \\
	\hline
\multicolumn{7}{c}{Non-profit Hospitals}\\
\hline
HRRP and HVBP	&	0.025**	&	0.002	&	-0.014***	&	-0.020***	&	0.020***	&	0.003	\\
	&	(0.010)	&	(0.019)	&	(0.003)	&	(0.006)	&	(0.006)	&	(0.003)	\\
HRRP or HVBP	&	0.013*	&	-0.008	&	-0.009***	&	-0.008**	&	0.008**	&	-0.001	\\
	&	(0.007)	&	(0.013)	&	(0.002)	&	(0.004)	&	(0.004)	&	(0.002)	\\
	\hline
\multicolumn{7}{c}{For-profit Hospitals}\\													
\hline													
HRRP and HVBP	&	-0.011	&	0.000	&	-0.021*	&	-0.015	&	0.015	&	0.003	\\
	&	(0.034)	&	(0.040)	&	(0.011)	&	(0.020)	&	(0.020)	&	(0.003)	\\
HRRP or HVBP	&	-0.017	&	-0.023	&	-0.011***	&	-0.014	&	0.014	&	-0.001	\\
	&	(0.019)	&	(0.027)	&	(0.004)	&	(0.010)	&	(0.010)	&	(0.002)	\\
\hline
\end{tabular}
}
\setlength{\captionmargin}{.5 \textwidth} \addtolength{\captionmargin}{-.5\wd\gfxbox}
\begin{table}[!h]
\centering
\caption{Results by Profit Status}
\label{tab:byprofit}
\usebox{\gfxbox}
\par
\begin{minipage}{\wd\gfxbox}
\footnotesize
Notes: All regressions include county and year fixed effects.  Further controls include those in our baseline specification for mean payments.  In cases in which independent variables are missing, we recode them and control for missing variable binary variables to ensure a balanced panel.  Standard errors are clustered at the hospital level.  *** p-value<0.01, ** p-value<0.05, * p-value<0.1.
\end{minipage}
\end{table}


\newpage
\savebox{\gfxbox}{
\footnotesize
\begin{tabular}{lllllll}
\hline	
\hline
 			& Log 				& Log				& Medicare 	   	& Public   		& Other  			& Profit  \\
			& Payment		& Charge			& Share      		& Share       	& Share        	& Index  \\
\hline
\multicolumn{7}{c}{Small Markets} \\
\hline
HRRP and HVBP   	&	0.004	&	-0.008	&	-0.016***	&	-0.018*	&	0.018*	&	0.002	\\
        	&	(0.016)	&	(0.028)	&	(0.006)&	(0.010)	&	(0.010)	&	(0.006)	\\
\hspace{0.1in} * Med. Mkt. Share	&	-0.015	&	0.017	&	0.001	&	0.000	&	0.000	&	-0.002	\\
        	&	(0.016)	&	(0.026)	&	(0.005)	&	(0.009)	&	(0.009)&	(0.006)	\\
\hspace{0.1in} * High Mkt. Share         	&	-0.011	&	-0.004	&	0.003	&	-0.006	&	0.006	&	0.002	\\
        	&	(0.016)	&	(0.029)	&	(0.005)	&	(0.009)	&	(0.009)	&	(0.006)	\\
HRRP or HVBP	&	0.003	&	-0.004	&	-0.008	&	-0.004	&	0.004	&	-0.001	\\
        	&	(0.012)	&	(0.019)	&	(0.005)	&	(0.007)	&	(0.007)	&	(0.005)	\\
\hspace{0.1in} * Med. Mkt. Share	&	0.001	&	0.024	&	-0.001	&	0.000	&	0.000	&	-0.002	\\
        	&	(0.013)	&	(0.022)	&	(0.005)	&	(0.007)	&	(0.007)	&	(0.004)	\\
\hspace{0.1in} * High Mkt. Share        	&	0.012	&	-0.003	&	-0.001	&	-0.003	&	0.003	&	0.003	\\
        	&	(0.013)	&	(0.023)	&	(0.005)	&	(0.008)	&	(0.008)	&	(0.005)	\\
Market Share	&	0.009	&	-0.052*	&	-0.020**	&	-0.029***	&	0.029***	&	-0.005	\\
\hspace{0.1in} Medium	&	(0.013)	&	(0.029)	&	(0.009)	&	(0.011)	&	(0.011)	&	(0.005)	\\
Market Share 	&	0.030	&	-0.064*	&	-0.016*	&	-0.022*	&	0.022*	&	-0.007	\\
\hspace{0.1in} High     	&	(0.020)	&	(0.037)	&	(0.009)	&	(0.012)	&	(0.012)	&	(0.007)	\\
\hline
\multicolumn{7}{c}{Large Markets} \\
\hline
HRRP and HVBP   	&	0.030	&	-0.045	&	-0.017***	&	-0.024**	&	0.024**	&	0.000	\\
        	&	(0.021)	&	(0.031)	&	(0.006)	&	(0.010)	&	(0.010)	&	(0.005)	\\
\hspace{0.1in} * Med. Mkt. Share	&	-0.008	&	0.006	&	-0.004	&	-0.004	&	0.004	&	0.004	\\
        	&	(0.018)&	(0.027)	&	(0.003)	&	(0.008)	&	(0.008)	&	(0.005)	\\
\hspace{0.1in} * High Mkt. Share         	&	0.001	&	0.031	&	-0.001	&	0.006	&	-0.006	&	0.000	\\
        	&	(0.018)	&	(0.028)	&	(0.004)	&	(0.007)	&	(0.007)	&	(0.004)	\\
HRRP or HVBP	&	-0.004	&	-0.084***	&	-0.013***	&	-0.021***	&	0.021***	&	0.003	\\
        	&	(0.015)	&	(0.023)	&	(0.004)	&	(0.007)	&	(0.007)	&	(0.005)	\\
\hspace{0.1in} * Med. Mkt. Share	&	0.008	&	0.039	&	0.000	&	0.005	&	-0.005	&	-0.003	\\
        	&	(0.016)	&	(0.024)	&	(0.003)	&	(0.006)	&	(0.006)	&	(0.005)	\\
\hspace{0.1in} * High Mkt. Share        	&	0.008	&	0.069***	&	0.004	&	0.007	&	-0.007	&	-0.005	\\
        	&	(0.014)	&	(0.023)	&	(0.003)	&	(0.007)	&	(0.007)	&	(0.005)	\\
Market Share	&	-0.004	&	-0.001	&	-0.003	&	-0.011	&	0.011	&	0.004	\\
\hspace{0.1in} Medium	&	(0.019)	&	(0.025)	&	(0.004)	&	(0.008)	&	(0.008)&	(0.006)	\\
Market Share 	&	-0.008	&	-0.119***	&	-0.022***	&	-0.031***	&	0.031***	&	0.008	\\
\hspace{0.1in} High     	&	(0.023)	&	(0.039)	&	(0.006)	&	(0.011)	&	(0.011)	&	(0.009)	\\
\hline
\end{tabular}
}
\setlength{\captionmargin}{.5 \textwidth} \addtolength{\captionmargin}{-.5\wd\gfxbox}
\begin{table}[!h]
\centering
\caption{Triple Differences by Market Share}
\label{tab:bymktshare}
\usebox{\gfxbox}
\par
\begin{minipage}{\wd\gfxbox}
\footnotesize
Notes: All regressions include county and year fixed effects. Further controls include those in our baseline specification for mean payments.  In cases in which independent variables are missing, we recode them and control for missing variable binary variables to ensure a balanced panel.  We split the sample by market size because of the strong negative correlation between market size and market share. Standard errors are clustered at the hospital level.  *** p-value<0.01, ** p-value<0.05, * p-value<0.1.
\end{minipage}
\end{table}




\newpage
\savebox{\gfxbox}{
\begin{tabular}{cc}
\hline	
 		& Log (Payment)  				 \\
\hline
HRRP and HVBP	&	0.072***	\\
	&	(0.020)	\\
\hspace{0.1in} * Public Share 2	&	-0.065***	\\
	&	(0.018)	\\
\hspace{0.1in} * Public Share 3	&	-0.103***	\\
	&	(0.021)	\\
\hspace{0.1in} * Public Share 4	&	-0.108***	\\
	&	(0.023)	\\
HRRP or HVBP	&	0.024	\\
	&	(0.015)	\\
\hspace{0.1in} * Public Share 2	&	-0.021	\\
	&	(0.017)	\\
\hspace{0.1in} * Public Share 3	&	-0.042***	\\
	&	(0.016)	\\
\hspace{0.1in} * Public Share 4	&	-0.064***	\\
	&	(0.019)	\\
Public Share 2	&	0.055***	\\
	&	(0.013)	\\
Public Share 3	&	0.105***	\\
	&	(0.015)	\\
 Public Share 4	&	0.154***	\\
	&	(0.018)	\\
\hline
\end{tabular}
}
\setlength{\captionmargin}{.5 \textwidth} \addtolength{\captionmargin}{-.5\wd\gfxbox}
\begin{table}[!h]
\centering
\caption{Triple Differences by Public Share}
\label{tab:publicshare}
\usebox{\gfxbox}
\par
\begin{minipage}{\wd\gfxbox}
\footnotesize
Notes: All regressions include county and year fixed effects.  Further controls include those in our baseline specification for mean payments.  The share of a hospital's patients insured by the public sector is broked into quartiles and interacted with penalty variables.  In cases in which independent variables are missing, we recode them and control for missing variable binary variables to ensure a balanced panel.  Standard errors are clustered at the hospital level.  We restrict the sample to include at least 25 admissions per hospital per year.  *** p-value<0.01, ** p-value<0.05, * p-value<0.1.
\end{minipage}
\end{table}




\newpage
\savebox{\gfxbox}{
\footnotesize
\begin{tabular}{lllllll}
\hline	
\hline
 			& Log 				& Log				& Medicare 	   	& Public   		& Other  			& Profit  \\
			& Payment		& Charge			& Share      		& Share       	& Share        	& Index  \\
			\hline
\multicolumn{7}{c}{Non-Profit Hospitals with Penalty Specific Trends} 		\\	
			\hline										
HRRP and HVBP	&	0.019*	&	0.002	&	-0.008**	&	-0.009	&	0.009	&	0.001	\\
	&	(0.011)	&	(0.021)	&	(0.004)	&	(0.006)	&	(0.006)	&	(0.004)\\
HRRP or HVBP	&	0.009	&	-0.008	&	-0.004	&	0.000	&	0.000	&	-0.002	\\
	&	(0.008)	&	(0.014)	&	(0.003)&	(0.004)	&	(0.004)	&	(0.003)	\\
				\hline
\multicolumn{7}{c}{For-Profit Hospitals with Penalty Specific Trends} 		\\		
			\hline									
HRRP and HVBP	&	-0.015	&	0.000	&	-0.021*	&	-0.014	&	0.014	&	-0.001	\\
	&	(0.036)	&	(0.040)	&	(0.012)	&	(0.021)	&	(0.021)	&	(0.008)	\\
HRRP or HVBP	&	-0.019	&	-0.023	&	-0.011***	&	-0.013	&	0.013	&	0.002	\\
	&	(0.019)	&	(0.027)	&	(0.004)	&	(0.010)	&	(0.010)	&	(0.005)	\\
				\hline
\multicolumn{7}{c}{Restricted Sample: 2010-2013} 			\\		
			\hline								
HRRP and HVBP	&	0.023*	&	0.009	&	-0.014***	&	-0.025***	&	0.025***	&	0.001	\\
	&	(0.012)	&	(0.023)	&	(0.004)	&	(0.007)	&	(0.007)	&	(0.004)	\\
HRRP or HVBP	&	0.006	&	0.000	&	-0.008***	&	-0.011**	&	0.011**	&	-0.004	\\
	&	(0.008)	&	(0.015)&	(0.003)	&	(0.004)	&	(0.004)	&	(0.003)	\\
				\hline
\multicolumn{7}{c}{Non-Medicaid Expansion States} 			\\		
			\hline								
HRRP and HVBP	&	0.004	&	-0.041	&	-0.010*	&	-0.002	&	0.002	&	-0.003	\\
	&	(0.014)	&	(0.026)	&	(0.006)	&	(0.009)	&	(0.009)	&	(0.005)	\\
HRRP or HVBP	&	0.003	&	-0.029*	&	-0.007**	&	-0.002	&	0.002	&	-0.005*	\\
	&	(0.009)	&	(0.017)	&	(0.003)	&	(0.005)	&	(0.005)	&	(0.003)	\\
				\hline
\multicolumn{7}{c}{Medicaid Expansion States} 			\\			
			\hline							
HRRP and HVBP	&	0.030**	&	0.023	&	-0.019***	&	-0.030***	&	0.030***	&	0.004	\\
	&	(0.013)	&	(0.023)	&	(0.004)	&	(0.007)	&	(0.007)	&	(0.004)	\\
HRRP or HVBP	&	0.012	&	0.006	&	-0.010***	&	-0.014***	&	0.014***	&	0.003	\\
	&	(0.010)	&	(0.017)	&	(0.003)	&	(0.005)	&	(0.005)	&	(0.003)	\\
				\hline
\multicolumn{7}{c}{Controlling for Medicaid Expansion} 		\\		
			\hline									
HRRP and HVBP	&	0.022**	&	-0.004	&	-0.017***	&	-0.021***	&	0.021***	&	0.002	\\
	&	(0.010)	&	(0.018)	&	(0.003)	&	(0.006)	&	(0.006)	&	(0.003)	\\
HRRP or HVBP	&	0.008	&	-0.013	&	-0.010***	&	-0.010***	&	0.010***	&	-0.001	\\
	& (0.007)	&	(0.012)	&	(0.002)	&	(0.004)	&	(0.004)	&	(0.002)	\\
				\hline
\multicolumn{7}{c}{Controlling for Overall HCAHPS Hospital Rating} 		\\		
			\hline									
HRRP and HVBP	&	0.022**	&	-0.004	&	-0.016***	&	-0.021***	&	0.021***	&	0.002	\\
	&	(0.010)	&	(0.018)	&	(0.003)&	(0.006)	&	(0.006)	&	(0.003)	\\
HRRP or HVBP	&	0.008	&	-0.013	&	-0.010***	&	-0.010***	&	0.010***	&	-0.001	\\
	&	(0.007)	&	(0.012)	&	(0.002)	&	(0.004)	&	(0.004)	&	(0.002)	\\
	\hline
\end{tabular}
}
\setlength{\captionmargin}{.5 \textwidth} \addtolength{\captionmargin}{-.5\wd\gfxbox}
\begin{table}[!h]
\centering
\caption{Robustness Checks}
\label{tab:bymktshare}
\usebox{\gfxbox}
\par
\begin{minipage}{\wd\gfxbox}
\footnotesize
Notes: All regressions include county and year fixed effects. Further controls include those in our baseline specification for mean payments.  In cases in which independent variables are missing, we recode them and control for missing variable binary variables to ensure a balanced panel.  Standard errors are clustered at the hospital level.  *** p-value<0.01, ** p-value<0.05, * p-value<0.1.
\end{minipage}
\end{table}







\newpage
\section{Cost-shifting in Bargaining Model}
\label{app:cs_bargaining}
To more formally examine the presence of cost-shifting in a bargaining context, we embed the hospital cost-shifting model from \cite{dranove1988} in a hospital-insurer bargaining model similar to that in \cite{ho2016} (HL), \cite{gowrisankaran2015}, \cite{lewis2015}, and \cite{dor2004}. Specifically, we consider a not-for-profit hospital whose objective is to maximize some function of profits and quantity of care provided, denoted by
\begin{equation}
 U\left( \pi_{j} = \sum_{i=1}^{N_{j}} \pi_{i,j}^{h} + \pi_{g,j}^{h}, \sum_{i=1}^{N_{j}} D_{i,j}^{h}, D_{g,j}^{h} \right),
\label{eqn:nfp_objective}
\end{equation}
where $\pi_{j}$ denotes total profits for hospital $j$ and $D_{i,j}^{h}$ denotes hospital demand from insurer $i$. Following HL, we assume $$\pi_{i,j}^{h}=D_{i,j}^{h}(p_{i,j}-c_{i})$$, where $p_{i,j}$ denotes the negotiated price between insurer $i$ and hospital $j$. We also follow HL in assuming that patients are ``unaware or unable to determine their [financial] liability prior to choosing their provider.'' In other words, the negotiated price $p_{i,j}$ does not affect demand for a specific hospital. The subscript $g$ denotes profits or demand from public (or government) insurers, for which the price is administratively set at $p_{g}$. Finally, again following HL, we assume that profits for insurer $i$ are
\begin{equation}
\pi_{i}^{M} = D_{i} \left( \theta_{i} - \eta_{i} \right) - \sum_{j=1}^{N_{i}} D_{i,j}^{h} p_{i,j},
\label{eqn:ins_profit}
\end{equation}
where $D_{i}$ denotes the number of enrollees for insurer $i$, $\theta_{i}$ denotes the insurer's premiums, $\eta_{i}$ denotes insurer costs per-enrollee other than inpatient hospital care, and $D_{i,j}^{h} p_{i,j}$ reflects payments to hospitals for care provided to the insurer's enrollees.

The negotiated price between hospital $j$ and insurer $i$ is such that
\begin{equation}
 p_{ij}= \arg \max_{p_{ij}} \left(\triangle U_{j} \right)^{b_{j}} \times \left(\triangle \pi^{M}_{i} \right)^{1-b_{j}},
 \label{eqn:neg_price}
\end{equation}
where $\triangle U_{j}$ denotes the change in hospital $i$'s utility from reaching an agrement with insurer $i$, and similarly $\triangle \pi^{M}_{i}$ denotes the change in insurer profits from an agreement with hospital $i$. $b_{j}$ denotes the bargaining weight of hospital $j$, expressed as the weight to which the hospital's payoffs are given in the overall net value.

The first order condition for equation \ref{eqn:neg_price} can be simplified to
\begin{equation}
 b_{j} \triangle \pi_{i}^{M} \pderiv{U_{j}}{\pi_{ij}^{h}} - (1-b_{j}) \triangle U_{j} = 0.
\label{eqn:price_foc}
\end{equation}
Applying the implicit function theorem yields the relevant comparative static:
\begin{equation}
\deriv{p_{ij}}{p_{g}} = \frac{- b_{j} \triangle \pi_{i}^{M} \pderiv{^{2}U_{j}}{\pi_{j}^{2}}D_{g}^{h}}{D_{ij}^{h}\left(b_{j} D_{ij}^{h} \pderiv{^{2}U_{j}}{\pi_{j}^{2}} - (1-b_{j}) \pderiv{U_{j}}{\pi_{j}} \right)}.
\label{eqn:comp_static}
\end{equation}

We can see immediately from equation \ref{eqn:comp_static} that $\deriv{p_{ij}}{p_{g}}<0$ whenever $\pderiv{^{2}U_{j}}{\pi_{j}^{2}}$. Hospitals must therefore have some diminishing marginal utility of profits for dynamic cost-shifting to occur. Interestingly, this result exists without hospitals deriving utility directly from quantity of care provided, which is necessary for cost-shifting to occur in \cite{dranove1988}. Moreover, a hospital's incentive to cost-shift is larger as the insurer's outside option decreases (i.e., $\triangle \pi_{i}^{M}$ increases) and as the number of public-payer patients increases, $D_{g}^{h}$, while the incentive to cost-shift is reduced if the hospital receives a larger number of patients from insurer $i$.

Practically, equation \ref{eqn:comp_static} suggests that hospitals will be more likely to cost-shift if they have some relative market power, where the insurer is heavily dependent on the hospital but where the hospital does not receive a large number of patients from the insurer. The incentive to cost-shift is also increasing in the hospital's bargaining weight, $b_{j}$, provided the hospital is not ``too'' risk-averse.\footnote{Formally, $D_{ij}^{h}\frac{U''(\cdot)}{U'(\cdot)}>-1$ is a sufficient (but not necessary) condition for $\deriv{p_{ij}}{p_{g}}$ to be decreasing in $b_{j}$.}


\end{document}
