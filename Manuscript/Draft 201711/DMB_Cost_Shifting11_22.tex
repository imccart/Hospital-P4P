\documentclass[12pt]{article}
\usepackage{graphicx,amssymb,amsmath,setspace,comment,verbatim,titling,pgf,lscape}
\usepackage[left=2cm,right=2cm,top=2.5cm,bottom=2cm]{geometry}
\usepackage[round]{natbib}
\usepackage{hyperref}
\usepackage{array}
\usepackage{bbm}
\usepackage{csquotes}


\usepackage[justification=centering]{caption}
\newcommand{\deriv}[2]{\frac{\mathrm{d}#1}{\mathrm{d}#2}}
%%\usepackage{breqn}
\newcommand{\pderiv}[2]{\frac{\partial#1}{\partial#2}}
%\usepackage{siunitx}
\newcolumntype{P}[1]{>{\raggedright\arraybackslash}p{#1}}
\hypersetup{colorlinks,%
						citecolor=black,%
						filecolor=black,%
						linkcolor=black,%
						urlcolor=blue,%
						}


\setlength{\droptitle}{-50pt}

\begin{document}

\title{Do Hospitals Cost-Shift? New Evidence from the Affordable Care Act}
\author{%
  Michael Darden, Ian McCarthy, and Eric Barrette\thanks{Darden: George Washington University. McCarthy: Emory University and NBER. Barrette: Health Care Cost Institute.  Correspondence: \href{mailto:darden@gwu.edu}{darden@gwu.edu}; \href{mailto:ian.mccarthy@emory.edu}{ian.mccarthy@emory.edu}; \href{mailto:ebarrette@healthcostinstitute.org}{ebarrette@healthcostinstitute.org}.}
}
\date{November 2017}

\maketitle

\begin{abstract}
A longstanding debate in health economics and health policy concerns the extent to which hospitals increase prices to private insurance patients following reductions in public funding.  This ``cost-shifting" hypothesis is ubiquitous in health policy debates, but evidence of significant cost-shifting is mixed, and the rationale for such behavior is unclear.  We enter this debate by examining plausibly exogenous variation in Medicare reimbursement rates generated by two policies under the Affordable Care Act: the Hospital Readmission Reduction Program (HRRP) and the Hospital Value Based Payment (HVBP) program.  We merge rich hospital level information on reimbursements to actual private-payer payment  data from a large, multi-payer database.  Our data include roughly 50$\%$ of inpatient prospective payment hospitals in the United States from 2010 to 2015.  We find that hospitals that faced reimbursement reductions from both programs were able to negotiate 1.6$\%$ higher average private payments - approximately $\$$165 extra for the average acute care claim.  This result is completely driven by non-profit hospitals.  We find the largest increases in payments for circulatory system (2.5$\%$), nervous system (2.2$\%$), and neonatal (2.1$\%$) claims.  Furthermore, we find significant heterogeneity by payer mix, where hospitals with larger shares of private patients cost-shift more.  Taken together, our results provide support modest levels of cost-shifting and are consistent with a model in which hospitals maximize utility over profits and quantity while bargaining over prices with private insurance firms.
\end{abstract}
\noindent \textit{JEL Classification:} I11; I18; L2 \\\\
\noindent \textit{Keywords:} Cost-Shifting; Hospital Behavior; Affordable Care Act.\\\\
\setstretch{1.3}

\newpage
\section{Introduction}
A longstanding debate in health economics and health policy concerns hospital ``cost-shifting'' - the ability of a hospital to pass public patient reimbursement cuts to privately insured patients by negotiating for higher payments from private insurance companies.  Insurance companies have long maintained that Medicare in particular is not paying its fair share \citep{frakt2011}.  Furthermore, the assumption of cost-shifting as common hospital practice is ubiquitous in health care policy debates.  For example, in the debate over the Affordable Care Act, President Obama said:
\begin{quote}
\textit{``You and I are both paying 900 bucks on average - our families - in higher premiums because of uncompensated care.''\footnote{For additional examples, see the many excerpts in \cite{dranove2017}, including additional statements from President Obama and the U.S. Supreme Court regarding the Affordable Care Act.}- Barack Obama}
\end{quote}
And there is some evidence to support the notion of cost-shifting. Studying California hospitals from 1993-2001, \cite{zwanziger2006} estimate large effects on private payments due to reductions in Medicare and Medicaid reimbursement rates, mirroring the findings of \cite{lee2003} and \cite{zwanziger2000}. \cite{zwanziger2006} estimate that cost-shifting can explain 12.3\% of the total increase in private payers' payments from 1997 to 2001.  Yet, in a systematic review of the literature, \citet{frakt2011} concludes that cost-shifting, if it exists, is not widespread and is not a main driver of increased health care costs.   While a simple model of monopoly pricing with price discrimination would suggest different prices for public and private patients \citep{hay1983}, that model would also suggest a \textit{decrease} in the private price for a decrease in public reimbursements.  Furthermore, for both for-profit and non-profit firms, it is unclear why a hospital with the power to raise prices would not have already done so.\footnote{Several papers find zero or even negative price effects given public reimbursement cuts. \cite{white2013} examines market level pricing from the Truven MarketScan data and finds that private prices decrease following reductions in Medicare reimbursement rates. Exploiting a change in Medicaid payment policies in California, \citet{dranove1998} similarly found little evidence of cost-shifting. More recently, using the 2008 stock market collapse as an exogenous change to hospital endowments, \cite{dranove2017} find that the average hospital does not appear to cost-shift, with some evidence of cost-shifting among hospitals with sufficient market share.}

We enter the cost-shifting debate by exploiting the 2013 adoption of the hospital readmission reduction program (HRRP) and hospital value based purchasing program (HVBP), which are both components of the Affordable Care Act's (ACA) cost containment strategy, and which both serve as plausibly exogenous changes in Medicare reimbursement rates.  Initially, the HRRP penalized hospitals for which 30-day readmissions for acute myocardial infarction (AMI), heart failure (HF), and pneumonia (PN) exceeded risk-adjusted thresholds constructed as a function of national averages.  Starting in Fiscal Year (FY) 2013 (October 2012-September 2013), hospitals faced a maximum cut in Medicare reimbursements of 1$\%$ across all Diagnosis Related Groups (DRG); the maximum potential penalty increased to 3$\%$ by 2015.  The \cite{cbo2010} estimates that HRRP will reduce hospital reimbursements from Medicare by $\$$113 billion through 2019.  \citet{mellor2016} finds that HRRP has been associated with declines in AMI 30-day readmission rates, and \citet{gupta2016} finds a 5$\%$ reduction in overall readmissions and 3$\%$ reduction in all-cause mortality, mostly driven by quality improvements.\footnote{However, recent evidence from \citet{gupta2017} suggests that the HRRP may have increased heart failure mortality despite decreasing 30-day readmissions.}  By contrast, starting in FY 2013, HVBP reduces reimbursements to all hospitals by 2$\%$, but it rewards hospitals with incentive payments for their quality of care over a variety of quality domains which represent over 85$\%$ of total Medicare reimbursements.  As a result of these potential rebates, hospitals have the opportunity to receive a net reimbursement increase if the rebates exceed the initial 2$\%$ reduction.  \citet{norton2016} provide evidence that quality of care improved as a result of HVBP, but only for services with the highest marginal incentives to improve quality of care.  We argue that both HRRP and HVBP generated significant and plausibly exogenous cross-sectional and temporal variation in Medicare reimbursements.  

An important contribution of our paper is the use of data on actual payments from private insurance firms to hospitals.\footnote{Throughout, rather than use the term ``price'', we refer to the financial transfer for a given procedure as the ``payment'' from a private insurance firm to a hospital.  A payment is distinctly different than a hospital ``charge," which represents a hospital's list price for a give procedure.  Private insurance firms negotiate substantial discounts from charges.}  Our data, maintained by the Healthcare Cost Institute, contain all claims made for acute care hospital admissions to three national commercial insurers.  \cite{gowrisankaran2015} and \cite{cooper2015} use similar data to model hospital/insurer price bargaining, but to our knowledge, we are the first to use actual payment data in a study of cost-shifting.   These unique data include payments for every claim, which capture the negotiated payments between hospitals and insurers, and which may differ substantially from charge-based estimates of payments often used in the literature \citep{dafny2009,dranove2017}. Indeed, in our data, the correlation between actual payments and a payment proxy estimated from the Healthcare Cost Report Information System (HCRIS) is 0.435, suggesting that charge-based estimates of payments may contain significant measurement error.  Furthermore, with payment data and a balanced panel of hospitals, we are able to investigate the extent to which hospital fixed effects adequately control for the mean payment to charge ratio within a hospital in a model of log charges \citep{cutler2000}.  We augment our payment data with publicly available information on hospital costs and local area characteristics.  Our final hospital payment data cover approximately 28$\%$ of individuals under the age of 65 who have employer-sponsored insurance (ESI), and they constitute a balanced panel of  50$\%$ of all inpatient prospective payment hospitals in the United States between 2010 and 2015.       


Because net reimbursement rate changes under HVBP may be positive or negative (as opposed to HRRP, which are always zero or negative), we calculate the total change in reimbursement rate from both policies and construct a binary variable that equals one if the net change is negative.   Our baseline empirical specification is a hospital fixed effects estimator in which we estimate the difference in average payments between those hospitals with a net penalty under the HRRP and HVBP relative to those not penalized before and after 2013.  This specification holds constant the case mix of hospital.  Our results reveal an increase in average payments of 1.6$\%$ for hospitals penalized by these policies relative to those not penalized, equivalent to a roughly $\$$165 increase at the mean, in the period from 2013 through 2015. Consistent with reimbursement reductions, we find that penalized hospitals decreased Medicare discharges by 2.2$\%$, while discharges for private patients did not significantly change.  We also show no significant effect of reimbursement reductions on hospital profit, which demonstrates that penalized hospitals must have compensated for rate cuts on some dimension.   

Our results are unbiased if there are no unobserved, time-varying factors that influenced payments that were also correlated with penalty status.  While we cannot directly test this assumption, three facts provide supporting evidence for a causal interpretation of our results.  First, while implementation of the ACA generates concern that our observed effects are being driven by un-modeled changes in the health care environment, the quality metrics that enter into construction of the HRRP and HVBP reimbursement rates in a given year are calculated based on data from the previous three years.  Thus, in 2014 and 2015, our net penalty variable was largely pre-determined.  And still, a competing explanation for our results may be that hospitals penalized under the HRRP and HVBP may have been able to negotiate higher private prices not because of penalties, but because of anticipated differential impacts of the full ACA.\footnote{There is evidence that hospitals located in low-SES areas were more likely to be penalized under HRRP, despite risk-adjustment at the patient level.}  However, controls for whether a hospital operates in a state that expanded Medicaid, as well as the inclusion of county fixed effects, do not change our main results.   Second, we demonstrate that allowing trends in payments to vary by whether a hospital ever experiences a reimbursement reduction does not significantly change our conclusions.  Finally, our results are robust to variety of specification checks, including dropping fiscal 2012, when the policies affected only a subset of hospitals.  Our baseline results provide support for the notion of a small degree of cost-shifting in the modern health care environment.  

Because theoretical predictions suggest that purely profit-motivated firms are unlikely to cost-shift \citep{hay1983}, economic rationales for cost-shifting have focused on a hospital's objective function.  \cite{dranove1988} models a hospital as a utility maximizer, where utility is defined over both profit and quantity.  A hospital may directly value quantity of care for reasons of altruism or prestige, or simply because a non-profit hospital must provide some form of ``community benefit'' to maintain its tax-exempt status.\footnote{The Congressional Budget Office defines community benefits as services geared toward ``promoting the health of any broad class of persons'' \citep{cbo2006}.}.  Thus, if cost-shifting exists, evidence is anticipated to be isolated primarily among non-profit hospitals. Consistent with this prediction, when we break our analysis by profit status, we find that our baseline result of a 1.6$\%$ increase in private payments is likely to be driven by non-profit hospitals.  We find an insignificant and statistically smaller effect of a penalty on payments to for-profit hospitals.    

Complicating matters, hospital payments are not merely set by hospitals, but rather they are negotiated with private insurance firms. In the context of such a negotiation, the mechanism for cost-shifting in response to \textit{penalties} incurred from lower-than-expected quality is particularly unclear. How can a penalty for low quality enable the hospital to negotiate a higher payment? We argue that three potential scenarios may allow for such an outcome: 1) if the quality information revealed by the penalty is not new information to the market, then the penalty is simply a reduction in the public reimbursement rate and the underlying source of the penalty is irrelevant; 2) if the information is new, hospitals may exploit the penalty in other service areas (i.e., those not tied to quality scores) where they have a comparative quality advantage in the market; or 3) hospitals with very high shares of public patients may be particularly burdened by the HRRP and HVBP penalties such that the hospital relies on higher private payments to operate. In this latter case, private insurers may allow higher prices so as to maintain leverage in negotiations with other hospitals in the local market. We address each of these mechanisms with a series of alternative specifications and hospital samples.

To address the first scenario, we add a control variable for the overall hospital quality ranking given annually by Hospital Consumer Assessment of Healthcare Providers and Systems (HCAHPS).  We find no difference in the effect of reimbursement cuts on payments when comparing hospitals with identical overall quality rankings relative to omitting this control.  To address the second scenario, we construct measures of average payments within specific acute care admission categories within a hospital.  We re-estimate our preferred specification separately for each admission category, and we find marked increases in payments for circulatory system (2.5$\%$), nervous system (2.2$\%$), and neonatal (2.1$\%$) claims, which suggest that cost-shifting may occur regardless of whether a specialty is directly related to reimbursement cuts (such as AMI and HF within HRRP).  For the final scenario, we find that our main effect of interest varies by the share of a hospital's patient mix with public insurance; however, consistent with \cite{wu2010}, we find that hospitals with a \textit{smaller} share of Medicare patients engage in more cost-shifting.  We argue that, with evidence from \citet{wu2010}, hospitals with the largest shares of private patients may represent an important client to private insurers, and hospitals may leverage this power when faced with reimbursement cuts.  

Because of the confidential nature of hospital/insurer bargaining, we cannot completely rule out that higher prices for penalized hospitals where due to other, unobserved factors that are correlated with penalties under HRRP and HVBP.  However, this problem is present in any study that lacks information on the bargaining process, and our results are consistent across number specification checks. 
Finally, while we emphasize that payment data is important in precisely estimating cost-shifting, we demonstrate that hospital level fixed effects may adequately control for within hospital mean differences in payments and charges.  Taken together, our results demonstrate that a small degree cost-shifting occurred following the HRRP and HVBP.   

Following a discussion of the setting and a detailed explanation of each policy in Section \ref{sec:Discussion}, we present our data, empirical methods, and baseline findings in Section \ref{sec:Empirical}. Section \ref{sec:Ext} considers several extensions and testable hypotheses implied by the standard theoretical model of cost-shifting, and Section \ref{sec:Discussion} puts these results in context and presents an extension of the model in \cite{dranove1988}.  Section \ref{sec:Conclusion} concludes.

\section{Background}
\label{sec:Background}

The debate over whether, and the extent to which, hospitals shift costs has been ongoing for decades. While private insurers are naturally averse to higher private prices, hospitals have emphasized the need to cost-shift in an attempt to lobby for larger public reimbursement. For example:
\begin{quote}
\textit{``Cost shifts have been a fact of hospital financial survival for decades.... The data show ...  how private payment is a mirror image of public payment over time and that the cost shift occurs. Hospitals must make up for shortfalls through a combination of approaches and cost-shifting is among them.'' -Rich Umbdenstock, Former President and CEO of American Hospital Association}\footnote{\href{$http://blog.aha.org/post/costshifting-in-hospitals-$}{American Hospital Association Stat}}
\end{quote}

In the end, the argument that cost-shifting occurs is easily motivated by observed trends.  In 2015, 55 million Americans were enrolled in Medicare, up from 37.5 million in 1995, and from 1980 to 2014, the share of hospital costs attributable to Medicare rose from 34.6$\%$ to 40.2$\%$.  Meanwhile, in 2014, hospitals endured a shortfall of $\$$35 billion of Medicare payments relative to Medicare patient costs, as compared with a $\$$5 billion surplus relative to costs in 1997.  During this same period, patients insured by private payers became increasingly lucrative: in 2014, the payment-to-cost ratio of privately insured patients was roughly 140$\%$.\footnote{All statistics from the American Hospital Association Trendwatch Chartbook, 2016} Consistent with the trend in profitability of private insurance patients to hospitals, average premiums for covered workers with family coverage increased by 69$\%$ from 2000 to 2005.\footnote{Kaiser Family Foundation}

While the current conditions for cost-shifting appear to be ripe, much of the evidence of significant cost-shifting comes from the 1980s and 1990s.  For example, \cite{cutler1998costshift} studies cost-shifting during the phase-in of Medicare prospective payments during the 1980s, which resulted in an average 2$\%$ per year reduction in Medicare reimbursements.  He found evidence of dollar-for-dollar cost-shifting.  More recently, \cite{zwanziger2006} study the late 1990s and found that, between 1997 and 2001, cost-shifting was responsible for roughly 12$\%$ of the observed increase in total private payer prices.

In contrast, the simplest argument against cost-shifting as a significant mechanism in the hospital market is one of basic microeconomics.  A for-profit firm with market power who sells to two groups should not respond to an exogenous decline in the price to one group by raising prices to the second group.  \cite{hay1983} shows that, even when the government commits to reimbursing the full average cost of Medicare patients, hospitals will: a) still charge a higher price to privately insured patients; and b) respond to lower Medicare reimbursements with \textit{lower} private prices.\footnote{\cite{dor1996} demonstrate that payer-specific marginal costs may be evidence of differential treatment by hospitals.}

In spite of the evidence presented by \cite{cutler1998costshift} and \cite{zwanziger2006}, the empirical evidence of cost-shifting is notably weak.  In a 2011 review of this literature, Austin Frakt states:
\begin{quote}
\textit{``In fact, as a whole, the
evidence does not support the notion that cost-shifting is both large and
pervasive. Instead, it reveals that cost-shifting can occur but may not
always do so. When it has occurred, it has generally been measured at a
rate far below dollar-for-dollar''}.
\end{quote}

Indeed, numerous studies have found zero or even negative overall price effects, including \cite{dranove2008impact}, \cite{wu2010}, \cite{frakt2014}, and \cite{dranove2017}, but potentially important positive price effects for certain subgroups.  For example, \cite{wu2010} shows that hospitals with large shares of private patients (relative to Medicare patients) were able to cost-shift following the 1997 Balanced Budget Act, perhaps due to greater bargaining power.

We argue that identification of cost-shifting behavior is inherently difficult for three reasons.  First, the hospital market is incredibly complex.  In addition to many different types of payers, the industry is heavily regulated, and policy changes occur frequently.  We study two exogenous sources of reimbursement variation in which complexity is arguably a benefit to identification -- as discussed below, a common complaint among hospitals prior to the implementation of HRRP and HVBP was that the policies were opaque with respect to the reimbursement reduction calculation.  Furthermore, because reimbursement reductions were based on past quality metrics, hospitals could only avoid future penalties with quality improvements.  And yet, the conditions and procedures being evaluated and the formulas by which reimbursements are calculated, continually change.  Second, measurement error in private payments may be severe.  Because private payments are typically not observed, many of the referenced papers above must proxy for private payments, often with charges or costs.  In our paper, as discussed below, we observe private payments -- the actual dollar amount of reimbursement from three large private insurers to hospitals.  Finally, heterogeneity in hospital responses to public reimbursement cuts may muddle instances of important cost-shifting.  Indeed, as we emphasize below, our bargaining model of hospital payments predicts that cost-shifting will be largest in markets in which a hospital has the greatest \textit{relative} market power.  That is, the market power of a hospital in the provider market is large relative to the market power of any given insurer in the insurance market.  Yet our data do not include good measures of insurance market concentration at a local level.  We attempt to examine this heterogeneity with a series of alternative specifications and supplemental analyses, the most important of which proxies for hospital market-power with the share of public patients.

\subsection{Policy Environment: HRRP and HVBP}
The adoption of the Medicare prospective payment system (PPS) in 1983, in which hospital reimbursement changed from pure fee-for-service to a capitated amount for each inpatient stay depending on diagnosis, generated incentives for hospitals to cut ``excessive'' procedures. PPS also created incentives for hospitals to discharge patients quickly.  By 2011, Medicare paid $\$$24 billion per year for 1.8 million hospital \textit{readmissions} - admissions to any hospital within 30-days of discharge for the same condition.  While some readmissions are unavoidable, the HRRP was a cost containment in the ACA designed to create hospital reimbursement penalties for ``excessive'' readmissions.  Starting in 2013, hospitals with risk-adjusted readmissions in acute myocardial infarction, heart failure, and pneumonia that exceeded national comparison averages saw overall Medicare reimbursement cuts of up to 1$\%$.   In 2015, the maximum penalty increased to 3$\%$, total penalties rose to $\$$420m (Rau, 2015), and applicable conditions where expanded to include chronic obstructive pulmonary disease and total hip and knee replacements.  Evidence has been suggestive that HRRP has reduced readmissions in the tested conditions.  For example, Mellor \textit{et al.} (2016) found that HRRP was associated with declines in AMI readmission, which were not due to delay of treatment, changes in intensity, or selective patient mix. And Gupta (2016) find a 5$\%$ reduction in overall readmissions and 3$\%$ reduction in all-cause mortality, which was mostly driven by quality improvement.

By contrast, the Hospital Value-Based Purchasing (HVBP) program is rooted in a standard principal-agent model in which the principal (CMS in this case) contracts with agents (hospitals) to provide quality care to Medicare enrollees. The HVBP program scores hospitals based on their achievement (comparison to other hospitals) as well as their improvement (comparison to their own previous performance).  Similar to the HRRP, the HVBP bases changes in reimbursement on past quality.  However, unlike the HRRP, the HVBP program is funded by reducing all hospitals' base operating Medicare severity diagnosis-related group (MS-DRG) payments by 2$\%$ and creating rebate incentives depending on defined quality metrics.  The program defines several quality domains and converts measures of quality within each domain to points, which are aggregated and mapped to a total point score.  The total point score determines the magnitude of the reimbursement change, which may be positive or negative depending on if a hospital generates a rebate large enough to offset the 2$\%$ reduction.  Not surprisingly, \cite{norton2016} show that HVBP generated quality improvements in the quality domains with the highest marginal incentive to improve care.  



\section{Empirical Methods}
\label{sec:Empirical}


\subsection{Data}
Our primary data come from three large health insurance firms that insure roughly 28$\%$ of all individuals under the age of 65 with employer sponsored health insurance over the period of 2010 through 2015.  To these data, we merge information on ACA based reimbursement penalties from archived Hospital Compare reports; hospital cost information from Healthcare Cost Report Information System (HCRIS); hospital-level characteristics such as bed count, for-profit status, and system membership from the American Hospital Association (AHA) annual surveys; data on a hospital's payer mix (i.e., the number and share of Medicare, Medicaid, or private insurance patients) also from HCRIS; and local area level characteristics from the American Community Survey (ACS).  We restrict our sample to those hospitals with at least 25 admissions in a given year and those with nonmissing values of payments.  Our final sample consists of 1,644 hospitals and 9,864 hospital/year observations.

Following \citet{gowrisankaran2015}, who use similar payment data from Northern Virginia, we aggregate payments to the hospital level by dividing the total payment for each claim by the appropriate diagnosis group weight (DRG) and regressing this amount on individual (claimant) and hospital characteristics.  Using the estimated regression results, we predict the risk-adjusted mean hospital payment for a given year, which reflects the mean bargained payment.  Table 1 presents mean payments across hospitals over time. While average risk-adjusted payments received by hospitals increase roughly 5$\%$ between 2010 to 2015, shares of public, Medicare, Medicaid, and private patients remain relatively stable over time.  Importantly, while shares remain stable, within-hospital patient mix may vary considerably over time as a function of public reimbursements, which is why we treat payer-specific discharges as a separate dependent variable.  The last column of Table 1 shows the fraction of hospitals subject to a net Medicare reimbursement reduction.  Because the CMS fiscal year is from October through the following September, 30$\%$ of hospitals faced a penalty in calendar year 2012 (after October) because of discrepancies between the fiscal year of the hospital and that of CMS.  By 2015, 80$\%$ of hospitals faced some reimbursement reduction.  

Because our baseline empirical specification below depends on within hospital variation, we split our sample by whether a hospital was ever observed face a reimbursement reduction under HRRP and HVBP during our sample period.  Table 2 presents summary statistics of our main dependent variable and some independent variables by ever-penalized status.  Log payments to ever-penalized hospitals are marginally higher than to those penalized when averaged over the period of 2010 to 2015.  Non-profit hospitals constituted a much larger share of never-penalized hospitals, suggesting that non-profit hospitals may be of higher quality, at least in terms of HRRP and HVBP.  However, case mix is significantly more severe in the ever-penalized hospitals, which suggest that CMS risk-adjustment in HRRP and HVBP may not perfectly adjust penalty thresholds.  Ever-penalized hospitals tend to be in more competitive markets, have lower Medicare share and higher profits, and come from more heavily populated counties.  Evidence from Table 2 suggests that controlling for hospital fixed effects is important in models of hospital payments because of time-invariant differences based on penalties.  

\subsection{Research Design}
Our baseline empirical specification isolates within hospital variation in private payments over time by whether a hospital faced a net penalty from the HRRP and HVBP. This analysis therefore focuses on the extensive margin of penalties.  Equation 1 presents our main empirical model:
\begin{equation}
\label{eq: reg}
y_{hct} = \alpha_{h} + x^{'}_{hct}\beta + \delta1[Penalty]  + \theta_{t}  +  \epsilon_{hct},
\end{equation}
where outcome $y_{hct}$ at hospital $h$ in county $c$ in fiscal year $t$ is a function of a hospital specific intercept, $\alpha_{h}$; a vector of time-varying exogenous characteristics, $x_{hct}$; an indicator for a net penalty under both of the policies; controls for year effects, $\theta_t$; and an i.i.d. error term $\epsilon_{hct}$.  Because the indicator for penalty is zero for all hospitals 2010 and 2011 and because we include hospital fixed effects, Equation 1 represents a difference-in-differences estimator, where our parameter of interest, $\delta$, captures the extent to which hospitals penalized under the ACA receive differential private payments relative to hospitals with no penalty.  The underlying assumption in Equation 1 is that there are no time-varying unobserved characteristics that differentially affect payments in penalized hospitals relative to non-penalized hospitals.  

Table 3 presents results from Equation 1 for the log of payments, the log of charges, a variety of (logged) payer-specific discharge variables, and for an index that captures the profitability of a hospital.  The first column of Table 3 demonstrates that hospitals that faced reimbursement reductions increased payments by 1.6$\%$ over the period of 2012-2015.  This represents a roughly $\$$165 increase in the average payment at the mean payment reported in Table 1.  Column 2 presents estimates from a similar model in which we replace payments with the log of charges.  Results in column 2 suggest a smaller and statistically insignificant change in charges for penalized hospitals, which we argue demonstrates the importance of payment data.  Columns 3 and 4 of Table 3 show movement \textit{away} from Medicare and Medicaid patients for penalized hospitals, with discharges decreasing by 3.7$\%$ and 2.2$\%$, respectively.  Both of these effects are significant at least the 90$\%$ level of confidence.  Finally, the last column of Table 3 shows results from a model of hospital profit.  \citet{dranove2017} argue that hospitals may absorb the cost of financial shocks in the form of lower profits, but our results suggest that profits may (weakly) increase if hospitals are able to shift the burden of reimbursement cuts to private insurers.  

As initially examined in \cite{dranove1988}, a hospital may pursue a cost-shifting strategy if the hospital's objective function includes something other than pure profit (e.g., if the hospital receives direct utility from the quantity of services provided). For this reason, cost-shifting is thought to more likely occur among non-profit hospitals, if at all. Indeed, to maintain their tax exempt status, non-profit hospitals are required by the IRS to provide community benefits.\footnote{Of course, this does not mean that non-profit hospitals are fully altruistic. In fact, evidence on non-profit hospital behavior relative to for-profit hospital behavior is mixed. For example, \cite{silverman2004} and \cite{dafny2005} find evidence that non-profits ``upcode'' less frequently, while \cite{gaynor2003} find that non-profit hospitals have lower marginal costs but higher markups than for-profit hospitals.} Since over 80$\%$ of hospitals in our sample are non-profit, this implies that the objective function of the majority of hospitals in our analysis extends beyond pure profit-maximization.  Importantly, the model posited in \cite{dranove1988} assumes that hospitals set payments unilaterally, and it is not immediately clear whether this prediction extends to a modern managed care market in which hospitals and private insurers negotiate over private prices. In Section \ref{sec:Discussion}, we offer a minor extension to the model of \cite{dranove1988} and show that much of the same intuition applies in a bargaining context as well. One interesting difference is that the hospital may still cost-shift even if the objective function is based solely on profit - what is required is that there exists diminishing marginal utility of profits, a mechanism that we would expect to be strongest among non-profit hospitals.

To investigate the extent to which cost-shifting varies by profit status, we re-estimate Equation 1 separately for non-profit and for-profit hospitals.  The results are presented in Table 4.  Consistent with the literature, our results suggest a statistically insignificant (and smaller in magnitude) effect of the HRRP and HVBP penalties on private payments for for-profit hospitals.  The estimated coefficients on penalty for non-profit and for-profit hospitals are significantly different from each other at all standard levels of significance.  As noted above, our model of hospital bargaining does suggest that for-profit hospitals may cost-shift, but our empirical results suggest that the results presented in Table 3 are likely driven by non-profit hospitals.  Indeed, the point estimate for non-profit hospitals is statistically significant and identical to results presented in Table 3.





\section{Empirical Extensions}
\label{sec:Ext}
The underlying mechanisms for cost-shifting in our context are unclear if we consider the role of hospital and insurer bargaining. In particular, the reduction in public payments that serve to identify the presence of cost-shifting in our analysis derive from a lower-than-expected performance on some set of quality metrics. For cost-shifting to occur, hospitals must translate a signal of low quality into higher prices.  If the quality information revealed by the penalty is new information to the market, penalized hospitals may target other, unrated service areas where they have a comparative quality advantage in the market.  Results in Table 3 suggest evidence of cost-shifting on average, but to investigate further, we estimate models of the log of payments within acute care admission service categories.  Because two of the three original conditions rated by the HRRP (AMI and heart failure) were circulatory system conditions, an open question remains as to whether a cost-shifting hospital would renegotiate payments for these services.  For example, the revelation of a penalty for poor performance with respect to specific circulatory conditions may force a cost-shifting hospital to focus negotiation on other services.  

Results of $\delta$, our parameter of interest, are presented in Table 5 for several categories of acute admissions.  For each admission category, we restrict our sample to those hospitals with at least 25 admissions in that category in each year of our sample.  Table 5 demonstrates significant increases in payments for nervous, circulatory, and neonatal admissions, suggesting that cost-shifting may occur for multiple types of admissions.  

In the context of price negotiations for a single insurer-hospital pair, insurers generally want to pay lower prices while hospitals want to receive higher prices; however, insurers may not pursue such negotiations independently across hospitals. For example, most hospital markets are relatively concentrated with just a handful of hospitals in operation. The loss of one hospital in the market may therefore increase the remaining hospitals' bargaining power over an existing insurer. As such, insurers may be willing to increase prices for a single hospital if such a price increase prohibits the hospital from otherwise closing or merging with a competitor. In this case, hospitals most exposed to the HRRP and HVBP (i.e., hospitals more dependent on publicly insured patients) may be able to disproportionately increase private insurance prices out of insurers' willingness to maintain more competitors in the hospital market.  However, \citet{wu2010}, who studies hospital cost-shifting as a result of the Balanced Budget Act of 1997 and proxies for hospital bargaining power with a hospital's share of private paying patients, finds the opposite result - hospitals with larger shares of private patients were more able to pass Medicare reimbursement reductions on to private payers.  The intuition behind this result is that a hospital with a large share of private payers represents a more important client for the insurance market, and the hospital may leverage this power when public reimbursements are cut.  

To investigate, we construct a variable which represents the quartile of hospital public patient share, and we interact our penalty variable with binary variables for each quartile.  Results are presented in Table 6.  Similar to \citet{wu2010}, we find that our results in Table 3 are driven by hospitals with the smallest share of public patients.  Indeed, the first column of Table 6 demonstrates that payments increased by 3.9$\%$ for hospitals with the smallest share of public patients, and that increase was completely canceled out for hospitals with the second and largest shares of public patients.  

The model of \cite{dranove1988} shows that hospitals with a large market share have a greater ability to cost-shift.  However, as shown in Section \ref{sec:Discussion} below, we would expect cost-shifting to occur most in areas where hospitals not only have a large market share but also where a hospital's market share is large relative to an insurer's. This is the natural extension of \cite{dranove1988} to a bargaining context.  However, this prediction is less clear when we incorporate the insurer's choice of premiums in the insurance market. If the insurance market is heavily concentrated, then insurers can pass health care price changes to their plan enrollees \citep{ho2016}.  This intuition leads to conflicting conclusions: cost-shifting is likely to occur when insurers have a particularly small market share (such that hospitals can leverage market power), but perhaps also when insurers have a particularly large market share (such that insurers can pass price increases on to insureds). The role of insurance markets on the prevalence or magnitude of cost-shifting is therefore empirically difficult to measure without detailed data on insurance premiums and insurer market shares (at a local level).  Because we lack reliable information on local area insurance concentration, we leave as an open question the extent to which cost-shifting is more prominent in markets with both provider market power and a concentrated insurance market. 


\subsection{Robustness}
Results in Table 3 reflect the causal effect of the HRRP and HVBP on payments if there are no unobserved, time-varying factors that influence payments and are also correlated with penalty status.  While we cannot completely rule out this possibility, we estimate  a variant of Equation \ref{eq: reg} in which we allow the trend in payments to vary by whether a hospital is ever-penalized.  Differential trends in prices conditional on penalty status and other controls would be suggestive of time-varying unobserved heterogeneity and would generally bias our estimate of $\delta$ towards zero.  Rows 1 and 2 of Table 7 report estimates of $\delta$ with differential trends for non-profit and for-profit hospitals, respectively.  Our results are broadly consistent with Table 4.

 While row 3 of Table 7 demonstrates that our results are robust to the inclusion of county-level fixed effects, which suggests that local area variation in provider or insurer markets are not driving our results, row 4 presents results from a model that includes controls for whether a state expanded Medicaid in 2014.  Because the 2014 saw the full implementation of the ACA, we are concerned that other changes in the hospital-insurer relationship may drive payments after 2014.\footnote{A model in which we dropped 2014 and 2015, relying on only 2013 as our ``post'' period found an estimate of $\delta$ of 0.08, which was not significant.  Given that, to cost-shift, a hospital must renegotiate its payments, we argue that one year after the policy is insufficient to detect a significant effect.  Furthermore, because hospitals complained about uncertainty with respect to reimbursement cuts associated with the ACA, we argue that it was unlikely that hospitals anticipated with full reimbursement cut prior to the implementation of HRRP and HVBP}.  Row 4 demonstrates that our results are robust to the inclusion of controls for other aspects of the ACA.     

Existing findings from \cite{dranove2008} and others tend to find relatively small effects of quality reporting on hospital choice. As \cite{dranove2008} state, ``report cards do not always convey `news' about quality; in some cases the rankings confirm with prior beliefs about quality.'' To the extent that penalties from the HVBP and HRRP do not reveal any new information to the market, then the penalty acts simply as a reduction in public reimbursements and the intuition from our model in Section \ref{sec:Discussion} applies. We examine this issue with an alternative specification in which we control for a hospital's overall quality as measured by patients' overall hospital rating from the Hospital Consumer Assessment of Healthcare Providers and Systems (HCAHPS).  Row 5 of Table 7 reports results from this model, and our results with respect to log payments are identical to those in Table 3.  

Finally, because of discrepancies across hospitals and between hospitals and CMS in fiscal year definition, the exact timing of the realization of reimbursement cuts varies across our sample. In row 6 of Table 7, we report results from a model in which we drop fiscal 2012 from our analysis, and our results are similar to those in Table 3. 

The novel aspect of our data are that, for a given acute care claim, we observe the actual payment from a private insurer to hospital.  Payments are distinctly different than charges, which reflect a hospital's list price for a given service, as the payment made by an insurer reflects an endogenously bargained discount on the charge.  As noted above, the correlation in our data between payments and charges is 0.435, which suggests that measurement error in a model of log charges could be significant.  Most studies of hospital payments proxy for payments with hospital charges and argue that hospital fixed effects control for mean differences between charges and payments \citep{cutler2000}.  The last row of Table 7 presents results of Equation 1 without hospital fixed effects for each of our dependent variables.  Estimates of $\delta$ for both log payments and log charges are negative and significant, which suggests that a.) permanent and unobserved hospital-level heterogeneity is an important driver of outcomes in our setting and b.) given the similar response to both payments and charges, that hospital fixed effects may in fact control for mean differences between charges and payments.  However, we emphasize the importance of payment data with respect to the precision and measurement of hospital-insurer bargaining, noting the lack of statistical significance in our model of log charges presented in Table 3.

\section{Discussion}
\label{sec:Discussion}


To more formally examine the presence of cost-shifting in a bargaining context, we embed the hospital cost-shifting model from \cite{dranove1988} in a hospital-insurer bargaining model similar to that in \cite{ho2016} (HL), \cite{gowrisankaran2015}, \cite{lewis2015}, and \cite{dor2004}. Specifically, we consider a non-profit hospital whose objective is to maximize a function of profits and quantity of care provided, denoted by
\begin{equation}
 U\left( \pi_{j} = \sum_{i=1}^{N_{j}} \pi_{i,j}^{h} + \pi_{g,j}^{h}, \sum_{i=1}^{N_{j}} D_{i,j}^{h}, D_{g,j}^{h} \right),
\label{eqn:nfp_objective}
\end{equation}
where $\pi_{j}$ denotes total profits for hospital $j$ and $D_{i,j}^{h}$ denotes hospital demand from insurer $i$. Following HL, we assume $$\pi_{i,j}^{h}=D_{i,j}^{h}(p_{i,j}-c_{i}),$$ where $p_{i,j}$ denotes the negotiated payment between insurer $i$ and hospital $j$. We also follow HL in assuming that patients are ``unaware or unable to determine their [financial] liability prior to choosing their provider.'' In other words, the negotiated payment $p_{i,j}$ does not affect demand for a specific hospital. The subscript $g$ denotes public (or government) insurers, for which the payment is administratively set at $p_{g}$. Finally, again following HL, we assume that profits for insurer $i$ are
\begin{equation}
\pi_{i}^{M} = D_{i} \left( \theta_{i} - \eta_{i} \right) - \sum_{j=1}^{N_{i}} D_{i,j}^{h} p_{i,j},
\label{eqn:ins_profit}
\end{equation}
where $D_{i}$ denotes the number of enrollees for insurer $i$, $\theta_{i}$ denotes the insurer's premiums, $\eta_{i}$ denotes insurer costs per-enrollee other than inpatient hospital care, and $D_{i,j}^{h} p_{i,j}$ reflects payments to hospitals for care provided to the insurer's enrollees.

The negotiated price between hospital $j$ and insurer $i$ is such that
\begin{equation}
 p_{ij}= \arg \max_{p_{ij}} \left(\triangle U_{j} \right)^{b_{j}} \times \left(\triangle \pi^{M}_{i} \right)^{1-b_{j}},
 \label{eqn:neg_price}
\end{equation}
where $\triangle U_{j}$ denotes the change in hospital $i$'s utility from reaching an agreement with insurer $i$, and similarly $\triangle \pi^{M}_{i}$ denotes the change in insurer profits from an agreement with hospital $i$. $b_{j}$ denotes the bargaining weight of hospital $j$, expressed as the weight to which the hospital's payoffs are given in the overall net value.

The first order condition for equation \ref{eqn:neg_price} can be simplified to
\begin{equation}
 b_{j} \triangle \pi_{i}^{M} \pderiv{U_{j}}{\pi_{ij}^{h}} - (1-b_{j}) \triangle U_{j} = 0.
\label{eqn:price_foc}
\end{equation}
Applying the implicit function theorem yields the relevant comparative static:
\begin{equation}
\deriv{p_{ij}}{p_{g}} = \frac{- b_{j} \triangle \pi_{i}^{M} \pderiv{^{2}U_{j}}{\pi_{j}^{2}}D_{g}^{h}}{D_{ij}^{h}\left(b_{j} D_{ij}^{h} \pderiv{^{2}U_{j}}{\pi_{j}^{2}} - (1-b_{j}) \pderiv{U_{j}}{\pi_{j}} \right)}.
\label{eqn:comp_static}
\end{equation}

We can see immediately from equation \ref{eqn:comp_static} that $\deriv{p_{ij}}{p_{g}}<0$ whenever $\pderiv{^{2}U_{j}}{\pi_{j}^{2}}<0$. Hospitals must therefore have some diminishing marginal utility of profits for cost-shifting to occur. Interestingly, this result exists without hospitals deriving utility directly from quantity of care provided, which is necessary for cost-shifting to occur in \cite{dranove1988}.  Practically, equation \ref{eqn:comp_static} suggests that hospitals will be more likely to cost-shift if they have some relative market power, where the insurer is heavily dependent on the hospital but where the hospital does not receive a large number of patients from the insurer. 


Equation \ref{eqn:comp_static} suggests that a hospital's incentive to cost-shift is larger as the number of public-payer patients increases, $D_{g}^{h}$, which stands in contrast to our results presented in Table 6 and to those of \cite{wu2010}.  
However, the incentive to cost-shift is increasing in the hospital's bargaining weight, $b_{j}$, provided the hospital is not ``too'' risk-averse.\footnote{Formally, $D_{ij}^{h}\frac{U''(\cdot)}{U'(\cdot)}>-1$ is a sufficient (but not necessary) condition for $\deriv{p_{ij}}{p_{g}}$ to be decreasing in $b_{j}$.} A more general model would endogenize the bargaining weight to reflect local market characteristics.  We argue that a hospital's bargaining power should be increasing in the number of patients insured by a given insurer, which may reconcile our empirical results with the theory.  We leave this as an opening for future work.  




\section{Conclusion}
\label{sec:Conclusion}
This paper uses novel payment data from a large, multi-payer database to investigate the extent to which hospitals, faced with public-sector reimbursement cuts, compensate by negotiating for higher payments from private insurers.  We use variation in reimbursements generated by two cost-containment policies within the ACA - the hospital readmissions reduction program and the hospital value base purchasing program - to estimate the role of a net reimbursement reduction on average hospital payments over a fiscal year.  Our results suggest support for a modest degree of ``cost-shifting''.  The change in payments for hospitals facing reimbursement reduction was 1.6$\%$ higher than that for non-penalized hospitals.  This result was driven by non-profit hospitals and those with small shares of public patients.  We find no significant effect in for-profit hospitals.  An open question remains how cost-shifting behavior varies by \textit{relative} market share of hospital - that is, the market share of a hospital relative to that of the local insurance market.  A potentially important addition to this literature would be an estimated structural bargaining model that accounts for hospital, insurer, and insured incentives and that allows for endogenous bargaining power.



\newpage
\bibliographystyle{authordate1}
\bibliography{BibTeX_Library}


\clearpage
\newpage
\appendix
\section{Tables}
\label{app:tables}

\newsavebox{\gfxbox}
\savebox{\gfxbox}{
\begin{tabular}{cccccccc}
\hline \hline
%\multicolumn{9}{c}{}\\
Fiscal & Sample 		&  Payment $\$$					& Public  	   & Medicare   & Medicaid  		& Other & Net \\
Year   &  Size    		&  Mean (St. Dev.) 				& Share      & Share       	& Share        	& Share & Penalty \\
 \hline
2010 & 1,644			& 	10,250.60   (8,940.35)	& 0.48   &   0.34   &   0.14    &  0.52  & 0.00  \\
2011 & 1,644 		& 	10,420.89   (4,965.80)		& 0.47   &   0.34    &  0.13    &  0.53  & 0.00   \\
2012 & 	1,644 		& 	10,147.18   (4,468.36) 	&  0.46  &    0.33   &   0.13   &   0.54  & 0.30   \\
2013 & 	1,644		& 	10,118.09   (6,370.39)	& 0.45   &   0.33    &  0.12   &   0.55  & 0.74  \\
2014 & 	1,644 	&	10,343.40   (4,742.95)		& 0.45  &    0.33   &   0.12   &   0.55 & 0.76 \\
2015 &    1,644 & 10,782.29   (5,670.77)		&	0.45  &    0.33   &   0.12   &   0.55& 0.80 \\
\hline
Total & 	9,864		& 10,338.67   (6,083.84)	&	0.45  &    0.33   &   0.12   &   0.55& 0.42 \\
\hline
\end{tabular}
}
\setlength{\captionmargin}{.5 \textwidth} \addtolength{\captionmargin}{-.5\wd\gfxbox}
\begin{table}[!h]
\centering
\caption{Characterization of Research Sample over Time}
\label{tab:summarystats}
\usebox{\gfxbox}
\par
\begin{minipage}{\wd\gfxbox}
\footnotesize
Notes: Balanced panel of hospitals over time between 2010 and 2015.  Payment represents the mean dollar amount paid to a hospital in a year over all acute care admissions.  Share variables are measured at the hospital level.  Penalty is a binary variable for whether the combination of HRRP and HVBP resulted in a net reimbursement reduction.
\end{minipage}
\end{table}



\newpage
\savebox{\gfxbox}{
\begin{tabular}{lccc}
\hline \hline
Variable 	& Never 				& Ever  				&   	  \\
		   		&  Penalized    		& Penalized			&    p-value   				\\
 \hline
Log(Payment)					& 9.323			& 9.299  			& 0.040\\
System  Membership      	&       0.669    	&     0.765   	&     0.000\\
Non-profit     &       0.839    &    0.706    &     0.000\\
Log(Case Mix Index)        &       0.422      &   0.446      &   0.000\\
\multicolumn{4}{l}{Local Hospital}\\
\hspace{0.05in} Monopoly  &      0.150    &     0.113    &     0.000 \\
\hspace{0.05in} Duopoly    &     0.226    &     0.154   &      0.000\\
\hspace{0.05in} Triopoly    &    0.112     &     0.108     &     0.655\\
\multicolumn{4}{l}{Market Share}\\
\hspace{0.05in} Medicare  &      0.347  &       0.329    &     0.000\\
\hspace{0.05in} Medicaid    &     0.117      &    0.126    &     0.004\\
\hspace{0.05in} Public      &    0.465     &    0.456     &    0.016\\
\hspace{0.05in} Other     &      0.535    &     0.544    &     0.016\\
\hspace{0.05in} Profit Index     &         0.581    &    0.604    &    0.000\\
Total Pop. (1000s)     &     710.362   &  1166.051   &     0.000 \\
\multicolumn{4}{l}{County Age Distribution}\\
\hspace{0.05in}[18, 34]      &    0.240   &     0.239    &    0.406\\
\hspace{0.05in}[35, 64]   &      0.396    &    0.392   &     0.000\\
\hspace{0.05in}>65    &   0.132    &    0.131     &   0.189\\
\multicolumn{4}{l}{County Race Distribution}\\
\hspace{0.05in}White     &     0.743   &     0.733    &    0.026\\
\hspace{0.05in}Black    &     0.151  &      0.137   &     0.000\\
\multicolumn{4}{l}{County Income Distribution}\\
\hspace{0.05in}<$\$$50k   &    0.182   &     0.180      &  0.000\\
\hspace{0.05in}[$\$$50k, 75k]   &   0.126   &     0.123    &    0.000\\
\hspace{0.05in}[$\$$100k, 150k]       & 0.138     &   0.132  &      0.000\\
\hspace{0.05in}>$\$$150k   &    0.108    &    0.101 &       0.000\\
\multicolumn{4}{l}{County Education Distribution}\\
\hspace{0.05in}High School Only   &     0.272   &     0.272   &     0.924\\
\hspace{0.05in}Bachelor's Only      &     0.196    &    0.190   &     0.000\\
\hline
\end{tabular}
}
\setlength{\captionmargin}{.5 \textwidth} \addtolength{\captionmargin}{-.5\wd\gfxbox}
\begin{table}[!h]
\centering
\caption{Hospital Characteristics by Penalties}
\label{tab:bypenalty}
\usebox{\gfxbox}
\par
\begin{minipage}{\wd\gfxbox}
\footnotesize
Notes: Summary statistics are split by whether a hospital is ever observed to receive a net penalty in 2012-2015, or both. Payment represents the mean dollar amount paid to a hospital in a year over all acute care admissions.   County level characteristics are from the American Community Survey.
\end{minipage}
\end{table}

\newpage
\savebox{\gfxbox}{
\scriptsize
\begin{tabular}{lllllll}
\hline	
\hline
 			& Log 				& Log				& Log Medicaid 	   	& Log Medicare   		& Log Private  			& Profit  \\
			& Payment		& Charge			& Discharges      		& Discharges       	& Discharges        	& Index  \\
\hline
Penalty  										&	0.016***	&	0.010	&	-0.037*	&	-0.022***	&	-0.001	&	0.002	\\
													&	(0.005)	&	(0.008)	&	(0.020)	&	(0.006)&	(0.010)	&	(0.001)	\\
\multicolumn{7}{l}{Hospital Characteristics}\\
\hspace{0.1in} Market Power		&	-0.003	&	-0.048**	&	0.197***	&	0.290***	&	0.379***	&	-0.002	\\
\hspace{0.15in} Medium				&	(0.011)	&	(0.019)	&	(0.051)	&	(0.039)	&	(0.047)	&	(0.004)	\\
\hspace{0.1in} Market Power		&	-0.016	&	-0.099***	&	0.277***	&	0.479***	&	0.626***	&	0.001	\\
\hspace{0.15in} High					&	(0.014)	&	(0.030)	&	(0.073)	&	(0.047)	&	(0.059)	&	(0.005)	\\
\hspace{0.1in} Large Market		&	-0.061**	&	-0.088***	&	-0.051	&	0.070***	&	0.206***	&	0.003	\\
													&	(0.031)	&	(0.023)	&	(0.046)	&	(0.018)	&	(0.038)	&	(0.003)	\\
\hspace{0.1in}Teaching Type 1	&	-0.016	&	-0.066**	&	-0.039	&	-0.019	&	-0.006	&	0.000	\\
													&	(0.012)	&	(0.028)	&	(0.038)	&	(0.014)	&	(0.020)	&	(0.003)	\\
\hspace{0.1in}Teaching Type 2 	&	0.006	&	-0.004	&	-0.001	&	0.000	&	0.002	&	-0.002	\\
													&	(0.006)	&	(0.010)	&	(0.025)	&	(0.009)	&	(0.011)	&	(0.002)	\\
\hspace{0.1in}System					&	0.032*	&	-0.004	&	-0.057	&	-0.051***	&	-0.064***	&	0.005*	\\
													&	(0.017)	&	(0.018)	&	(0.037)	&	(0.016)	&	(0.017)	&	(0.003)	\\
\hspace{0.1in}Non-profit				&	0.020	&	-0.001	&	0.101*	&	0.032	&	0.015	&	0.000	\\
													&	(0.024)	& (0.026)	&	(0.054)	&	(0.026)	&	(0.029)	&	(0.002)	\\
\multicolumn{7}{l}{County Age Share}\\
\hspace{0.1in}[18,34]				&	-1.085*	&	1.133	&	3.087	&	-2.656***	&	-0.737	&	-0.213	\\
													&	(0.635)	&	(0.727)	&	(2.348)	&	(0.874)	&	(0.969)	&	(0.178)	\\
\hspace{0.1in}[35,64]				&	-0.050	&	1.572	&	2.503	&	-3.450***	&	0.017	&	-0.269	\\
													&	(0.864)	&	(1.016)	&	(2.744)	&	(1.100)	&	(1.228)	&	(0.229)	\\
\hspace{0.1in} >64	&	-0.658	&	0.261	&	-1.715	&	-0.266	&	-1.400	&	-0.032	\\
													& (0.771)	&	(1.091)	&	(2.441)	&	(1.047)	&	(1.184)	&	(0.178)	\\
\multicolumn{7}{l}{County Share in Race Group}\\
\hspace{0.1in} Share White			&	-0.381**	&	0.092	&	-1.006	&	-0.115	&	0.497	&	-0.014	\\
													&	(0.187)	&	(0.270)	&	(0.673)	&	(0.233)	&	(0.314)	&	(0.050)	\\
\hspace{0.1in} Share Black			&	-0.192	&	-0.073	&	-0.732	&	0.296	&	0.599	&	-0.054	\\
													&	(0.208)	&	(0.392)	&	(0.855)	&	(0.302)	&	(0.658)	&	(0.078)	\\
\multicolumn{7}{l}{County Share in Income Group}\\
\hspace{0.1in}	50k-75k 				&	-0.398	&	-1.540**	&	1.008	&	-0.630	&	0.212	&	-0.165	\\
													&	(0.361)	&	(0.606)	&	(1.354)	&	(0.514)	&	(0.729)	&	(0.112)	\\
\hspace{0.1in} 75k-100k			&	-0.542	&	0.918	&	-0.132	&	-0.499	&	-0.187	&	-0.153	\\
													&	(0.445)	&	(0.721)	&	(1.657)	&	(0.560)	&	(0.725)	&	(0.108)	\\
\hspace{0.1in} 100k-150k			&	-0.935**	&	-0.334	&	-0.805	&	-0.154	&	-0.111	&	0.019	\\
													&	(0.420)	&	(0.630)	&	(1.455)	&	(0.552)	&	(0.708)	&	(0.102)	\\
\hspace{0.1in}$>$150k				&	0.846**	&	1.242**	&	1.880	&	1.307***	&	-1.556**	&	0.019	\\
													&	(0.372)	&	(0.562)	&	(1.291)	&	(0.463)	&	(0.617)	&	(0.101)	\\
\hline
\end{tabular}
}
\setlength{\captionmargin}{.5 \textwidth} \addtolength{\captionmargin}{-.5\wd\gfxbox}
\begin{table}[!h]
\centering
\caption{Baseline Results}
\label{tab:baselineresults}
\usebox{\gfxbox}
\par
\begin{minipage}{\wd\gfxbox}
\footnotesize
Notes: $n=9,425$.  All regressions include hospital and year fixed effects and other hospital level controls include bed count and labor force.  Market power variables are constructed as the overall county market share tercile.  Large market is a binary variable for a hospital in top half of the market size distribution.  In cases in which independent variables are missing, we recode them and control for missing variable binary variables to ensure a balanced panel.  Standard errors are clustered at the hospital level.  *** p-value<0.01, ** p-value<0.05, * p-value<0.1.
\end{minipage}
\end{table}

\newpage
\savebox{\gfxbox}{
\begin{tabular}{lllllll}
\hline	
\hline
 			& Log 				& Log				& Medicaid 	   	& Medicare   		& Private  			& Profit  \\
			& Payment		& Charge			& Discharges      		& Discharges       	& Discharges        	& Index  \\
	\hline
\multicolumn{7}{c}{Non-profit Hospitals}\\
\hline
Net Penalty & 0.016***	&	0.008	&	-0.035	&	-0.026***	&	-0.009	&	0.002	\\
					& (0.005)	&	(0.008)	&	(0.023)	&	(0.007)	&	(0.011)	&	(0.002)	\\
	\hline
\multicolumn{7}{c}{For-profit Hospitals}\\													
\hline													
Net Penalty & 0.013	&	0.014	&	-0.037	&	-0.005	&	0.035*	&	0.002	\\
				& (0.012)	&	(0.019)	&	(0.048)	&	(0.017)	&	(0.020)	&	(0.003)	\\
\hline
\end{tabular}
}
\setlength{\captionmargin}{.5 \textwidth} \addtolength{\captionmargin}{-.5\wd\gfxbox}
\begin{table}[!h]
\centering
\caption{Results by Profit Status}
\label{tab:byprofit}
\usebox{\gfxbox}
\par
\begin{minipage}{\wd\gfxbox}
\footnotesize
Notes: All regressions include hospital and year fixed effects.  Further controls include those in our baseline specification for mean payments.  In cases in which independent variables are missing, we recode them and control for missing variable binary variables to ensure a balanced panel.  Standard errors are clustered at the hospital level.  *** p-value<0.01, ** p-value<0.05, * p-value<0.1.
\end{minipage}
\end{table}




\newpage
\savebox{\gfxbox}{
\begin{tabular}{ccccccc}
\hline	
\hline
 							& Nervous  				& Respiratory  	   	& Circulatory    & Musculoskeletal   		& Labor and & Neonatal \\
							&  System						&  System      	&  System     	&  System        		& Delivery   &	\\
\hline
Net Penalty & 0.022**	&	-0.002	&	0.025***	&	0.001	&	0.000	&	0.021**	\\
& (0.010)	&	(0.010)	&	(0.007)	&	(0.007)	&	(0.005)	&	(0.009)	\\
\hline
n& 1,638	&	2,027	&	3,105	&	3,516	&	5,890	&	3,602.00	\\
Mean &13,734.44	&	11,963.87	&	13,136.28	&	12,970.62	&	11,409.36	&	8,969.000	\\
\hline
\end{tabular}
}
\setlength{\captionmargin}{.5 \textwidth} \addtolength{\captionmargin}{-.5\wd\gfxbox}
\begin{table}[!h]
\centering
\caption{Log Payments for Condition Specific Admissions}
\label{tab:eachcondition}
\usebox{\gfxbox}
\par
\begin{minipage}{\wd\gfxbox}
\footnotesize
Notes: All regressions include hospital and year fixed effects.  The dependent variable is the log of average payments for each condition.  Further controls include those in our baseline specification for mean payments.  The dependent variable in each column is the log of the payment for the associated acute care admission.  In cases in which independent variables are missing, we recode them and control for missing variable binary variables to ensure a balanced panel.  Standard errors are clustered at the hospital level.  We restrict the sample to include at least 25 admissions per hospital per year.  *** p-value<0.01, ** p-value<0.05, * p-value<0.1.
\end{minipage}
\end{table}



%\newpage
%\savebox{\gfxbox}{
%\footnotesize
%\begin{tabular}{lllllll}
%\hline	
%\hline
% 			& Log 				& Log				& Medicaid 	   	& Medicare   		& Private  			& Profit  \\
%			& Payment		& Charge			& Discharges      		& Discharges       	& Discharges        	& Index  \\
%\hline
%\multicolumn{7}{c}{Small Markets} \\
%\hline
%Net Penalty	&	0.003	&	-0.007	&	-0.049	&	-0.034*	&	-0.029	&	0.002	\\
%	&	(0.010)	&	(0.015)	&	(0.043)	&	(0.018)	&	(0.022)	&	(0.003)	\\
%\hspace{0.1in} *Med. Mkt. Share 	&	-0.004	&	0.030*	&	0.031	&	0.008	&	0.035	&	-0.003	\\
%	&	(0.012)	&	(0.018)	&	(0.046)	&	(0.020)	&	(0.024)	&	(0.004)	\\
%\hspace{0.1in} *High Mkt. Share 	&	0.009	&	-0.004	&	-0.002	&	0.014	&	0.030	&	0.000	\\
%	&	(0.012)	&	(0.019	)&	(0.051)	&	(0.022)	&	(0.024)	&	(0.004)	\\	
%Market Share	&	0.009	&	-0.056**	&	0.167**	&	0.216***	&	0.321***	&	0.004	\\
%\hspace{0.1in} Medium	&	(0.011)	&	(0.026)	&	(0.080)	&	(0.038)	&	(0.049)	&	(0.004)	\\
%Market Share	&	0.012	&	-0.047	&	0.294***	&	0.288***	&	0.428***	&	0.004	\\
%\hspace{0.1in} High	&	(0.017)	&	(0.035)	&	(0.102)	&	(0.054)	&	(0.063)	&	(0.006)	\\
%\hline
%\multicolumn{7}{c}{Large Markets} \\
%\hline
%Net Penalty	&	0.036***	&	-0.001	&	-0.072*	&	-0.022	&	0.010	&	0.000	\\
%	&	(0.012)	&	(0.018)	&	(0.040)	&	(0.018)	&	(0.025)	&	(0.003)	\\
%\hspace{0.1in} *Med. Mkt. Share 	&	-0.013	&	0.018	&	0.046	&	0.018	&	0.019	&	0.006*	\\
%	&	(0.013)	&	(0.018)	&	(0.038)	&	(0.018)	&	(0.022)	&	(0.003)	\\
%\hspace{0.1in} *High Mkt. Share	&	-0.022*	&	0.038*	&	0.051	&	0.024	&	-0.003	&	0.002	\\
%	&	(0.013)	&	(0.020)	&	(0.039)	&	(0.019)	&	(0.025)	&	(0.003)	\\	
%Market Share	&	-0.001	&	-0.049**	&	0.364***	&	0.360***	&	0.395***	&	-0.003	\\
%\hspace{0.1in} Medium	&	(0.016)	&	(0.021)	&	(0.057)	&	(0.047)	&	(0.060)	&	(0.005)	\\
%Market Share	&	0.009	&	-0.138***	&	0.515***	&	0.513***	&	0.688***	&	-0.005	\\
%High 	&	(0.019)	&	(0.032)	&	(0.082)	&	(0.065)	&	(0.097)	&	(0.007)	\\
%
%\hline
%\end{tabular}
%}
%\setlength{\captionmargin}{.5 \textwidth} \addtolength{\captionmargin}{-.5\wd\gfxbox}
%\begin{table}[!h]
%\centering
%\caption{Triple Differences by Market Share}
%\label{tab:bymktshare}
%\usebox{\gfxbox}
%\par
%\begin{minipage}{\wd\gfxbox}
%\footnotesize
%Notes: All regressions include hospital and year fixed effects. Further controls include those in our baseline specification for mean payments.  In cases in which independent variables are missing, we recode them and control for missing variable binary variables to ensure a balanced panel.  We split the sample by market size because of the strong negative correlation between market size and market share. Standard errors are clustered at the hospital level.  *** p-value<0.01, ** p-value<0.05, * p-value<0.1.
%\end{minipage}
%\end{table}




\newpage
\savebox{\gfxbox}{
\begin{tabular}{lll}
\hline	
 		& Log (Payment) & Log (Charge)  				 \\
\hline
Penalty &	0.039***	&	0.045***	\\
&	(0.010)	&	(0.012)	\\
\hspace{0.1in}* Public Share 2 &	-0.016	&	-0.016	\\
&	(0.011)	&	(0.013)	\\
\hspace{0.1in}* Public Share 3&	-0.030**	&	-0.049***	\\
&	(0.012)	&	(0.014)	\\
\hspace{0.1in}* Public Share 4&	-0.048***	&	-0.070***	\\
&	(0.012)	&	(0.016)	\\
Public Share 2 &	0.009	&	0.046***	\\
&	(0.009)	&	(0.013)	\\
Public Share 3 &	0.019*	&	0.085***	\\
&	(0.011)	&	(0.015)	\\
Public Share 4 &	0.026**	&	0.146***	\\
&	(0.012)	&	(0.018)	\\
\hline
\end{tabular}
}
\setlength{\captionmargin}{.5 \textwidth} \addtolength{\captionmargin}{-.5\wd\gfxbox}
\begin{table}[!h]
\centering
\caption{Triple Differences by Public Share}
\label{tab:publicshare}
\usebox{\gfxbox}
\par
\begin{minipage}{\wd\gfxbox}
\footnotesize
Notes: All regressions include hospital and year fixed effects.  Further controls include those in our baseline specification for mean payments.  The share of a hospital's patients insured by the public sector is broked into quartiles and interacted with penalty variables.  In cases in which independent variables are missing, we recode them and control for missing variable binary variables to ensure a balanced panel.  Standard errors are clustered at the hospital level.  We restrict the sample to include at least 25 admissions per hospital per year.  *** p-value<0.01, ** p-value<0.05, * p-value<0.1.
\end{minipage}
\end{table}




\newpage
\savebox{\gfxbox}{
\footnotesize
\begin{tabular}{lllllll}
\hline	
\hline
 			& Log 				& Log				& Medicaid 	   	& Medicare   		& Private  			& Profit  \\
			& Payment		& Charge			& Discharges      		& Discharges       	& Discharges        	& Index  \\
			\hline
\multicolumn{7}{c}{Non-Profit Hospitals with Penalty Specific Trends} 		\\	
			\hline										
Penalty 	&	0.012**	&	0.012	&	-0.029	&	-0.021***	&	-0.013	&	0.003	\\
				&	(0.005)	&	(0.009)	&	(0.025)	&	(0.007)&	(0.012)	&	(0.002)	\\
				\hline
\multicolumn{7}{c}{For-Profit Hospitals with Penalty Specific Trends} 		\\		
			\hline									
Penalty &	0.005	&	0.027	&	-0.028	&	-0.018	&	0.022	&	0.002	\\
&	(0.013)	&	(0.021)	&	(0.048)	&	(0.017)	&	(0.021)	&	(0.003)	\\
				\hline
\multicolumn{7}{c}{Hospital, Year, and County Fixed Effects} 			\\		
			\hline								
Penalty &	0.017***	&	0.009	&	-0.040*	&	-0.021***	&	0.000	&	0.002	\\
&	(0.005)	&	(0.008)	&	(0.021)	&	(0.007)	&	(0.010)	&	(0.001)	\\
				\hline
\multicolumn{7}{c}{Controlling for Medicaid Expansion States} 			\\			
			\hline							
Penalty &	0.016***	&	0.010	&	-0.037*	&	-0.022***	&	-0.001	&	0.002	\\
&	(0.005)	&	(0.008)	&	(0.020)	&	(0.006)	&	(0.010)	&	(0.001)	\\								
				\hline
\multicolumn{7}{c}{Controlling for Overall HCAHPS Hospital Rating} 		\\		
			\hline									
Penalty &	0.016***	&	0.009	&	-0.036*	&	-0.022***	&	-0.001	&	0.002	\\
&	(0.005)	&	(0.008)	&	(0.020)	&	(0.006)	&	(0.010)	&	(0.001)	\\
	\hline
\multicolumn{7}{c}{Dropping Fiscal 2012} 		\\		
			\hline									
Penalty &	0.015***	&	0.010	&	-0.036*	&	-0.022***	&	-0.003	&	0.001	\\
&	(0.005)	&	(0.008)	&	(0.022)	&	(0.007)	&	(0.011)	&	(0.002)	\\
	\hline	
	\multicolumn{7}{c}{Year Fixed Effects Only} 			\\		
			\hline								
Penalty	&	-0.068***	&	-0.057***	&	0.229***	&	0.098***	&	0.070***	&	0.004	\\
	&	(0.014)	&	(0.017)	&	(0.042)	&	(0.024)	&	(0.021)	&	(0.005)	\\
				\hline
\end{tabular}
}
\setlength{\captionmargin}{.5 \textwidth} \addtolength{\captionmargin}{-.5\wd\gfxbox}
\begin{table}[!h]
\centering
\caption{Robustness Checks}
\label{tab:bymktshare}
\usebox{\gfxbox}
\par
\begin{minipage}{\wd\gfxbox}
\footnotesize
Notes: Further controls include those in our baseline specification for mean payments.  In cases in which independent variables are missing, we recode them and control for missing variable binary variables to ensure a balanced panel.  Standard errors are clustered at the hospital level.  *** p-value<0.01, ** p-value<0.05, * p-value<0.1.
\end{minipage}
\end{table}



\end{document}
